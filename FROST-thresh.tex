\documentclass[11pt]{article}
\usepackage{graphicx} % Required for inserting images
\usepackage{amsmath,amsfonts,amssymb,amsthm}
\usepackage{enumerate}
\usepackage[hidelinks]{hyperref}
\usepackage[dvipsnames]{xcolor} % Adds colors for todo macros
\usepackage{cleveref}
\usepackage{aliascnt}
\usepackage{multicol}
\usepackage{tikz-cd}
\usepackage{titling}

 \renewcommand{\thefootnote}{\fnsymbol{footnote}}

\theoremstyle{definition}

\newtheorem{theorem}{Theorem}[section]
\newcommand{\theoremqed}{\hfill \qedsymbol}

\newaliascnt{lemma}{theorem}
\newtheorem{lemma}[lemma]{Lemma}
\aliascntresetthe{lemma}
\crefname{lemma}{Lemma}{Lemmas}

\newaliascnt{cor}{theorem}
\newtheorem{cor}[cor]{Corollary}
\aliascntresetthe{cor}
\crefname{cor}{Corollary}{Corollaries}
\AtEndEnvironment{lemma}{\theoremqed}

\newaliascnt{definition}{theorem}
\newtheorem{definition}[definition]{Definition}
\aliascntresetthe{definition}
\crefname{definition}{Definition}{Definitions}

\newaliascnt{example}{theorem}
\newtheorem{example}[example]{Example}
\aliascntresetthe{example}
\crefname{example}{Example}{Examples}

\newaliascnt{remark}{theorem}
\newtheorem{remark}[remark]{Remark}
\aliascntresetthe{remark}
\crefname{remark}{Remark}{Remarks}


\newcommand{\N}{\bbn}
\newcommand{\R}{\mathbb{R}}
\newcommand{\G}{\mathbb{G}}
\newcommand{\sk}{\texttt{sk}}
\newcommand{\sn}{\texttt{sn}}
\newcommand{\pk}{\texttt{pk}}
\newcommand{\vk}{\texttt{vk}}
\newcommand{\pn}{\texttt{pn}}
\newcommand{\lt}{\texttt{lt}}
\newcommand{\tlt}{\texttt{tlt}}
\newcommand{\sck}{\texttt{sck}}
\newcommand{\pck}{\texttt{pck}}
\newcommand{\tvk}{\texttt{tvk}}
\newcommand{\ring}{\texttt{ring}}
\newcommand{\SK}{\underline{\texttt{sk}}}
\newcommand{\VK}{\underline{\texttt{vk}}}
\newcommand{\TVK}{\underline{\texttt{tvk}}}
\newcommand{\LT}{\underline{\texttt{lt}}}
\newcommand{\dst}{\texttt{dst}}
\newcommand{\ch}{\texttt{ch}}
\newcommand{\resp}{\texttt{resp}}
\newcommand{\sig}{\texttt{sig}}
\newcommand{\psig}{\texttt{psig}}
\newcommand{\psigs}{\underline{\texttt{psig}}}
\newcommand{\seed}{\texttt{seed}}
\newcommand{\com}{\texttt{com}}
\newcommand{\bitstrings}{\left\{0,1\right\}^*}
\newcommand{\Zq}{\mathbb{Z}_q}
\newcommand{\bbn}{\mathbb{N}}
\newcommand{\secpar}{\lambda}
\newcommand{\rng}{F_{\texttt{PRNG}}}
\newcommand{\params}{\texttt{pars}}
\newcommand{\msg}{\texttt{msg}}
\newcommand{\setup}{\texttt{PGen}}
\newcommand{\setupI}{(\secpar)}
\newcommand{\setupO}{\params_\secpar}
\newcommand{\setupIO}{\setup\setupI\to\setupO}
\newcommand{\keygen}{\texttt{KGen}}
\newcommand{\keygenI}{(\setupO,n,r)}
\newcommand{\keygenIshort}{(n,r)}
\newcommand{\keygenO}{(\tvk,\VK,\SK)}
\newcommand{\keygenObase}{(\tvk, \VK, \SK)}
\newcommand{\keygenIO}{\keygen\keygenI\to\keygenO}
\newcommand{\keygenIObase}{\keygen\keygenI\to\keygenObase \in \mathcal{TVK} \times \mathcal{VK}^n \times \mathcal{SK}^n}
\newcommand{\preproc}{\texttt{PreProc}}
\newcommand{\preprocI}{(\setupO,\msg,\ring,\VK,\LT,\SK)}
\newcommand{\preprocO}{\com}
\newcommand{\preprocIO}{\preproc\preprocI\to\preprocO}
\newcommand{\sign}{\texttt{Sign}}
\newcommand{\aux}{\texttt{aux}}
\newcommand{\signI}{(\setupO,\msg,\ring,\VK,\LT,\sk)}
\newcommand{\signIbase}{(\setupO,\msg,\ring,\VK,\sk)}
\newcommand{\signO}{\psig}
\newcommand{\signIO}{\sign\signI\to\signO}
\newcommand{\signIObase}{\sign\signIbase\to\signO}
\newcommand{\combine}{\texttt{Combine}}
\newcommand{\PSIG}{\underline{\psig}}
\newcommand{\combineI}{(\setupO,\msg,\ring,\VK,\LT,\PSIG)}
\newcommand{\combineIbase}{(\setupO,\msg,\ring,\VK,\PSIG)}
\newcommand{\combineO}{\sig}
\newcommand{\combineIO}{\combine\combineI\to\combineO}
\newcommand{\combineIObase}{\combine\combineIbase\to\combineO}
\newcommand{\verify}{\texttt{Vf}}
\newcommand{\verifyI}{(\setupO,\msg,\ring,\sig)}
\newcommand{\verifyO}{b}
\newcommand{\verifyIO}{\verify\verifyI\to\verifyO}
\newcommand{\verifyIObase}{\verify\verifyI\to\verifyO}
\newcommand{\link}{\texttt{Link}}
\newcommand{\linkI}{(\setupO,\sig,\sig^\prime)}
\newcommand{\linkO}{\verifyO}
\newcommand{\linkIO}{\link\linkI\to\linkO}
\newcommand{\linkIObase}{\link\linkI\to\linkO}
\newcommand{\verifyshare}{\texttt{VfSh}}
\newcommand{\verifyshareI}{(\setupO,\msg,\ring,\VK,\psig)}
\newcommand{\verifyshareO}{\verifyO}
\newcommand{\verifyshareIO}{\verifyshare\verifyshareI\to\verifyshareO}
 \newcommand{\verifyshareIObase}{\verifyshare\verifyshareI\to\verifyshareO}
\newcommand{\chk}{\texttt{chk}}
\newcommand{\polysecpar}{O(\text{poly}(\secpar))}
\newcommand{\negl}{\text{negl}(\secpar)}
\newcommand{\FROST}{\texttt{FROST}}
\newcommand{\linkingtag}{\mathfrak{T}}
\newcommand{\sample}{\overset{\$}{\leftarrow}}
\newcommand{\fork}{\texttt{Fork}}
\newcommand{\corruptionOracle}{\mathcal{O}_{\texttt{corrupt}}}
\newcommand{\signingOracle}{\mathcal{O}_{\sign}}
\newcommand{\keyOracle}{\mathcal{O}_{\texttt{key}}}
\newcommand{\challengeKeySet}{\mathcal{L}_{\texttt{key}}}
\newcommand{\corruptedTotalKeys}{\mathcal{L}_{\texttt{corrupt}}^{\texttt{tot}}}
\newcommand{\corruptedKeyShareSet}{\mathcal{L}_{\texttt{corrupt}}^{\texttt{sh}}}
\newcommand{\corruptedKeySet}{\mathcal{L}_{\texttt{corrupt}}}
\newcommand{\signaturequery}{(\texttt{dst}_{j} \mid \mid \ring \mid \mid \tlt \mid \mid \underline{\widehat{Y}} \mid \mid \mathfrak{W} \mid \mid \underline{W} \mid \mid \underline{\mu} \mid \mid L_{j} \mid \mid R_{j} \mid \mid \msg)}
\newcommand{\signaturequerystar}{(\texttt{dst}_{j^*} \mid \mid \ring \mid \mid \tlt \mid \mid \underline{\widehat{Y}} \mid \mid \mathfrak{W} \mid \mid \underline{W} \mid \mid \underline{\mu} \mid \mid L_{j^*} \mid \mid R_{j^*} \mid \mid \msg)}

\newcommand{\subtitle}[1]{%
  \posttitle{%
    \par\end{center}
    \begin{center}\Large#1\end{center}
    \vskip0.5em}%
}


% From here down is Freeman's additions
% For new colors, look in the xcolor documentation
\newcommand{\bran}[1]{{\textcolor{Rhodamine}{#1}}}
\newcommand{\free}[1]{{\textcolor{Red}{#1}}}
\newcommand{\luke}[1]{{\textcolor{ForestGreen}{#1}}}
\newcommand{\rigo}[1]{{\textcolor{Blue}{#1}}}



% Used to format table in Rosetta Stone
\usepackage{booktabs,array}
\newcommand{\PreserveBackslash}[1]{\let\temp=\\#1\let\\=\temp}
\newcolumntype{C}[1]{>{\PreserveBackslash\centering}p{#1}}
\newcolumntype{R}[1]{>{\PreserveBackslash\raggedleft}p{#1}}
\newcolumntype{L}[1]{>{\PreserveBackslash\raggedright}p{#1}}


% Used to format the pseudocode blocks
\usepackage{caption,subcaption}
\newcommand{\Fp}{\mathbb{F}_p}
\newcommand{\rar}{\rightarrow}
\newcommand{\lar}{\leftarrow}
\newcommand{\lsamp}{\xleftarrow{\$}}


% Adds the oracle and game environments
\DeclareCaptionType[fileext=los,placement={!ht}]{oracle}
\DeclareCaptionType[fileext=los,placement={!ht}]{game}




%---------------- Used to add Josh Babb's codebase, Audit.tex ----------------%

% \usepackage[utf8]{inputenc}
% \usepackage[T1]{fontenc}
\usepackage{lmodern}                     % Improved font rendering
\usepackage{geometry}                    % Page geometry
\geometry{margin=1in}                    % Set margins to your preference
% \usepackage[hidelinks]{hyperref}         % Clickable links and refs
% \usepackage{graphicx}                    % For images
\usepackage{enumitem}                    % Better control of list environments
% \usepackage{xcolor}                      % Color text if needed
\usepackage{makeidx}                     % For creating an index
\makeindex                               % Initialize index creation
% \usepackage[style=alphabetic]{biblatex}  % BibLaTeX for references
% \addbibresource{references.bib}          % External .bib file for references
\usepackage[toc, acronym]{glossaries}    % Glossaries and acronyms 
% \makeglossaries                          % Initialize glossary creation
\usepackage{mdframed}                    % For framed boxes
\usepackage{tikz}
\usetikzlibrary{arrows.meta, positioning}


% \usepackage[style=long,nonumberlist,toc,xindy,acronym,nomain]{glossaries}
% \makeglossaries     




% Added this to allow for subsubsubsections
\usepackage{titlesec}


\newcounter{subsubsubsection}[subsubsection]
\renewcommand{\thesubsubsubsection}{\thesubsubsection.\arabic{subsubsubsection}}  % Number format

% Define the subsubsubsection command
\newcommand{\subsubsubsection}[1]{%
  \refstepcounter{subsubsubsection}  % Increment the counter
  \paragraph*{\thesubsubsubsection\ #1} % Format the section title
}


% -------------------- Glossary Entries --------------------%

    \newglossaryentry{monero-wallet (v0.1.0)}{
        name={\protect\texttt{monero-wallet (v0.1.0)}},
        description={A standard library crate, with the corresponding entry point at  \path{/wallet/src/lib.rs}. Handles all wallet functionality}
    }

    \newglossaryentry{monero-simple-request-rpc (v0.1.0)}{
        name={\protect\texttt{monero-simple-request-rpc (v0.1.0)}},
        description={A standard library crate, with the corresponding entry point at  \path{/rpc/simple-request/src/lib.rs}. Default RPC to avoid external dependences, e.g. reqwest}
    }
    
    \newglossaryentry{monero-rpc (v0.1.0)}{
        name={\protect\texttt{monero-rpc (v0.1.0)}},
        description={A standard library crate, with the corresponding entry point at \path{/rpc/src/lib.rs}, employing no file or directory modules. Handles RPC calls for interacting on the Monero network.}
    }

    \newglossaryentry{monero-serai (v0.1.4-alpha)}{
        name={\protect\texttt{monero-serai (v0.1.4-alpha)}},
        description={A standard library crate, with the corresponding entry point at \path{/src/lib.rs}. This is the overall transaction library.}
    }

    \newglossaryentry{monero-address (v0.1.0)}{
        name={\protect\texttt{monero-address (v0.1.0)}},
        description={A standard library crate, with the corresponding entry point at \path{/wallet/address/src/lib.rs}. Handles Monero addresses.}
    }

    \newglossaryentry{monero-borromean (v0.1.0)}{
        name={\protect\texttt{monero-borromean (v0.1.0)}},
        description={A standard library crate, with the corresponding entry point at \path{/ringct/borromean/src/lib.rs}. Employs no modules, and untested. Handles Borromean signatures and Borromean range proofs.}
    }

    \newglossaryentry{monero-bulletproofs (v0.1.0)}{
        name={\protect\texttt{monero-bulletproofs (v0.1.0)}},
        description={A standard library crate, with the corresponding entry point at \path{/ringct/bulletproofs/src/lib.rs}. Handles original bulletproofs and bulletproofs plus.}
    }

    \newglossaryentry{monero-clsag (v0.1.0)}{
        name={\protect\texttt{monero-clsag (v0.1.0)}},
        description={A standard library crate, with the corresponding entry point at \path{/ringct/clsag/src/lib.rs}. Handles CLSAG ring signatures and a FROST-like thresholdization.}
    }

    \newglossaryentry{monero-mlsag (v0.1.0)}{
        name={\protect\texttt{monero-mlsag (v0.1.0)}},
        description={A standard library crate, with the corresponding entry point at \path{/ringct/mlsag/src/lib.rs}. Employs no modules, and untested. Handles MLSAG ring signatures.}
    }

    \newglossaryentry{monero-primitives (v0.1.0)}{
        name={\protect\texttt{monero-primitives (v0.1.0)}},
        description={A standard library crate, with the corresponding entry point at \path{/primitives/src/lib.rs}.}
    }

    \newglossaryentry{monero-generators (v0.4.0)}{
        name={\protect\texttt{monero-generators (v0.4.0)}},
        description={A standard library crate, with the corresponding entry point at \path{/primitives/src/lib.rs}.  Handles hashing to elliptic curve group elements, and computing fixed generators for use in the Monero protocol}
    }

    \newglossaryentry{monero-io (v0.1.0)}{
        name={\protect\texttt{monero-io (v0.1.0)}},
        description={A standard library crate with entry point at \path{/io/src/lib.rs}. Employs no modules, and untested.  Handles reading and writing various data structures used in Monero protocol computations (e.g.\ bytes, scalars, group elements, lists whose entries are the same type)}
    }
    
    \newglossaryentry{decoys-module}{
        name={\protect\path{/wallet/src/decoys.rs}},
        description={The module handling decoys.}
    }

    \newglossaryentry{extra-module}{
        name={\protect\path{/wallet/src/extra.rs}},
        description={The module handling the extra field of transactions.}
    }

    \newglossaryentry{output-module}{
        name={\protect\path{/wallet/src/output.rs}},
        description={The module handling transaction outputs.}
    }

    \newglossaryentry{scan-module}{
        name={\protect\path{/wallet/src/scan.rs}},
        description={The module handling scanning.}
    }

    \newglossaryentry{view-pair-module}{
        name={\protect\path{/wallet/src/view_pair.rs}},
        description={The module handling the (public-spend, private-view) keys.}
    }

    \newglossaryentry{send-module}{
        name={\protect\path{/wallet/src/send/mod.rs}},
        description={The module handling the sending transactions.}
    }

    \newglossaryentry{block-module}{
        name={\protect\path{/src/block.rs}},
        description={The module handling blocks.}
    }

    \newglossaryentry{merkle-module}{
        name={\protect\path{/src/merkle.rs}},
        description={Module for handling Merkle trees}
    }

    \newglossaryentry{ring-signatures-module}{
        name={\protect\path{/src/ring_signatures.rs}},
        description={Module for handling ring signatures}
    }

    \newglossaryentry{ringct-module}{
        name={\protect\path{/src/ringct.rs}},
        description={Module for handling ring confidential transactions}
    }

    \newglossaryentry{transaction-module}{
        name={\protect\path{/src/transaction.rs}},
        description={Module for handling transactions}
    }

    \newglossaryentry{base58-module}{
        name={\protect\path{/wallet/address/src/base58check.rs}},
        description={Module for handling base58 enc/dec}
    }

    \newglossaryentry{monero-serai-entry-point}{
        name={\protect\path{/src/lib.rs}},
        description={Module for handling RPC calls for communicating on the Monero network.}.
    }
    
    \newglossaryentry{wallet-tests}{
        name={\protect\path{/wallet/src/tests/runner/mod.rs}},
        description={Testing module for \texttt{monero wallet (v0.1.0)}}
    }

    \newglossaryentry{wallet-entry-point}{
        name={\protect\path{/wallet/src/lib.rs}},
        description={The entry point to the \gls{monero-wallet (v0.1.0)} crate}
    }

    \newglossaryentry{monero-simple-request-rpc-entry-point}{
        name={\protect\path{/rpc/simple-request/src/lib.rs}},
        description={The entry point to the \gls{monero-simple-request-rpc (v0.1.0)} crate}
    }
    
    \newglossaryentry{monero-rpc-entry-point}{
        name={\protect\path{/rpc/src/lib.rs}},
        description={Module for handling RPC calls for communicating on the Monero network.}.
    }

    \newglossaryentry{monero-serai-tests}{
        name={\protect\path{/src/tests/mod.rs}},
        description={Testing module for handling \gls{monero-serai (v0.1.4-alpha)} tests}
    }

    \newglossaryentry{address-tests}{
        name={\protect\path{/wallet/address/src/tests.rs}},
        description={Testing module for \gls{monero-address (v0.1.0)} tests}
    }
    
    \newglossaryentry{borromean-entry-point}{
        name={\protect\path{/ringct/borromean/src/lib.rs}},
        description={Entry point to \gls{monero-borromean (v0.1.0)}}
    }
    
    \newglossaryentry{bulletproofs-entry-point}{
        name={\protect\path{/ringct/bulletproofs/src/lib.rs}},
        description={Entry point to \gls{monero-bulletproofs (v0.1.0)}}
    }
    
    \newglossaryentry{bp-batch-verifier-module}{
        name={\protect\path{/ringct/bulletproofs/src/batch_verifier.rs}},
        description={Module for handling batch verification of bulletproofs}
    }

    \newglossaryentry{bp-core-module}{
        name={\protect\path{/ringct/bulletproofs/src/core.rs}},
        description={Module for handling the core folding computation of bulletproofs.}
    }

    \newglossaryentry{bp-point-vector-module}{
        name={\protect\path{/ringct/bulletproofs/src/point_vector.rs}},
        description={Module for handling the vectors of group elements in bulletproofs.}
    }

    \newglossaryentry{bp-scalar-vector-module}{
        name={\protect\path{/ringct/bulletproofs/src/scalar_vector.rs}},
        description={Module for handling the vectors of field elements/scalars in bulletproofs.}
    }

    \newglossaryentry{bp-original-module}{
        name={\protect\path{/ringct/bulletproofs/src/original/mod.rs}},
        description={Module for handling the original bulletproofs}
    }

    \newglossaryentry{bp-plus-module}{
        name={\protect\path{/ringct/bulletproofs/src/plus/mod.rs}},
        description={Module for handling bulletproofs plus}
    }

    \newglossaryentry{bp-test-module}{
        name={\protect\path{/ringct/bulletproofs/src/plus/mod.rs}},
        description={Module for handling bulletproofs tests}
    }

    \newglossaryentry{monero-clsag-entry-point}{
        name={\protect\path{/ringct/clsag/src/lib.rs}},
        description={Entry point for \gls{monero-clsag (v0.1.0)}}
    }
   
    \newglossaryentry{clsag-multisig-module}{
        name={\protect\path{/ringct/clsag/src/multisig.rs}},
        description={Module for handling CLSAG signatures.}
    } 

    \newglossaryentry{clsag-tests}{
        name={\protect\path{/ringct/clsag/src/tests.rs}},
        description={Test module for CLSAG signatures.}
    } 

    \newglossaryentry{monero-mlsag-entry-point}{
        name={\protect\path{/ringct/clsag/src/tests.rs}},
        description={Entry point for \gls{monero-mlsag (v0.1.0)}}
    } 

    \newglossaryentry{monero-primitives-entry-point}{
        name={\protect\path{/primitives/src/lib.rs}},
        description={Entry point for \gls{monero-primitives (v0.1.0)}}
    } 

    \newglossaryentry{unreduced-scalar-module}{
        name={\protect\path{/primitives/src/unreduced_scalar.rs}},
        description={Module for handling unreduced scalars.}
    } 

    \newglossaryentry{monero-primitives-tests}{
        name={\protect\path{/primitives/src/tests.rs}},
        description={Testing module for \gls{monero-primitives (v0.1.0)}}
    } 

    \newglossaryentry{monero-generators-entry-point}{
        name={\protect\path{/generators/src/lib.rs}},
        description={Entry point for \gls{monero-generators (v0.4.0)}}
    } 

    \newglossaryentry{hash-to-point-module}{
        name={\protect\path{/primitives/src/hash_to_point.rs}},
        description={Module for handling hashing data to elliptic curve group elements}
    } 
    
    \newglossaryentry{monero-generators-tests}{
        name={\protect\path{/primitives/src/tests/mod.rs}},
        description={Testing module for \gls{monero-generators (v0.4.0)}}
    } 

    \newglossaryentry{monero-io-entry-point}{
        name={\protect\path{/io/src/lib.rs}},
        description={Entry point for \gls{monero-io (v0.1.0)}}
    } 
    
    \newglossaryentry{monero-address-entry-point}{
        name={\protect\path{/wallet/address/src/lib.rs}},
        description={Entry point for \gls{monero-io (v0.1.0)}}
    } 

    \newglossaryentry{eventuality-module}{
        name={\protect\path{/wallet/src/send/eventuality.rs}},
        description={Module for handling \gls{eventualities}}
    } 
        
    \newglossaryentry{send-multisig-module}{
        name={\protect\path{/wallet/src/send/multisig.rs}},
        description={Module for handling multisig transactions}
    } 
    
    \newglossaryentry{send-tx-module}{
        name={\protect\path{/wallet/src/send/tx.rs}},
        description={Module for handling sending transactions}
    } 
    
    \newglossaryentry{send-tx-keys-module}{
        name={\protect\path{/wallet/src/send/tx_key.rs}},
        description={Module for handling keys in sending transactions}
    } 

    \newglossaryentry{eventualities}{
        name={Eventualities},
        description={A struct for handling the eventual output from \gls{SignableTransaction}s.}
    } 

\newglossaryentry{SignableTransaction}{
    name={SignableTransaction},
    description={A struct representing a Monero transaction prepared for signing, containing necessary inputs, outputs, and metadata but without signatures or key images. Located in \path{/wallet/src/send/}, it handles fee calculation and supports transformation into fully signed transactions through both single-signer and FROST multisig processes}
    }

\title{FROSTLASS: Flexible Ring-Oriented Schnorr-like Thresholdized Linkably Anonymous Signature Scheme}
\subtitle{Scheme Formalization \& Review of Rust Implementation}
\author{
    Joshua Babb\thanks{Cypher Stack} \\ % First line for the first author
    \and
    Brandon Goodell\protect\footnotemark[1] \\ % First line for the second author
    \and
    Luke Parker (kayabaNerve) \\
    \and 
    Rigo Salazar\protect\footnotemark[1] \\ % First line for the third author
    \and
    Freeman Slaughter\protect\footnotemark[1] \\
    \and 
    Luke Szramowski\protect\footnotemark[1]
}

\date{\today}

\begin{document}

\maketitle

\section{Introduction}

% \bran{Brandon}, \free{Freeman}, \luke{Luke}, \rigo{Rigo}, 
% In \cite{bellare2006multi}, Bellare and Neven propose a framework to describe multisignatures and to prove them secure with a generalized forking lemma.

% In \cite{komlo2021frost}, Komlo and Goldberg propose the FROST signature scheme, a thresholdization of Schnorr signatures.

% In \cite{boneh2018threshold}, Boneh, Gennaro, Goldfeder, Jain, Kim, Rasmussen, and Sahai propose a framework to describe universal thresholdizers and prove them secure.

% In \cite{liu2004linkable}, Liu, Wei, and Wong propose linkable spontaneous anonymous group (LSAG) signatures, related closely to ring signatures.

% In \cite{clsag}, Goodell, Noether, and Blue propose concise linkable spontaneous anonymous group (CLSAG) signatures, a scheme whose signatures are asymptotically sublinearly sized, for use in the Monero cryptocurrency protocol, to enable larger ring sizes.

% In \cite{goodell2018thring}, Goodell and Noether present a na\"{i}ve thresholdization of CLSAG signatures called \textit{thring signatures}.

% In \cite{SeraiRepo}, Luke Parker presents an opinionated Rust implementation of every major component of the Monero protocol, including a FROST-inspired thresholdization of the CLSAG signatures used to verify Monero transactions.

% Herein, we formalize Parker's FROST-inspired thresholdization of the CLSAG signature scheme and present security definitions. We also prove Parker's proposed protocol is secure up to the one-more discrete logarithm hardness assumption.


% \subsection{Rigo's attempt at an introduction}

Over the past decades, especially since Shamir's secret sharing and Shoup's threshold signatures, (\cite{shamir1979share}, \cite{shoup2000practical}) research into threshold and multiparty cryptographic schemes of different flavors has become fashionable. In \cite{bellare2006multi}, for example, Bellare and Neven famously proposed a framework to formalize multisignatures and to prove them secure with the generalized forking lemma. 

The general forking lemma, which goes back at least to \cite{li1988analysis}, is useful in proving a wide variety of modern cryptographic schemes secure, including ring signatures preceding \cite{zhang2002id} and the bulletproofs zero-knowledge proving system proposed in \cite{bunz2018bulletproofs}. 

Concise linkable spontaneous anonymous group (CLSAG) signatures, proposed in \cite{clsag} and built from the (LSAG) signatures from \cite{liu2004linkable}, are Schnorr-like ring signatures used in the Monero cryptocurrency protocol.
A na\"{i}ve thresholdization of CLSAG signatures, called \textit{thring signatures}, was proposed in \cite{goodell2018thring}, building off of the linkable spontaneous anonymous group (LSAG) signatures , which are used in the Monero cryptocurrency protocol. The FROST approach to thresholdizing Schnorr signatures, first described in \cite{komlo2021frost}, is sufficiently flexible to work for CLSAG signatures, and are superior to the thring signatures of \cite{goodell2018thring}.

% \bran{Concise linkable spontaneous anonymous group (CLSAG) signatures, proposed in \cite{clsag} and built from the linkable spontaneous anonymous group (LSAG) signatures from \cite{liu2004linkable}, are Schnorr-like ring signatures used in the Monero cryptocurrency protocol to enable larger ring sizes.Goodell and Noether introduced a na\"{i}ve thresholdization of CLSAG signatures, called \textit{thring signatures} in \cite{goodell2018thring}. The FROST approach to thresholdizing Schnorr signatures, first described in \cite{komlo2021frost}, is sufficiently flexible to work for CLSAG signatures, and are superior to the thring signatures of \cite{goodell2018thring}.}

An opinionated Rust implementation of every major component of the Monero protocol at \cite{SeraiRepo}, written by Luke Parker (kayabaNerve), contains an implementation of FROSTLASS. Herein, we formalize FROSTLASS, present a novel definition of linkability, and prove FROSTLASS strongly unforgeable up to the hardness of the $\kappa$-one-more discrete logarithm problem, and statistically linkable.

% \section{General Findings}

% \subsubsection{Recommended Changes}

% \begin{itemize}
% \item \textbf{Include Everything When Hashing.} Ensure that the linking tag is included in computing the aggregated key coefficients $\mu_Y$ and $\mu_k$... and in computing signature challenges $c$...\footnote{explanation}
% \item Replace the sampling of aggregation coefficients $\rho_i$, $\mu_Y$, and $\mu_k$ with XOF invocations; this ought to bring some mild efficiency gains compared to computing independent hashes of each.
% \end{itemize}


\section{Notation and Background Definitions}

\subsection{Notation}

% Notation from \cite{clsag} conflicts with the notation from \cite{frost}, and this scheme was first proposed in \cite{rust} with its own notation. Consequently, we present a ``Rosetta stone'' matching the notation in this paper with these other three sources in Table \cref{table:rosetta_stone}.

% For a set $S$, $\mathcal{P}(S)$ denotes the set of subsets of $S$, and $\mathcal{S}_S$ denotes the symmetric group on $S$. Instead of writing $\mathcal{S}_{[n]}$, we simplify notation by writing $\mathcal{S}_n$. We use $x \sample S$ to indicate $x$ is a single-observation independent sample from the set $S$ under the uniform distribution, and we use $\underline{x} = (x_1, \ldots, x_n) \sample S$ to indicate $n$ independent uniform samples. If $f:S \to T$ is a function, we use $x \mapsto y$ to indicate that $y = f(x)$. For an algorithm $\mathcal{A}$, we use $y \leftarrow \mathcal{A}(x)$ to denote the event that $\mathcal{A}$ inputs $x$ and outputs $y$.

% Let $q$ denote a prime modulus, and let $d \geq 2$.
% Let $\msg \in \bitstrings$ denote a finite-length bitstring message.
% We denote the set $\left\{1,2,\ldots,n\right\} = [n]$. We let $\mathcal{S}_n$ denote the symmetric group on $n$ letters.
% We generally write tuples with an underline, e.g.\ if $n \geq 1$ is an integer and $x_1, \ldots, x_n$ are some objects, we write the tuple $(x_1, \ldots, x_n) = \underline{x}$, with a few exceptions (e.g.\ keys).
% We abuse subset notation for tuples, so that $\underline{x} \subseteq \underline{y}$ denotes the existence of a permutation $\sigma \in \mathcal{S}_{\texttt{len}(\underline{y})}$ such that $y_{\sigma(i)} = x_i$ for each $1 \leq i \leq \texttt{len}(\underline{x})$.
% We let $\rng_\gamma$ denote a random number generator with seed $\gamma\in\bitstrings$.

% Let $\Zq$ the equivalence classes of integers modulo $q$, i.e.\ the field with $q$ elements; we refer to elements of $\Zq$ as \textit{scalars}. We typically use miniscule notation $x, y, z, \ldots$ to denote scalars (but not all miniscule variables are scalars).
% We let $\G$ represent an elliptic curve group with order $q$ written \textit{additively}; we refer to elements of $\G$ as \textit{points}. 
% We typically use majuscule notation $X, Y, Z, \ldots$ to denote points (but not all majuscule variables are points).
% Since $\G$ has order $q$, we may handle $\G$ as a $\Zq$ module, with module multiplication of any $G \in \G$ by any $x \in \Zq$ defined by interpreting $x$ as an unsigned integer on $\left\{0, 1, \ldots, q-1\right\}$ and setting $X = xG = \underbrace{G + G + \cdots + G}_{x\text{ summands}}$. 
% Any fixed $G \in \G$ induces a canonical function $\Zq \to \G$ defined by mapping scalars to points by $x \mapsto xG$. 
% We use the same character in different cases to describe points corresponding to scalars under this canonical function, i.e.\ $X = xG$, $Y = yG$, etc. 

% Now fix $G \in \G$. 
% Denote $\underline{G} = (G, G, \ldots, G) \in \G^d$. 
% We let $-\circ-:\Zq \times \G \to \G$ denote the Hadamard product mapping $(\underline{x}, \underline{Y}) \mapsto (x_iY_i)_i$.
% Let $H_{\Zq}: \bitstrings \to \Zq$, $H_{\G}:\bitstrings \to \G$, $H_\secpar: \bitstrings \to \left\{0,1\right\}^{\secpar}$ be cryptographic hash functions. We denote domain separating tags $\texttt{dst} \in \bitstrings$ and label them with descriptive subscripts.
% % We define the \textit{linking tag function} $\phi:\Zq \to \G$ by mapping $x \mapsto xH_{\G}(X)$, where $X = xG$ as above.


% In the sequel, a Lagrange interpolation of threshold keyshares is a \textit{threshold combination}, and a linear combination of keys with coefficients decided by random oracle is a \textit{key aggregation}.

% Given a specification of an algorithm, we say the algorithm is \textit{honestly computed} if the execution of the algorithm follows the specification exactly. We say the algorithm is \textit{semi-honestly computed} if all the specified steps occur in the specified order, possibly with other additional steps before, between, or after the specified steps.

 


% Similarly, all algorithms in the sequel may output an indexed symbol in the instance of failure, e.g.\ $\bot_{\sign, \texttt{idx}}, \bot_{\verify, \texttt{idx}}$ for indices $\texttt{idx}$, where the indices indicate distinct causes of failure; we omit these symbols for clarity. 

% Similarly, all algorithms in the sequel may input and output some auxiliary data, e.g.\ $\keygen(n, t, \texttt{aux}_{\keygen,in}) \to (\delta, \texttt{aux}_{\keygen,out})$, but we omit these auxiliary data for clarity except when necessary to specify them.

% Given any randomized algorithm $\mathcal{A}$ which outputs distinct failure symbols or some output $\texttt{out}$, we define the \textit{advantage} of $\mathcal{A}$ to be the probability that $\mathcal{A}$ does not output a failure symbol. We denote this probability with $\texttt{Adv}_\mathcal{A}$.

% In the sequel, we let $t \in \mathbb{R}_{> 0}$ denote a positive real number, we let $\epsilon \in [0,1]$ denote a probability, $n \in \bbn$ denote a number of keyholders, and $r \in \bbn$ denote a superthreshold number of keyholders.

Tuples are generally denoted with underlines, i.e.\ $\underline{x} = (x_1, \ldots, x_n)$, and we abuse set notation for these, e.g.\ $x_1 \in \underline{x}$. The set of all finite-length bitstrings is denoted with $\bitstrings$.  For $n \in \bbn$, denote the set $\left\{1, 2, \ldots, n\right\}$ with $\left[n\right]$. For sets $X, Y$ with $X \subseteq Y$, denote the set $\left\{y \in Y \mid y \notin X\right\}$ with $\overline{X}$. 

Denote a prime modulus with $q \in \bbn$, an abelian group of order $q$ with $\G$, and a generator of $\G$ with $G \in \G$. We say the tuple $(q, \G, G)$ are \textit{group parameters}. Given $\underline{x} = (x_1, \ldots, x_n) \in \Zq^n$ and $\underline{G} = (G_1, \ldots, G_n) \in \G^n$, we denote the Shur product $\underline{x} \circ \underline{G} = (x_1G_1, \ldots, x_n G_n)$.


Denote ``big-oh'' notation with $O$ and denote random oracles with $\mathcal{O}$. Denote algorithm run times with $t \geq 0$ and success probabilities with $\epsilon \in [0,1]$. Denote the event that a PPT algorithm $\mathcal{A}$ inputs some $\texttt{in}$ and outputs some $\texttt{out}$ with $\texttt{out} \leftarrow \mathcal{A}(\texttt{in})$. We use the same notation for oracles, but we refer to the inputs as queries, say $\texttt{query}$, and outputs as responses, say $\texttt{resp}$.


\subsection{Definitions}\label{sec:definitions}



\begin{definition}[$\kappa$ Random Oracle Distinguishing]\label{def:distinguisher}
Let $\kappa \geq 0$ be an integer, let $S, T$ be sets, $\mathcal{O}:S \to T$ be a random oracle, and $\phi: S \to T$ a function. Any PPT $(t,\epsilon)$-algorithm $\mathcal{A}$ which plays the following game is an $(\phi,\kappa)$-distinguisher.
\begin{enumerate}
\item The challenger samples $b \sample \left\{0,1\right\}$ and grants $\mathcal{A}$ access to an oracle $\mathcal{O}^\prime_b$, where 
\begin{enumerate}
\item $\mathcal{O}^\prime_0$ is a simple wrapper for $\mathcal{O}$, and
\item $\mathcal{O}^\prime_1$ is a simple wrapper for $\phi$.
\end{enumerate}
\item Eventually, $\mathcal{A}$ outputs a bit $b^\prime$, succeeding if and only if $b^\prime = b$ and $\mathcal{O}^\prime_b$ was queried at most $\kappa$ times.
\end{enumerate}
\end{definition}



\begin{definition}[$\kappa$-OMDL: One-More-Than-$\kappa$ Discrete Logarithms over $G \in \G$]\label{def:omdl}
Let $\kappa \geq 0$ be an integer. Let $ \Phi = \left\{(q_\secpar, \G_\secpar, G_\secpar)\right\}_{\secpar \in \bbn}$ be a parameterized family of group parameters. Let $t \geq 0$ and $\epsilon \in [0, 1]$ be real numbers. We say any PPT algorithm $\mathcal{A}$ that can successfully play the following game in time at most $t$ and with probability at least $\epsilon$ is a $(t,\epsilon)$-player of the one-more-than-$\kappa$ discrete logarithms game over $G_\secpar \in \G_\secpar$.
\begin{enumerate}
\item The challenger grants $\mathcal{A}$ access to a key generation oracle $\keyOracle:\left\{\ast\right\} \to \G_\secpar$ and a corruption oracle $\corruptionOracle: \G_\secpar \to \mathbb{Z}_{q_\secpar}$ which work as follows.
\begin{enumerate}
\item A valid query made to $\keyOracle$ is a simple request for a new key, which we model with a dummy singleton domain $\left\{\ast\right\}$. The response is some point $\texttt{resp}=X \in \G_\secpar$. We say the response is a \textit{challenge key}. Let $\challengeKeySet = \left\{X \in \G_\secpar \mid X \leftarrow \keyOracle\text{ occurred}\right\}$ denote the set of all responses from $\keyOracle$.

\item A valid query made to $\corruptionOracle$ a challenge key, $X \in \challengeKeySet$. The response to a valid query $X$ is a scalar $x \in \mathbb{Z}_{q_\secpar}$ such that $X = xG$, and the response to an invalid query is a distinct failure symbol. Let $\corruptedKeySet \subseteq \challengeKeySet$ be the subset of valid queries made to $\corruptionOracle$ be the \textit{corrupted keys} and let $\overline{\corruptedKeySet}$ be the subset of \textit{uncorrupted challenge keys}.
\end{enumerate}
   \item Eventually, the event $\texttt{out}_\mathcal{A} \leftarrow \mathcal{A}$ occurs. We say $\mathcal{A}$ succeeds at the $\kappa$-OMDL game if and only if all the following hold in this event:
\begin{enumerate}
\item $\left|\challengeKeySet\right| \geq \kappa + 1$,
\item $\left|\corruptedKeySet\right| \leq \kappa$,
\item $\texttt{out}_{\mathcal{A}} \in \mathbb{Z}_{q_\secpar}^{\kappa+1}$, and
\item $\left\{xG \mid x \in \texttt{out}_{\mathcal{A}}\right\} \subseteq \challengeKeySet$
% \item there exists some $\kappa \geq 0$ such that $\texttt{out}_\mathcal{A} \in \Zq^{\kappa+1}$; we denote the $(\kappa+1)$-tuple $\underline{x}=(x_1, \ldots, x_{\kappa}, x_{\kappa+1}) := \texttt{out}_\mathcal{A}$,
% \item for each $i \in [\kappa+1]$, $X_i \leftarrow \keyOracle$ occurred where $X_i = x_i G$, and
% \item there exists an index $i \in [\kappa+1]$ such that $\corruptionOracle$ was not queried with $X_i$. \free{I change this index to j in the game}
\end{enumerate}
\end{enumerate}
Moreover, if $t \in \polysecpar$ implies $\epsilon \in \negl$ for every $(t, \epsilon)$-player, we say the $\kappa$-OMDL game is hard over $\Phi$.
\end{definition}

In the sequel, we leave $\secpar$ implicit, suppressing it in our notation for clarity, except when there is case of confusion.

Note that the $0$-OMDL game is simply the discrete logarithm game. Moreover, an adaptive variation of this game is natural, where  $\kappa$ is determined in each instance of the game by the number of corruption oracle queries made by the adversary.



\begin{oracle}
    \centering
    \begin{suboracle}[ht]{0.2\textwidth}
        \begin{tabular}{|l|}
        \hline
        \multicolumn{1}{|c|}  {\textbf{Oracle} $\keyOracle(\ast)$} \\
        \hline 
        $x \lsamp \Zq$ \\
        $X=xG$ \\
        $\challengeKeySet = \challengeKeySet \cup \left\{X\right\}$ \\
        % , $\underline{X} = (\underline{X} || X)$ \\
        \textbf{return } $X$ \\
        \hline
        \end{tabular}
        % \caption{The challenge oracle}
        % \label{oracle:Chal}
    \end{suboracle}
    \hspace{1em}
    \begin{suboracle}[ht]{0.3\textwidth}
        \begin{tabular}{|l|}
        \hline
        \multicolumn{1}{|c|}  {\textbf{Oracle} $\corruptionOracle(X)$} \\
        \hline 
        \textbf{if} $X \in \challengeKeySet$ \\ \quad \quad
        $x=\log_G{X}$ \\ \quad \quad 
        $\corruptedKeySet = \corruptedKeySet \cup \left\{X\right\}$ \\ \quad \quad
        \textbf{return} $x$ \\
        \textbf{else return} $\perp$ \\
        

        % \textbf{if} $x \notin \underline{x}$, $b=\perp$ \\
        % \textbf{else} $b=x$ \free{needs $\kappa \lar \kappa+1$?} \\
        % \textbf{return} $b$ \\
        \hline
        \end{tabular}
        % \caption{The discrete log oracle}
        % \label{oracle:DLog}
    \end{suboracle}
\caption{The key generation and corruption oracles for the $\kappa$-OMDL game.}
\label{oracle:OMDL}
\end{oracle}

\begin{game}
    \centering
    \begin{tabular}{|l|}
    \hline
    \multicolumn{1}{|c|}  {\textbf{Game} $\kappa\text{-OMDL}(\underline{x})$} \\
    \hline 
    % $\texttt{out}_\mathcal{A}$ \\
    \textbf{if} $\left|\challengeKeySet\right| \geq \kappa+1$  
    \textbf{and }  $\left|\corruptedKeySet\right| \leq \kappa$ \\
    \textbf{and} $\underline{x} \in \Zq^{\kappa+1}$ \textbf{and} $\underline{x} \circ \underline{G} \subseteq \challengeKeySet$\\
    \quad \quad \textbf{return} $1$ \\
    \textbf{else return} $0$ \\
    \hline
    \end{tabular}
    \caption{Success condition for the $\kappa$-OMDL game.}
    \label{game:OMDL}
\end{game}







% The general forking algorithm presented in Lemma \cref{lem:general_forking_lemma} (from \cite{}, and which we use to prove unforgeability) uses the following definition.

\begin{definition}[General Forking Algorithm]\label{def:general_forking_algorithm}
Let $X, Y, H$ be finite sets, $\kappa \geq 1$ an integer parameter, and let $\mathcal{A}$ be a PPT algorithm which uses a random tape $\tau \in \bitstrings$, inputs some $(x, \underline{h}) \in X \times H^\kappa$, and  outputs a pair $(i, y) \in [\kappa] \times Y$ or a distinct failure symbol. Then the algorithm specified below, $\fork_{\mathcal{A}}$, is a PPT algorithm which inputs $x \in X$, outputs $(i, y, y^\prime) \in [\kappa] \times Y^2$ or a distinct failure symbol, and is called the \textit{general forking algorithm} for $\mathcal{A}$.
\begin{enumerate}
\item Sample $\tau \sample \bitstrings$ for $\mathcal{A}$ to use in both executions.
\item Sample $\underline{h}, \underline{h}^\prime \sample H^\kappa$.
\item Compute $\texttt{out} \leftarrow \mathcal{A}(x, \underline{h}; \tau)$.
\item If $\texttt{out}$ is a failure symbol, output a distinct failure symbol and terminate. Otherwise, $\texttt{out}$ is not failure symbol, so parse $(i, y) := \texttt{out}$.
\item Set $\underline{h}^* = (h_1, \ldots, h_{i-1}, h^\prime_i, h^\prime_{i+1}, \ldots, h^\prime_\kappa)$.
\item Compute $\texttt{out}^\prime \leftarrow \mathcal{A}(x, \underline{h}^*; \tau)$.
\item If $\texttt{out}^\prime$ is a failure symbol, output a distinct failure symbol and terminate. Otherwise, $\texttt{out}$ is not a failure symbol, so parse $(i^\prime, y^\prime) := \texttt{out}^\prime$.
\item If $i \neq i^\prime$ or $h_i = h_{i^\prime}^*$, then output a distinct failure symbol and terminate.
\item Otherwise, output $(i, y, y^\prime)$.
\end{enumerate}
\end{definition}


\begin{lemma}[General Forking Lemma]\label{lem:general_forking_lemma}
For any finite sets $X, H$, for any algorithm $\mathcal{A}$ as in  \cref{def:general_forking_algorithm} which runs in time at most $t$ and fails with probability at most $\epsilon$, for any probability mass function $F$ over $X$, the general forking algorithm $\fork_\mathcal{A}$ has advantage satisfying the following
\[\texttt{Adv}_{\fork_\mathcal{A}} \geq \epsilon \left(\frac{\epsilon}{\kappa} - \frac{1}{\left|H\right|}\right)\] where this probability is measured over $F$ and all randomness used in sampling.
\end{lemma}


\begin{definition}[LTM: Linkable Thring Multisignatures]\label{def:ltm}
A tuple of algorithms $(\setup, \keygen, \sign, \\ \combine, \verify, \link)$ as follows.
\begin{enumerate}
\item $\setupIO$. Input a security parameter $\secpar \in \bbn$, and output some public parameters $\params_\secpar$, which includes the description of secret signing key shares $\mathcal{SK}$, public verification key shares $\mathcal{VK}$, total verification keys $\mathcal{TVK}$, messages $\mathcal{MSG}$, signatures challenges $\mathcal{CH}$, partial signature shares $\mathcal{PSIG}$, and signatures $\mathcal{SIG}$.

\item $\keygenIO$. An interactive probabilistic algorithm executed by some \textit{capacity} of $n \geq 1$ participants called \textit{threshold keyholders}. Users share as common input the capacity $n$ and threshold $r \in n$. Output  \textit{total verification key} $\tvk \in \mathcal{TVK}$, \textit{public verification key shares} $\VK=(\vk_i)_{i=1}^{n} \in \mathcal{VK}^n$, and \textit{secret signing key shares} $\SK = (\sk_i)_{i=1}^{n} \in \mathcal{SK}^n$.



\item $\signIObase$. Non-interactive probabilistic algorithm executed by a threshold keyholder. Input a message $\msg \in \mathcal{MSG}$, a tuple of $m \geq 1$ total verification keys $\ring = (\tvk_j)_{j=1}^{m} \in \mathcal{TVK}^m$ called a \textit{ring}\footnote{A better term would be \textit{anonymity tuple}, but we keep with tradition.}, some $r \geq 1$ public verification key shares $\VK = (\vk_i)_{i=1}^{r} \in \mathcal{VK}^r$ called \textit{signers' coalition key shares},  and a secret key share $\sk \in \mathcal{SK}$. Output a ring signature share $\psig \in \mathcal{PSIG}$.

% \item $\verifyshare(\params_\secpar, \msg, \ring, \VK, \psig) \to b \in \left\{0,1\right\}$. Non-interactive deterministic executed by a user called a \textit{share verifier}. Input message $\msg \in \mathcal{MSG}$, a ring $\ring = (\tvk_j)_{j=1}^{m} \in \mathcal{TVK}^m$, a signers' coalition of key shares $\VK=(\vk_i)_{i=1}^{r}$, and a ring signature share $\psig \in \mathcal{PSIG}$. Outputs a bit.

\item $\combineIObase$. Non-interactive deterministic algorithm executed by a user called the \textit{combiner}. Input a message $\msg \in \mathcal{MSG}$, a ring $\ring = (\tvk_j)_{j=1}^{m} \in \mathcal{TVK}^m$,  a signers' coalition of key shares $\VK = (\vk_i)_{i=1}^{r} \in \mathcal{VK}^r$,  and ring signature shares $\psigs=(\psig_i)_{i=1}^{r} \in \mathcal{PSIG}^r$. Output a ring signature $\sig \in \mathcal{SIG}$.

\item $\verifyIObase$. Non-interactive deterministic algorithm executed by a user called the \textit{verifier}. Input message $\msg \in \mathcal{MSG}$, a ring  $\ring = (\tvk_j)_{j=1}^{m} \in \mathcal{TVK}^m$, and a ring signature $\sig \in \mathcal{SIG}$. Output a bit.

\item $\linkIObase$. A non-interactive deterministic algorithm executed by a user called the \textit{linker}.  Input ring signatures $\sig, \sig^\prime \in \mathcal{SIG}$, and output a bit.
\end{enumerate}

\end{definition}

\cref{def:ltm} extends naturally to a \textit{verifiable} scheme by allowing the verification of signature shares with the following additional algorithm.
Adding this additional level of verifiability requires modifying \cref{def:correctness} below in the natural way.


\begin{itemize}
\item $\verifyshareIObase$. Non-interactive deterministic executed by a user called a \textit{share verifier}. Input message $\msg \in \mathcal{MSG}$, a ring $\ring = (\tvk_j)_{j=1}^{m} \in \mathcal{TVK}^m$, a signers' coalition of key shares $\VK=(\vk_i)_{i=1}^{r}$, and a ring signature share $\psig \in \mathcal{PSIG}$. Outputs a bit.
\end{itemize}

Any of the algorithms in \cref{def:ltm} may input or output auxiliary data $\texttt{aux}$, which we only include in notation when relevant. Following our convention for group parameter notation, we leave $\params_\secpar$ implicit in our notation, as all algorithms require it.



\begin{definition}\label{def:correctness}
Let $\Pi$ be an LTM scheme. We define correctness using the following events.
\begin{enumerate}
\item Let $E_1$ be the event in which some signers' coalitions of key shares $\VK^\prime, \VK^{\prime \prime} \subseteq \VK$ is used to compute ring signature shares $\psig_i^\prime$ and $\psig_i^{\prime \prime}$ semi-honestly. That is to say, the following holds.
\begin{enumerate}
\item For some $\msg^\prime, \msg^{\prime \prime} \in \mathcal{MSG}$,
\item for some $n, r \in \bbn$ such that  $r \in [n]$ and some event $(\tvk, \VK,  \SK) \leftarrow \keygen\keygenIshort$ occurs, 
\item for some $\VK^\prime, \VK^{\prime \prime}$
such that $\VK^\prime, \VK^{\prime \prime} \subseteq \VK$, $r^\prime = \left|\VK^\prime\right|$, $r^{\prime \prime} = \left|\VK^{\prime \prime}\right|$, and  $r \leq \min\left\{r^\prime, r^{\prime \prime}\right\}$,
\item for some $\sigma^\prime \in \mathcal{S}_n$ such that, for every $i \in [r^\prime]$, $\vk^{\prime}_i = \vk_{\sigma^\prime(i)}$ and  $\sk^\prime_i = \sk_{\sigma^\prime(i)}$,
\item for some $\sigma^{\prime\prime} \in \mathcal{S}_n$ such that, for every $i \in [r^{\prime \prime}]$, $\vk^{\prime\prime}_i = \vk_{\sigma^{\prime\prime}(i)}$ and  $\sk^{\prime\prime}_i = \sk_{\sigma^{\prime\prime}(i)}$,
\item for some $\ring^\prime, \ring^{\prime \prime} \subseteq \mathcal{TVK}$ such that $\tvk \in \ring^{\prime} \cap \ring^{\prime \prime}$,

\item for each $i \in [r^\prime]$, $\psig_i^\prime \leftarrow \sign(\msg^\prime, \ring^\prime, \VK^\prime,  \sk^\prime_i)$, and
\item for each $i \in [r^{\prime \prime}]$, $\psig_i^{\prime\prime} \leftarrow \sign(\msg^{\prime\prime}, \ring^{\prime \prime}, \VK^{\prime\prime},  \sk^{\prime \prime}_i)$.
\end{enumerate}

\item Let $E_1^*$ be a similar event in which some $\VK^{**} \subseteq \VK^*$ compute ring signature shares $\psig^*_i$ semi-honestly from a different $\tvk^*$, i.e.\ all the following hold.
\begin{enumerate}
\item For some $\msg^* \in \mathcal{MSG}$,
\item for some $n^*, r^* \in \bbn$ such that  $r^* \in [n^*]$ and some event $(\tvk^*, \VK^*,  \SK^*) \leftarrow \keygen\keygenIshort$ occurs such that $(\tvk^*, \VK^*,  \SK^*) \neq (\tvk^\prime, \VK^\prime,  \SK^\prime)$ and $(\tvk^*, \VK^*,  \SK^*) \neq (\tvk^{\prime\prime}, \VK^{\prime\prime},  \SK^{\prime\prime})$,
\item for some $\VK^{**}$
such that $\VK^{**} \subseteq \VK^*$, $r^{**} = \left|\VK^*\right|$, and  $r^* \leq r^{**}$,
\item for some $\sigma^* \in \mathcal{S}_n$ such that, for every $i \in [r^{**}]$, $\vk^*_i = \vk_{\sigma^*(i)}$ and  $\sk^*_i = \sk_{\sigma^*(i)}$,
\item for some $\ring^* \subseteq \mathcal{TVK}$ such that $\tvk^* \in \ring^*$,

\item for each $i \in [r^{**}]$, $\psig_i^{**} \leftarrow \sign(\msg^*, \ring^*, \VK^*,  \sk^*_i)$.
\end{enumerate}



% \item Let $E_2 \subseteq E_1^* \cap E_1$ that the signature shares are valid, i.e.\ all the following hold.
% \begin{enumerate}
% \item for each $1 \leq i \leq r^\prime$, $\verifyshare(\msg^\prime, \ring^\prime, \VK^\prime, \psig^\prime_i) = 1$,
% \item for each $1 \leq i \leq r^{\prime\prime}$, $\verifyshare(\msg^{\prime \prime}, \ring^{\prime \prime}, \VK^{\prime\prime}, \psig^{\prime \prime}_i) = 1$, and
% \item for each $1 \leq i \leq r^{**}$, $\verifyshare(\msg^*, \ring^*, \VK^*, \psig^*_i) = 1$.
% \end{enumerate}


\item Let $E_2 \subseteq E_1^* \cap E_1$ be the event that the ring signature shares $\psig_i^\prime$, $\psig_i^{\prime \prime}$, and $\psig_i^{*}$ are combined semi-honestly, i.e.\ all of the following hold.
\begin{enumerate}
\item $\sig^\prime \leftarrow \combine(\msg^\prime, \ring^\prime, \psigs^\prime)$,
\item $\sig^{\prime\prime} \leftarrow \combine(\msg^{\prime\prime}, \ring^{\prime\prime},  \psigs^{\prime\prime})$,
\item $\sig^* \leftarrow \combine(\msg^*, \ring^*, \psigs^*)$.
\end{enumerate}

\item  Let $E_3 \subseteq E_2$ be the event that the combined signatures are valid, i.e.\ all the following hold.
\begin{enumerate}
\item $\verify(\msg^{\prime}, \ring^{\prime}, \sig^\prime) = 1$,
\item $\verify(\msg^{\prime\prime}, \ring^{\prime\prime}, \sig^{\prime\prime}) = 1$, and
\item $\verify(\msg^{\prime\prime}, \ring^{\prime\prime}, \sig^{\prime\prime}) = 1$.
\end{enumerate}

\item Let $E_4 \subseteq E_3$ be the event that $\link$ is commutative, i.e.\ all the following hold.
\begin{enumerate}
\item $\link(\sig^\prime, \sig^{\prime \prime}) = \link(\sig^{\prime \prime}, \sig^\prime)$,
\item $\link(\sig^\prime, \sig^*) = \link(\sig^*, \sig^\prime)$, and
\item $\link(\sig^*, \sig^{\prime \prime}) = \link(\sig^{\prime \prime}, \sig^*)$
\end{enumerate}

\item Let $E_5 \subseteq E_2$ that $\link(\sig^\prime, \sig^{\prime \prime}) = 1$.

\item Let $E_6 \subseteq E_3$ that $\link(\sig^\prime, \sig^*) =  \link(\sig^*, \sig^{\prime \prime})=0$.

\end{enumerate}
We say $\Pi$ has \textit{correct ring signature share verification} if $\mathbb{P}[E_2] = 1$, has \textit{correct ring signature verification} if $\mathbb{P}[E_4] = 1$, has \textit{commutative linking} if $\mathbb{P}[E_5]=1$, has \textit{correct positive linkability} if $\mathbb{P}\left[E_6\right]=1$, and has \textit{correct negative linkability} if $\mathbb{P}\left[E_7\right]=1$, where these probabilities are computed over all choices of $n, r, n^*, r^*, \msg^\prime, \msg^{\prime \prime}, \msg^*$, all executions of $\keygen$, all choices of $\VK^\prime, \VK^{\prime \prime}, \VK^*$, and all randomness used by all algorithms. If $\Pi$ satisfies all four notions of correctness, we simply say $\Pi$ is a \textit{correct} LTM. 
\end{definition}


For convenience, we present the following common setup useful for linkability and unforgebality.

\begin{definition}[Common Setup with Key Generation, Corruption, and Signing Oracles]\label{def:common_setup}
Let $\Pi$ be an LTM scheme. Let $\mathcal{A}$ be any PPT algorithm which runs in time at most $t > 0$, and successfully plays the following game with probability at least $\epsilon \in [0,1]$.
\begin{enumerate}
\item $\mathcal{A}$ is granted to oracles $\keyOracle$, $\corruptionOracle$, and $\signingOracle$ as follows.
\begin{enumerate}

\item $(\tvk, \VK) \leftarrow \keyOracle(n, r).$ A valid query made to $\mathcal{O}_{\texttt{key}}$ is a simple request for $r$-of-$n$ keys, which we model with the pair $(n, r)$ such that $r \in [n]$. The response to a valid query is some $\texttt{resp} = (\tvk, \VK) \in \mathcal{TVK} \times \mathcal{VK}^n$, and the response to an invalid query is a distinct failure symbol. Let $\challengeKeySet = \left\{(\tvk, \VK) \mid \exists (n, r), (\tvk, \VK) \leftarrow \keyOracle(n, r)\right\}$ denote the set of all responses from $\keyOracle$.

\begin{oracle}
    \centering
        \begin{tabular}{|l|}
        \hline
        \multicolumn{1}{|c|}  {\textbf{Oracle} $\keyOracle(n,r)$} \\
        \hline 
        \textbf{if} $r \in [n]$ \\  
        \quad \quad \textbf{sample} $(\tvk, \VK)$ \\
        \quad \quad $\challengeKeySet = \challengeKeySet \cup \left\{(\tvk, \VK)\right\}$ \\
        \quad \quad \textbf{return} $(\tvk, \VK)$ \\
        \textbf{else return} $\perp$ \\
        \hline
        \end{tabular}
\caption{The key generation oracle in the game of common setup.}
\label{oracle:key}
\end{oracle}

\item $\sk \leftarrow \corruptionOracle(i, \tvk, \VK).$ A valid query made to $\corruptionOracle$ is some $\texttt{query}=(i, \tvk, \VK)$ where $(\tvk, \VK) \in \challengeKeySet$ is associated with some $(n, r) \in \bbn^2$ such that $r \in [n]$ and $(\tvk, \VK) \leftarrow \keyOracle(n,r)$ occurred, and $i$ is an index in $[n]$. The response to a valid query is a secret signing key $\sk_i$ corresponding to $\vk_i \in \VK$, and the response to an invalid query is a distinct failure symbol. 

Upon success, we say the verification key share $\vk_i$ has been corrupted, and if $r$ or more key shares have been corrupted associated with $\tvk$, then we say $\tvk$ has been totally corrupted. Let $\corruptedKeyShareSet = \left\{\vk_i \mid \sk_i \leftarrow \corruptionOracle(i, \tvk, \VK)\text{ occurred}\right\}$ be the set of corrupted key shares and let $\corruptedTotalKeys$ be the set of totally corrupted keys.

% inputs  $(i, (\tvk, \VK))$, outputs a secret signing key share $\sk_i$ corresponding to public verification key share and linking tag basepoint share $(\vk_i, \lt_i) \in \VK \times \LT$ in the event that $(\tvk, \VK, \LT) \leftarrow \keyOracle(n, r)$ for some $n, r$ and $i \in [n]$, or a distinct failure symbol; we say $(\vk_i, \lt_i)$ are corrupted if $\corruptionOracle$ does not output a failure symbol. Keep the list of corrupted verification key shares in $\corruptedKeyShareSet$. If $(\tvk, \VK, \LT) \leftarrow \keyOracle(n, r)$ and at least $r$ elements $\vk_i \in \VK \cap \corruptedKeyShareSet$, we say $\tvk$ is totally corrupted. Keep the list of totally corrupted verification key shares $\corruptedTotalKeys$. Denote the list of total verification challenge keys which have not yet been corrupted with $\overline{\corruptedTotalKeys}$. %Denote the list of queries made to $\corruptionOracle$ with $\mathcal{L}_{\texttt{corrupt}}$. If at least $r$ key shares corresponding to $\tvk$ have been corrupted, we say $\tvk$ has been \textit{totally corrupted}, and we keep a list of these $\corruptedTotalKeys$.



\begin{oracle}
    \centering
        \begin{tabular}{|l|}
        \hline
        \multicolumn{1}{|c|}  {\textbf{Oracle} $\corruptionOracle(i, \tvk, \VK)$} \\
        \hline 
        \textbf{if} $(\tvk, \VK) \lar \keyOracle(n,r)$\textbf{ and }$i \in [n]$\textbf{ then } \\ 
        \quad \quad $\corruptedKeyShareSet = \corruptedKeyShareSet \cup \left\{\vk_i\right\}$ \\ 
        \quad \quad \textbf{if} $|\VK \cap \corruptedKeyShareSet| \geq r$  
        \textbf{then}\\
        \quad \quad \quad \quad $\corruptedTotalKeys \lar \corruptedTotalKeys \cup \left\{\tvk\right\}$ \\
        \quad \quad \textbf{return} $\sk_i$ \\
        \textbf{else return} $\perp$ \\
        
        \hline
        \end{tabular}
\caption{The corruption oracle in the game of common setup.}
\label{oracle:corrupt}
\end{oracle}




\item $\sig \leftarrow \signingOracle(\msg, \ring, \tvk, \VK, i).$ A valid query to $\signingOracle$ is a tuple $(\msg, \ring, \tvk, \VK, i)$, where $\msg \in \bitstrings$, $\ring \in \mathcal{TVK}^m$ for some $m \in \bbn$, $\tvk \in \mathcal{TVK}$, $\VK \in \VK^r$ for some $r \in \bbn$, and $i \in \bbn$, such that all the following hold.
\begin{enumerate}
\item $\tvk \in \ring$,
\item there exists some $n$, $\VK^\prime$ such that $r \in [n]$ and $(\tvk, \VK^\prime) \leftarrow \keyOracle(n,r)$ occurred,
\item the query $\VK$ is a subset $\VK \subseteq \VK^\prime$ such that $\left|\VK\right| \geq r$, and
\item $i \in [r]$.
\end{enumerate}

\begin{oracle}
    \centering
        \begin{tabular}{|l|}
        \hline
        \multicolumn{1}{|c|}  {\textbf{Oracle} $\signingOracle(\msg, \ring, \tvk, \VK, i)$} \\
        \hline 
        \textbf{parse} $r:=\texttt{len}(\VK)$ \\
        \textbf{if} $\tvk \in \ring$ \\ 
        \textbf{and} $\exists n,r^\prime,\VK^\prime$ s.t. $(\tvk, \VK^\prime) \leftarrow \keyOracle(n,r^\prime)$ \\ 
        \textbf{and} $\VK \subseteq \VK^\prime$ \textbf{and} $r \geq r^\prime$ \textbf{ and } $i \in [r]$ \\
        \quad \quad \textbf{return} $\psig$ \\
        \textbf{else} $\perp$ \\
        \hline
        \end{tabular}
\caption{The signing oracle}
\label{oracle:sign}
\end{oracle}

The response to an invalid query is a distinct failure symbol, and the response to a valid query is a valid partial signature $\psig$ which is combinable with other valid signatures, and links to a challenge key, as follows.
\begin{itemize}
\item \textbf{Well-Formed Queries Combine to Valid Signatures.} 
If oracle response events
$\psig_{\vk^*} \leftarrow \signingOracle(\msg, \ring, \tvk, \VK, \vk^*)$ occur for each  $\vk^* \in \VK$, the combination of these responses
$\sig = \combine(\msg, \ring, \VK, \PSIG)$ is valid, $\verify(\msg, \ring, \sig)=1$. % Since each of these signing oracle queries share $\msg, \ring, \tvk, \VK^\prime$ as common input data and each $\vk^\prime \in \VK^\prime$, we say these queries are \textit{well-formed}.

\item \textbf{Links to Challenge Key.}
If $\msg^\prime$ is a message, $\VK^{\prime \prime} \subseteq \VK^\prime$ such that $\left|\VK^{\prime \prime}\right| \geq r$, $\ring^\prime$ is a ring with $\tvk \in \ring \cap \ring^\prime$, and there exist signing oracle responses for each $\vk^{**} \in \VK^{\prime \prime}$, $\psig^\prime_{\vk^{**}} \leftarrow \signingOracle(\msg^\prime, \ring^\prime, \tvk, \VK^{\prime \prime}, \vk^{**})$, then $\link(\params_\secpar, \sig, \sig^\prime)=1$ where $\sig^\prime = \combine(\msg^\prime, \ring^\prime, \VK^{\prime \prime}, \PSIG^\prime)$ is the combined signature.
\item \textbf{Verifiability.} If $\Pi$ is verifiable, then $\verifyshare(\msg, \ring, \VK, \psig)=1$
\end{itemize}


% \item To query $\mathcal{O}_{\sign}$, $\mathcal{A}$ sends a message-ring-index-index-coalition tuple $(\msg, \ring, \ell, i, \VK)$. If $1 \leq \ell \leq \kappa$ and $\tvk^{(\ell)} \in \ring$ and $\VK^\prime \subseteq \VK^{(\ell)}$ such that $r_\ell \leq \left|\VK^\prime\right|$ and $1 \leq i \leq \left|\VK^\prime\right|$, then the event $\psig^{(\ell)}_i \leftarrow \mathcal{O}_{\sign}(\msg, \ring, \ell, i, \VK)$ occurs. Moreover, this $\psig^{(\ell)}_i$ is a valid ring signature share, i.e.\ $\verifyshare(\msg, \ring, \VK^\prime, \psig^{(\ell)}_i) = 1$, and $\psig^{(\ell)}_i$ is \textit{combinable} with other oracle-generated signatures in the following sense.
% If there exists some $r_\ell \leq r$, some message $\msg$, some ring $\ring$, some index $\ell \in [\kappa]$, and signers' coalition of key shares $\VK^\prime$ such that the conjunction of the events $\psig_i \leftarrow \mathcal{O}_{\sign}(\msg, \ring, \ell, i, \VK^\prime)$ for every index $i \in [r]$, $\psigs = (\psig_i)_{i=1}^{r}$, and $\sig = \combine(\msg, \ring, \psigs^{(\ell)})$, then $\verify(\msg, \ring, \sig) = 1$. 


% \item Use $\mathcal{L}_c$ to denote the list of query-response pairs in the transcript with $\corruptionOracle$, and similarly $\mathcal{L}_s$ for $\mathcal{O}_{\sign}$.
\end{enumerate}
\item Eventually, $\mathcal{A}$ outputs some $\texttt{out}_\mathcal{A}$ which includes one or more message-ring-signature triples $(\msg, \ring, \sig)$ such that all the signatures are valid (in which case we say $\mathcal{A}$ succeeds) or a distinct failure symbol (in which case we say $\mathcal{A}$ fails).
\end{enumerate}

Further assume that, if $\mathcal{A}$ requires more oracle queries allowed than the following bounds, or if $\mathcal{A}$ is about to make an oracle query which will cause the oracle to fail, then $\mathcal{A}$ outputs a distinct failure symbol and terminates.
\begin{enumerate}
\item There exists $\kappa_{\texttt{key}}$, $\kappa_{\texttt{corrupt}}$, $\kappa_{\sign} \geq 0$ such that, in every successful transcript, $\mathcal{A}$ makes at most $\kappa_{\texttt{key}}$ respective queries to $\keyOracle$, at most $\kappa_{\texttt{corrupt}}$ queries to $\corruptionOracle$, and at most $\kappa_{\sign}$ queries to $\signingOracle$. 

\item For each random oracle, say $H$, to which $\mathcal{A}$ has access, there exists as similar integer $\kappa_{H}$ such that $\mathcal{A}$ makes at most $\kappa_{H}$ queries to the corresponding random oracle $H$ in every successful transcript. 

\item There exists an integer $n_{\texttt{key}}$ such that all queries $\texttt{query} = (n, r)$ made to $\keyOracle$ satisfies $n \leq n_{\texttt{key}}$ in every successful transcript.

% \item $\mathcal{A}$ prefers to output a distinct failure symbol than to output an unsuccessful forgery, or to make a failed oracle query.
\end{enumerate}

If $\mathcal{A}$ succeeds with probability at least $\epsilon \in [0,1]$ is $(t,\epsilon)$-\textit{player of the game with common setup} with key generation oracle access, corruption oracle access, and signing oracle access. 
% Moreover, if  $t \in \polysecpar$ implies $\epsilon \in \negl$ for every $(t, \epsilon)$-player, then we say $\Pi$ is \textit{secure} with respect to the game of common setup.
\end{definition}

Winning this game is trivial, so the notion of security against players of this game is vacuous. However, players of our unforgeability and linkability games in \cref{def:suf} and \cref{def:linkable} are also players of the game of common setup, just with nontrivial success conditions.

% Evident additional conditions for \cref{def:common_setup} can account for $\verifyshare$ in the case that $\Pi$ is verifiable. Similar definitions with/without oracle access are also evident. %$\kappa$-common setup, $\kappa$-common setup with corruption, and $\kappa$-common setup with signing.


% Inspired by the TS-SUF-3 definition of threshold signature strong unforgeability in \cite{bellare2022better}, we present the following game.

\begin{definition}[LTM-SUF-1: Strong Unforgeability]\label{def:suf}
Let $\Pi$ be an LTM scheme. Let $\mathcal{A}$ be any $(t,\epsilon)$-player of the game with common setup with key generation, corruption, and signing oracle access such that every successful output of $\mathcal{A}$ has some $(\msg, \ring, \sig) \in \texttt{out}_{\mathcal{A}}$ satisfying all the following.
\begin{enumerate}
\item The signature is valid, $\verify(\msg, \ring, \sig)=1$.

\item All ring members are challenge keys, $\ring \subseteq \challengeKeySet$.

\item If all the following hold, then $\link(\sig, \sig^\prime) = 1$:
\begin{enumerate}
\item there exists some
$(j, \tvk_j) \in [m] \times \ring$, 
capacity  $n \in \bbn$, and a 
threshold $r \in [n]$ such that
a key generation oracle query event $(\tvk_j, \VK) \leftarrow \keyOracle(n,r)$ occurs,
\item there exists some $\msg^\prime \in \bitstrings$, some $\ring^\prime \in \mathcal{P}(\mathcal{TVK})$ such that $\tvk_j \in \ring \cap \ring^\prime \setminus \corruptedTotalKeys$,  some signers' coalition of public verification keys $\VK^\prime \subseteq \VK$ such that $\left|\VK^\prime\right| \geq r$,  and events $\psig_{\vk^\prime} \leftarrow \signingOracle(\msg^\prime, \ring^\prime, \tvk_j, \VK^\prime, \vk^\prime)$ for each $\vk^\prime \in \VK^\prime$ which occurred, and
\item $\sig^\prime = \combine(\msg^\prime,\ring^\prime,\VK^\prime,\PSIG)$.
\end{enumerate}

% There exists a challenge key index $\ell \in [\kappa]$ and a superthreshold coalition size $r_\ell \leq r \leq n_\ell$ such that $\tvk^{(\ell)} \notin \corruptedTotalKeys$ and, for each $i \in [r]$, there exists a query $\psig_i \leftarrow \mathcal{O}_{\sign}(\msg^\prime_i, \ring^\prime_i, \ell, i, \VK^\prime_i)$, such that  \[\link(\sig, \combine(\msg^\prime, \ring^\prime, \underline{\psig}))=1.\]

\item For any $\msg^\prime$, for any $\ring^\prime$, for any $\VK^\prime$, if every event $(\psig_i \leftarrow \signingOracle(\msg^\prime, \ring^\prime, \VK^\prime, \vk_i)$ occurs for each  $\vk_i \in \VK$, then $\sig \neq \combine(\msg^\prime, \ring^\prime, \VK^\prime, \PSIG^\prime)$.
\end{enumerate}
Then we say $\mathcal{A}$ is an \textit{LTM strong forger} for $\Pi$. Moreover, if  $t \in \polysecpar$ implies $\epsilon \in \negl$ for every LTM strong forger, then we say $\Pi$ is \textit{strongly unforgeable}.
\end{definition}

We connect this definition to the definition of strong unforgeability for threshold digital signature schemes, TS-SUF-4 from \cite{bellare2022better}. Indeed, we include as ``trivial'' all ring signatures which are a superthreshold combinations of oracle-generated signature shares which all use a common query. This way, if an attacker can combine seemingly unrelated ring signature shares to obtain a valid signature, we count this as a forgery. 

However, ring signatures have their own hierarchy of security definitions, and some of these depend on how many adversarially-selected ring members are allowable in a forgery. We call the previous definition LTM-SUF-1, because an unforgeable scheme under this definition stops forgers from generating ostensibly valid signatures, but only when all ring members are challenge keys. Natural extensions may be a fruitful area of further research.


The following is, as far as the authors are aware, a novel definition of linkability for ring signatures.

\begin{definition}[$\kappa$-Linkability]\label{def:linkable}
Let $\Pi$ be a LTM signature scheme and $\mathcal{A}$ be a $(t,\epsilon)$-player of the game of common setup with key generation, corruption, and signing oracle access such that every successful output of $\mathcal{A}$ has some $\left\{(\msg_u, \ring_u, \sig_u)\right\}_{u\in[\kappa+1]}$ satisfying all the following properties.
\begin{enumerate}
\item For each $u \in [\kappa+1]$, $\verify(\msg_u, \ring_u, \sig_u) = 1$.
% \item For each $u \in [\kappa+1]$, $\link(\sig_u, \sig_u) = 1$.
\item For each $u, v \in [\kappa+1]$, $\link(\sig_u, \sig_v) = \delta_{u,v}$, the Kronecker delta function.
\item At most $\kappa$ keys can be under adversarial control, $\left|\cup_u \ring_u \setminus \overline{\corruptedTotalKeys}\right| \leq \kappa$
\end{enumerate}
Then we say $\mathcal{A}$ is a \textit{$\kappa$-linkability breaker} for $\Pi$. Moreover, $t \in \polysecpar$ implies $\epsilon \in \negl$ for every $\kappa$-linkability breaker, we say $\Pi$ is $\kappa$-\textit{linkable.}
\end{definition}

Note that if we remove oracle access from \cref{def:linkable}, we recover the notion of pigeonhole linkability. On the other hand, if we retain oracle access but set $\kappa = 1$, we recover the notion of ACST linkability.


\section{FROSTLASS Construction}

We now provide a formal description of FROSTLASS. We note that the definition provided here varies from the Rust implementation at \cite{SeraiRepo} in several ways to improve readability, in a few security-neutral ways. We discuss these in \cref{sec:variations_from_implem}.

\begin{definition} \label{def:frostlass}
Let $F_{\texttt{PRNG}}$ be a seedable pseudorandom number generator. FROSTLASS consists of the following algorithms. 
\begin{enumerate}
\item $\setup(\secpar) \to (q,\G, G, d, \underline{H})$ where $q \geq 1$ is a prime modulus, $\G$ is an abelian group of order $q$, $G \in \G$ is a generator, $d \in \bbn$ is a key dimension, and $\underline{H}$ are the following random oracles.
\begin{enumerate}
\item $H_{\texttt{base}}:\bitstrings \to \G$,
\item $H_{\seed}:\bitstrings \to \left\{0,1\right\}^\secpar$,
\item $H_{\FROST,i}:\bitstrings \to \Zq$ for each $i \in \mathbb{N}$,
\item $H_{\lt}^*:\bitstrings \to \Zq$.
\item $H_{\lt,k}:\bitstrings \to \Zq$ for each $1 \leq k \leq d-1$,
\item $H_{\texttt{ch}}:\bitstrings \to \Zq$.
\end{enumerate}
\item $\keygen(n, r, \underline{z}) \to (\tvk, \VK, \LT, \SK)$. 
An interactive PPT algorithm which requires $n \geq 2$ participants.  Participants decide upon  a threshold $1 \leq r \leq n$, and scalars $\underline{z} = z_1, \ldots, z_{d-1} \in \Zq$ via secure side channel; they share these data as common input.  Participants do the following. 
\begin{enumerate}
\item Participants use FROST key generation such that, for each $1 \leq i \leq n$, the $i^{th}$ participant obtains the total verification FROST key $Y$, secret signing key share\footnote{The secret share $y_i$ is denoted $s_i$ in the original FROST paper; however, we use $s_i$ for signature data to maintain consistency with previous ring signature publications.}  $y_i$, and public verification key share $Y_i$.
\item For each $i \in [n]$, the $i^{th}$ participant computes $Z_k = z_k G$ for each $k \in [d-1]$. These are called the \textit{auxilliary keys}.
\item Compute the \textit{main linking tag share} $\mathfrak{T}_i = y_i H_{\texttt{base}}(Y)$ and the \textit{auxilliary linking tags} $\mathfrak{D}_k = z_k \cdot H_{\texttt{base}}(Y)$ for $k \in [d-1]$.
\item Set the following.
\begin{align*}
\sk_i &= (y_i, z_1, \ldots, z_{d-1}), \\
\vk_i &= (Y_i, Z_1, \ldots, Z_{d-1}), \\
\tvk &= (Y, Z_1, \ldots, Z_{d-1}),\text{ and}\\
\lt_i &= (\mathfrak{T}_i, \mathfrak{D}_1, \ldots, \mathfrak{D}_{d-1})
\end{align*}
The key $Y \in \tvk$ is called the \textit{linking key}. 
\end{enumerate}
At the end of this process, each participant has learned their total verification key $\tvk$ for the group, secret key shares $\sk_i$, public key shares $\vk_i$, and linking tag share $(\tvk, \vk_i, \lt_i, \sk_i)$. We call these participants \textit{signers}.


\item $\sign(\msg, \ring, \VK,  \sk, (\tlt, \texttt{com}, d, e)) \to \psig$. 
A non-interactive PPT algorithm individually carried out by signers. Signers are expected to interactively decide upon a message $\msg$, a ring $\ring$, a signers' coalition of public verification key shares $\VK^\prime$, a total linking tag $\tlt$, and a hash table $\texttt{com}$ by secure side channel with authentication in a pre-processing step before executing $\sign$; see $\preproc$ below in \cref{sec:extensions}. Input a message $\msg$, a ring $\ring = (\tvk_1, \ldots, \tvk_m)$, a signers' coalition of public verification key shares $\VK^\prime = (\vk_i^\prime)_{i=1}^{r}$, a secret key $\sk$, and auxiliary data $(\tlt, \texttt{com}, d, e)$, where $\tlt$ is a total linking tag, $\com = \left\{(\vk_i, (D_i, E_i, D_i^\prime, E_i^\prime))\right\}_{i=1}^{r}$ is a hash table with keys $\vk_i \in \VK$ and values which are quadruples $(D_i, E_i, D^\prime_i, E^\prime_i) \in \G^4$, and $d, e \in \Zq$ are secret scalars.
% Output a ring signature share $\psig$ or a distinct failure symbol. 
The signer does the following.
\begin{enumerate}
\item  Find the index $j^* \in [m]$ such that the ring member $\tvk_{j^*} \in \ring$ is the total verification key. If no such index exists, output a distinct failure symbol and terminate.
\item Find the index $i^* \in [r]$ such that $\sk$ corresponds to $\vk_{i^*} \in \VK$. If no such index exists, output a distinct failure symbol and terminate.
\item Parse:
\begin{enumerate}
\item $(y_{i^*}, z_1, \ldots, z_{d-1}) := \sk$, 
\item $(Y_i, Z_{i,1}, \ldots, Z_{i,d-1}) := \vk_i$ for $i \in [r]$,
% \item $(\mathfrak{T}_i, \mathfrak{D}_{i,1}, \ldots, \mathfrak{D}_{i,d-1}) := \lt_i$ for $i \in [r]$,
\item $\left\{(\vk_i, (D_i, E_i, D_i^\prime, E_i^\prime))\right\}_{i=1}^{r} := \texttt{com}$, 
\item $(Y^\prime_j, Z_{j,1}^\prime, \ldots, Z_{j,d-1}^\prime) := \tvk_j$ for $j \in [m]$, and
\item $(\mathfrak{T}, \mathfrak{D}_1, \ldots, \mathfrak{D}_{d-1}) := \tlt$.
\end{enumerate}

\item If there exists any $j \in [m]$, $k \in [d-1]$ such that $Z_{j,k} \neq Z^\prime_{j,k}$, output $\bot$ and terminate.


\item Compute:
\begin{enumerate}
\item the point hash $\widehat{Y}_{j^*} = H_{\texttt{base}}(Y_{j^*})$.
\item a seed\footnote{A functionally equivalent implementation uses an extendable output function $\texttt{xof}$ to extract $\underline{s}$ directly, bringing efficiency gains and reducing the risk of implementation errors.} $\gamma \leftarrow H_{\seed}(\VK \mid \mid \widehat{Y}_{j^*} \mid \mid \ring \mid \mid \tlt \mid \mid \msg \mid \mid \texttt{com})$, 
\item the Lagrange coefficients $\lambda_i = \prod_{i^\prime \neq i} \frac{i}{i^\prime-i} (\text{mod }q)$ for each $i \in [r] = \left\{1,2,\ldots,r\right\}$.

% $\lambda_j, \ldots, \lambda_r$ for interpolating these $\vk_1, \ldots, \vk_r$ into $\tvk_{j^*}$,
% \item the linking tag $\mathfrak{T} = \sum_i \lambda_i \mathfrak{T}_i$,
% \item the total linking tag $\tlt = (\mathfrak{T}, \mathfrak{D}_1, \ldots, \mathfrak{D}_{d-1})$,
\item FROST coefficients for $i \in [r]$, $\rho_i = H_{\FROST,i}(\msg \mid \mid \widehat{Y}_{j^*} \mid \mid \ring \mid \mid \VK \mid \mid \LT \mid \mid \com)$,



\item FROST nonces $F_i = D_i + \rho_i E_i$ for $i \in [r]$,
\item FROST-like nonces $F^\prime_i = D^\prime_i + \rho_i E^\prime_i$ for $i \in [r]$,

\item starting nonces $L_{j^*} = \sum_{i=1}^{r} F_i$ and $R_{j^*} = \sum_{i=1}^{r} F^\prime_i$, and
\item starting signature challenge \[c_{j^*+1} = H_{\texttt{ch}}\signaturequerystar\] where we identify $c_{m+1} \equiv c_1$ in the case that $j^* = m$.


\end{enumerate}

\item Sample $(s_j)_{j \neq j^*} \leftarrow F_{\texttt{PRNG}}(\gamma)$.


\item\label{step:commonstuff} Compute:
\begin{enumerate}

\item the point hashes of ring members' leading keys $\widehat{Y}_j = H_{\texttt{base}}(Y_j^\prime)$.

\item the aggregation coefficients
 % \footnote{FIX: We can avoid domain separating tags and associated implementation errors without losing efficiency by using an extendable output function $\texttt{xof}$ instead of hash functions here, extracting $(\mu_Y, \mu_1, \ldots, \mu_{d-1}) \leftarrow \texttt{xof}(\vk_{j^*})$ directly with one function call.} 
\begin{align*}
% changed 2, kb, 0 subscript into kb, y and removing dst
\mu_{Y} =& H_{\lt}^*(\ring  \mid \mid \tlt \mid \mid \underline{\widehat{Y}}), \text{ and} \\
\mu_{k} =& H_{\lt,k}(\ring \mid \mid \tlt \mid \mid \underline{\widehat{Y}})\text{ for each }k \in [d-1],
\end{align*}






\item the aggregated linking tag
$\mathfrak{W} = \mu_{Y} \mathfrak{T} + \sum_{k=1}^{d-1} \mu_{k} \mathfrak{D}_k$,



\item for each $j \in [m]$, the $j^{th}$ aggregated ring member $W_j = \mu_Y Y_j^\prime + \sum_{k=1}^{d-1} \mu_k Z_{j,k}^\prime$,




\item For $j^* < j \leq m$, compute the nonces and signature challenges:
\begin{align*}
L_j =& s_j G + c_j W_j \\ R_j =& s_j \widehat{Y}_j + c_j \mathfrak{W} \\
c_{j+1} =& H_{\texttt{ch}}\signaturequery % LINE 269 IN KAYABA'S RINGCT CODE
\end{align*}

\item Set $c_1 = c_{m+1}$ and, for $1 \leq j < j^*$, compute the nonces and signature challenges as in the previous step.

\end{enumerate}

\item Set $s_{j^*, i^*} = d + \rho_{i^*} e + \lambda_{i^*} \cdot c_{j^*} \cdot w_{i^*}$.

\item Output $\psig = (i^*, \lt_{i^*}, c_1, s_1, \ldots, s_{j^*-1}, s_{j^*, i^*}, s_{j^*+1}, \ldots s_m)$.
\end{enumerate}

\item $\combine(\msg, \ring, \psigs) \to \sig$. Input a message $\msg$, a ring $\ring$, and signature shares $\psig_1, \ldots, \psig_r$, and output a ring signature $\sig$. Do the following.
\begin{enumerate}
\item Parse $(\vk_1, \ldots, \vk_r) := \VK$, $(Y_i, Z_{i,1}, \ldots, Z_{i,d-1}) := \vk_i$ for $i \in [r]$, and $Y := \tvk$. Parse $(\tvk_1, \ldots, \tvk_m) := \ring$ and find the index $1 \leq j^* \leq m$ in $\ring$ such that $Y  \tvk_{j^*}$. Otherwise, output a distinct failure symbol and terminate.

\item Parse each $(i^\prime_i, \lt_i, c_{i,1}, (s_{i,j})_{j=1}^{m}) := \psig_i$ for each $i \in [r]$. 

\item If there exists indices $i_1 \neq i_2$ such that $\lt_{i_1} = \lt_{i_2}$, output a distinct failure symbol and terminate. 

\item Otherwise, if there exists indices $i_1 \neq i_2$ any signature challenges mismatch, $c_{i_1,1} \neq c_{i_2, 1}$, output a distinct failure symbol and terminate.

\item Otherwise, for any $1 \leq i_1, i_2 \leq r$, $1 \leq j \leq m$, if $j \neq j^*$ and $s_{i_1,j} \neq s_{i_2, j}$, output a distinct failure symbol and terminate.

\item Otherwise, set $c_1 = c_{1,1}$, set $\widehat{s}_j = s_j$ for each $j \neq j^*$, set $\widehat{s}_{j^*} = \sum_{i=1}^{r} s_{i,j^*}$, and output the signature $\sig = (c_1, (\widehat{s})_{j=1}^{m}, \tlt)$
\end{enumerate}

\item $\verify(\msg, \ring, \sig) \to b$. Input a message $\msg$, a ring $\ring = (\tvk_1, \ldots, \tvk_m)$, and a signature $\sig$. Output a bit. Works as follows.
\begin{enumerate}
\item Parse $(c_1, s_1, \ldots, s_m, \mathfrak{T}, \mathfrak{D}_1, \ldots, \mathfrak{D}_{d-1}) := \sig$. 
\item If $\mathfrak{T} = 0$ or any $\mathfrak{D}_{d-1} = 0$, output $0$ and terminate.
\item Using $j^* = 1$, execute step \cref{step:commonstuff} in $\sign$.
\item If $c_1 = c_{m+1}$, output $1$ and terminate; otherwise output $0$ and terminate.
\end{enumerate}


\item $\link(\sig, \sig^\prime) \to b$. Do the following.
\begin{enumerate}
\item Parse the following.
\begin{enumerate}
\item $(c_1, s_1, \ldots, s_m, \tlt) := \sig$, $(c_1^\prime, s_1^\prime, \ldots, s_m^\prime, \tlt^\prime) := \sig^\prime$,
\item $(\mathfrak{T}, \mathfrak{D}_1, \ldots, \mathfrak{D}_{d-1}) := \lt$,  $(\mathfrak{T}^\prime, \mathfrak{D}_1^\prime, \ldots, \mathfrak{D}_{d-1}^\prime) := \lt^\prime$. 
\end{enumerate}
\item Output a bit indicating whether $\mathfrak{T} = \mathfrak{T}^\prime$.%\signaturequery
\end{enumerate} 


\end{enumerate}
\end{definition}

Beware that siganture shares leak information, like the true signer's ring index. Executing $\verifyshare$ or $\combine$ can only be safely done with other signers.


\subsection{Extensions and Additional Algorithms}\label{sec:extensions}



FROSTLASS (\cref{def:frostlass}) can be made verifiable as described in \cref{sec:definitions} with a $\verifyshare$ algorithm as follows.

\begin{itemize}
\item $\verifyshare(\msg, \ring, \VK, \psig, \tlt, \texttt{com}) \to b$ inputs some $\msg$,  $\ring = (\tvk_1, \ldots, \tvk_m)$,
signers' coalition of key shares $\VK = (\vk_i)_{i=1}^{r}$, a signature share $\psig=(i^*, \lt, c_1, s_1, \ldots, s_m)$, a total linking tag $\tlt$, and a hash table $\texttt{com}$, and outputs a bit. Do the following.
\begin{enumerate}
\item Set $j^* = 1$.
\item Carry out step c of $\sign$.
\item Parse $(i^*, \lt, c_1, s_1, \ldots, s_m) := \psig$.
\item Carry out step d of $\sign$.
\item Carry out step e of $\sign$. If any $s_j$ mismatch their corresponding element in $\psig$, output $0$ and terminate.
\item Carry out step f of $\sign$, to obtain each $c_j^\prime$.
\item If $c_1 \neq c_1^\prime$, or $s_{j^*, i^*}G \neq \lambda_{i^*} c_{i^*} Y_{i^*}$, or $s_{j^*, i^*}\widehat{Y}_{j^*} \neq \lambda_{i^*}c_{j^*} \mathfrak{W}$, output $0$ and terminate.
\item Otherwise, output $1$ and terminate.
\end{enumerate}
\end{itemize}


FROSTLASS also admits a pre-processing step wherein participants may commit to their auxiliary signing data ahead of time and compute their total linking tag $\tlt$, which works as follows.

\begin{itemize}
\item $\preproc(\tvk, \VK, \SK) \to (\lt,  \texttt{com})$. An interactive PPT algorithm required to execute $\sign$ and $\verifyshare$, and which requires some $r \geq 1$ participants and a digital signature scheme $\Pi_{DSS}$ as a subroutine. Input total verification key $\tvk$, signers' coalition of public verification key shares $\VK = (\vk_i)_{i=1}^{r}$, and secret signing key shares $\SK = (\sk_i)_{i=1}^{r}$. Output a hash table $\texttt{com}$. 
 Participants do the following. 
\begin{enumerate}
\item Parse $(\mathfrak{T}_i, \underline{\mathfrak{D}}^{(i)}) := \lt_i$. If any $\underline{\mathfrak{D}}^{(i)} \neq \underline{\mathfrak{D}}^{(i^\prime)}$, output $\bot$ and terminate. 
\item Otherwise, carry out step d.ii and d.iii from $\sign$ to compute the Lagrange coefficients $\lambda_i$ and the linking tag $\mathfrak{T} = \sum_i \mathfrak{T}_i$.
\item Compute the point hash of the linking key $\widehat{Y} = H_{\texttt{base}}(Y)$.
\item For each $i \in [r]$, the $i^{th}$ participant samples $(d_i, e_i) \in \Zq^2$ and computes the following:
\begin{align*}
D_i &= d_i G, & D_i^\prime &= d_i \widehat{Y}, \\
E_i &= e_i G,\text{ and} &
E_i^\prime &= e_i \widehat{Y}.
\end{align*}
\item The $i^{th}$ participant sends $(\lt_i, D_i, E_i, D^\prime_i, E^\prime_i)$ in an authenticated all-to-all broadcast to the other signers\footnote{Equivalently, all users may send their commitment points to a single member, who may then broadcast $\texttt{com}$ back to all other users after executing step 3 of \texttt{PreProc}}.
% \item All participants compute $\mathfrak{T} = \sum_i \mathfrak{T}_i$
\item After using $\Pi_{DSS}$ to verify this communication, participants set $\texttt{com}$ to be a hash table with keys $\vk_i$ and values $\texttt{com}[\vk_i] = (D_i, E_i, D^\prime_i, E^\prime_i)$.
\end{enumerate}
\end{itemize}

% We can think of $\preproc$ as a way to break signing up into two rounds, where $prepro$

FROSTLASS also only links ring signatures according to whether the linking tags $\mathfrak{T}$ match, where $\mathfrak{T}$ is a collision-resistant image of the link of the signing key. That is, linking is not bound to the message, the other keys $Z_k$ of the signing key, or the ring. Variations of this scheme binding linking to more data can provide a hierarchical expansion of linkability; this may be a fruitful area of further research.


\subsection{Concrete Instantiation}\label{sec:concrete_instantiation}


To concretely implement random oracles in practice, we employ hash functions $H_{\G}: \bitstrings \to \G$, $H_{\Zq}:\bitstrings \to \Zq$, and $H_{\secpar}:\bitstrings \to \left\{0,1\right\}^\secpar$ with distinct domain separating tags $\dst_{\texttt{label}} \in \bitstrings$. For example, we concretely instantiate $H_{\texttt{base}}$ by mapping $x \mapsto H_{\G}(\dst_{\texttt{base}} \mid \mid x)$. Note that we can avoid domain separating tags and associated implementation errors without losing efficiency by using an extendable output function $\texttt{xof}$ instead of hash functions throughout, extracting $(\mu_Y, \mu_1, \ldots, \mu_{d-1}) \leftarrow \texttt{xof}(\vk_{j^*})$ directly with one function call.

\subsection{Variations From Older Versions of CLSAG and the Rust Implementation}\label{sec:variations_from_implem}

The scheme \cref{def:frostlass} varies from both older versions of CLSAG and the Rust implementation at \cite{SeraiRepo} in a few ways.

\begin{itemize}
\item We try to strictly follow the ``hash the complete transcript'' paradigm to prevent malleability in \cref{def:frostlass}. This causes some significant variations from previous versions of CLSAG and the Rust implementation; much more data is included in our hash pre-images.
\begin{itemize} 
\item For example, we include the point hashes of the ring members $\underline{\widehat{Y}}$ when computing aggregation coefficients in g.ii of $\sign$. This prevents an adversary from attempting to pick some aggregation coefficients before selecting a ring. 
\item Similarly, we also include the aggregated key image $\mathfrak{W}$ in the preimage of every signature challenge computation. This prevents an adversary from attempting to pick a signature challenge before deciding upon an aggregated key image. 
\end{itemize}
This binding prevents malleability. We see no obvious way to violate any of our security properties without taking such care, but we do so as a matter of good practice for the formal definition of the scheme. Practical implementations do not need to be quite so stringent. For example, although CLSAG signatures do not usually compute $c_{j+1}$ with the total linking tag $\tlt$ included in the pre-image, the most notable application of CLSAG signatures (the Monero cryptocurrency) includes these data in the message being signed. By including these data in $\msg$, those applications essentially enforce (some of) the ``include all data in all hashes'' paradigm.

\item The order of our computations in \cref{def:frostlass} is not necessarily faithful to the order of computations in \cite{SeraiRepo}.  Our description in \cref{def:frostlass} is ordered in a way which makes cross-referencing within this document easier, improving the readability of our description of $\verify$ and $\verifyshare$ substantially. None of our variations from \cite{SeraiRepo} cause security problems and are rather superficial.

\item All hashing in \cite{SeraiRepo} is deterministically computed from transcript data. The approach in this repo largely follows the ``hash the complete transcript'' paradigm. Indeed, the Rust implementation appends data to running transcripts constructed in a canonical way.  However, this transcript is pruned whenever data can be deterministically and verifiably computed from the state of the transcript at some point. For example, computing $x = H(\msg)$, $y = H(0 \mid \mid x)$, and $z = H(1 \mid \mid x)$ is safe; we do not need to compute $z = H(1 \mid \mid y \mid \mid x)$. This approach is consistent with, e.g.\ approaches in IETF standards like RFC 9591.

 % For example, \cite{SeraiRepo} does not include $\tlt$ in the pre-image for computing $\mu_Y$, allowing an adversary to attempt different choices of $\tlt$ with the same choice of $\mu_Y$. No known attack methods can exploit this malleability, 

\end{itemize}



% For example, our scheme in \cref{def:frostlass} computes the seed $\gamma \leftarrow H_{\texttt{seed}}(\VK \mid \mid \ring \mid \mid \tlt \mid \mid \msg \mid \texttt{com})$. On the other hand, \cite{SeraiRepo} computes $\seed \leftarrow H_{\texttt{seed}}(\texttt{transcript})$, where $\texttt{transcript}$ consists of all relevant data used by the signers' coalition up until that point, but is not easily human-readable. This includes the signers' coalition of keys $\VK$, the anonymity tuple $\ring$, the total linking tag $\tlt$, the message $\msg$, and the hash table $\texttt{com}$. Thus, $\gamma$ is bound to all these data. Unfortunately, \texttt{transcript} is not easily human-readable, but $\texttt{transcript}$ contains more data than this when $\texttt{seed}$ is computed. Thus, the result $\gamma$ is bound to the data in $(\VK \mid \mid \ring \mid \mid \tlt \mid \mid \msg \mid \texttt{com})$ as required.
The same can be said of all other hashes and randomness; this enforces a degree of semi-honesty among users of this Rust implementation.

% The same can be said of the FROST aggregation coefficients. However, there are some idiosyncrasies related to the transcript here. Indeed, the Rust implementation at \cite{SeraiRepo} uses the IETF standard for FROST, which uses a transcript in a different format to compute aggregation coefficients. Thus, at this point, $\texttt{transcript}$ is forked to form an IETF-friendly $\texttt{transcript}^\prime$ to obtain FROST keys. 



\section{FROSTLASS Security}


Correctness and linkability depend on the following lemmata, wherein we intentionally conflate linking tags $\mathfrak{T}$ with a function mapping a linking key to its linking tag. Recalling $H_{\G}$ was a hash function modeled as a random oracle (see \cref{sec:concrete_instantiation}), this lemma establishes that, as a corollary, $\mathfrak{T}$ is indistinguishable from a random oracle.


\begin{lemma}\label{lem:link_tags_are_oracles}
Let  $\dst \in \bitstrings$ be a domain separating tag, let $\theta:\G \to \G$ be any function, let $\phi: \Zq \times \G \to \G$ be the function defined by mapping $(x, Y) \mapsto x \cdot \theta(Y)$, and let $t_{\texttt{scmul}}$ denote the time it takes to multiply a point in $\G$ by a scalar in $\Zq$. If some PPT $(t,\epsilon)$ algorithm $\mathcal{A}$ (or $\mathcal{B}$, respectively) is a $\kappa$-distinguisher for $\phi$ (or $\theta$, respectively) under definition \cref{def:distinguisher}, then there exists a PPT $(t^\prime, \epsilon^\prime)$ algorithm $\mathcal{A}^\prime$ (or $\mathcal{B}^\prime$, respectively) which is a $\kappa$-distinguisher for $\theta$ (or $\phi$, respectively).
\end{lemma}
\begin{proof}
Assume the algorithm $\mathcal{A}$ can distinguish $\phi$ from a random oracle in \cref{def:distinguisher}. We build a $\mathcal{A}^\prime$ to distinguish $\theta$ as follows.
\begin{enumerate}
\item $\mathcal{A}^\prime$ is granted oracle access to $\mathcal{O}^\prime_b:\G \to \G$.
\item $\mathcal{A}^\prime$ runs $\mathcal{A}$ as a subroutine, handling oracle queries as follows. When $\mathcal{A}$ sends some query $(x, Y) \in \Zq \times \G$, $\mathcal{A}^\prime$ computes $Y^\prime \leftarrow \mathcal{O}^\prime_b(Y)$, sets $Z = xY^\prime$, and responds with $Z$.
\item When $\mathcal{A}$ outputs $b^\prime$, $\mathcal{A}^\prime$ outputs $b^\prime$.
\end{enumerate}
It is clear that this $\mathcal{A}^\prime$ correctly plays the $\kappa$-random oracle distinguisher game for $\theta$, succeeds if and only if $\mathcal{A}$ succeeds at distinguishing $\phi$, and takes only the additional time to compute $xY^\prime$ from $x$ and $Y^\prime$, i.e.\ a single scalar multiplication.

Likewise, assume $\mathcal{B}$ can distinguish $\theta$ from a random oracle. We build $\mathcal{B}^\prime$ similarly.
\begin{enumerate}
\item $\mathcal{B}^\prime$ is granted oracle access to $\mathcal{O}^\prime_b:\Zq \times \G \to \G$.
\item $\mathcal{B}^\prime$ runs $\mathcal{B}$ as a subroutine, handling oracle queries as follows. When $\mathcal{B}$ sends some query $Y \in \G$, $\mathcal{B}^\prime$ samples $x \sample \Zq$, computes $Y^\prime \leftarrow \mathcal{O}^\prime_b(x, Y)$, sets $Z = x^{-1}Y^\prime$, and responds with $Z$.
\item When $\mathcal{B}$ outputs $b^\prime$, $\mathcal{B}^\prime$ outputs $b^\prime$.
\end{enumerate}
It is also clear that this $\mathcal{B}^\prime$ correctly plays the $\kappa$-random oracle distinguisher game for $\phi$, succeeds if and only if $\mathcal{B}$ succeeds at distinguishing $\theta$, and takes additional time for sampling, inverting an element from $\Zq$, and multiplying a point by a scalar, i.e. extra time $t_{\texttt{sample}} + t_{\texttt{inv}} + t_{\texttt{scmul}}$.
\end{proof}


\begin{cor}\label{cor:key_images_are_random_oracles}
Let $r \geq 1$. Each of the maps $\mathfrak{T}:\Zq \to \G$ and $\mathfrak{T}_i:\Zq^r \times\G\to\G$ defined by mapping $y \mapsto yH_{\texttt{base}}(yG)$ and $(\underline{y}, Y) \mapsto y_i H_{\texttt{base}}(Y)$ are indistinguishable from random oracles.
\end{cor}

\subsection{Correctness}



\begin{theorem}
FROSTLASS is a correct LTM scheme under \cref{def:correctness}.
\end{theorem}
\begin{proof}
In event $E_2$, consider the ring signature shares $\psig_i^\prime$, $\psig_i^{\prime \prime}$, and $\psig_i^*$. These have some corresponding indices $j^\prime, j^{\prime \prime}$, and $j^*$, respectively, and we write these ring signature shares as follows.
\begin{align*}
\forall i \in \left[\left|\VK^\prime\right|\right], \psig_i^\prime =& (i, c_1^\prime, s_1^\prime, \ldots, s_{j^*-1}^\prime, s_{j^*, i}^\prime, s_{j^*+1}^\prime, \ldots, s_m^\prime),  \\
\forall i \in \left[\left|\VK^{\prime\prime}\right|\right], \psig_i^\prime =& (i, c_1^\prime, s_1^\prime, \ldots, s_{j^*-1}^{\prime\prime}, s_{j^*, i}^{\prime\prime}, s_{j^*+1}^{\prime\prime}, \ldots, s_m^{\prime\prime}),  \\
\forall i \in \left[\left|\VK^*\right|\right], \psig_i^* =& (i, c_1^*, s_1^*, \ldots, s_{j^*-1}^*, s_{j^*, i}^*, s_{j^*+1}^*, \ldots, s_m^*)
\end{align*}  

Each signer with index $i \in \left[\left|\VK^\prime\right|\right]$, computes the same seed, say $\gamma^\prime$ in event $E_1$, so $(s_j^\prime)_{j \neq j^{\prime}, i}$ is identical in each $\psig_i^\prime$, where $j^\prime$ is the ring index of the true signer for $\psig_i^\prime$. Similarly, each signer with index $i \in \left[\left|\VK^{\prime\prime}\right|\right]$ computes the same seed, say $\gamma^{\prime \prime}$, and $(s_j^{\prime \prime})_{j \neq j^{\prime \prime}, i}$ is identical in each $\psig_i^{\prime \prime}$, where $j^{\prime \prime}$ is the ring index of the true signer for $\psig_i^{\prime \prime}$. In event $E_1^*$, each signer with index $i \in \left[\left|\VK^{**}\right|\right]$ 
compute the same seed $\gamma^*$, and $(s_j^*)_{j \neq j^*, i}$ is identical in each $\psig_i^*$, where $j^*$ is the index of the true signer of $\psig_i^*$. Moreover, these ring signature shares were all output from honest executions of $\sign$. 

To show $\mathbb{P}[E_2] = 1$, it is sufficient to demonstrate that each $\psig_i^\prime$ passes $\verifyshare$ in $E_1$. Indeed, the ring signature shares $\psig_i^{\prime \prime}$ and $\psig_i^*$ are shown to be valid in a similar way, \textit{mutatis mutandis}. 

In $E_1$, the points $L_{j^\prime}^\prime = \sum_i F_i$ and $R_{j^\prime}^\prime = \sum_i F_i^\prime$ are computed along with the starting signature challenge $c_{j^\prime+1}$.
% not removing dst because it's a challenge
% \[c_{j^\prime+1}^\prime = H_{\texttt{ch}}(\dst_{j^\prime} \mid \mid \ring \mid \mid L_{j^\prime}^\prime \mid \mid R_{j^\prime}^\prime \mid \mid \msg^\prime)\]
Then, for $j^* < j \leq m$, the following computations take place.
\begin{align*}
L_j^\prime =& s_j^\prime G + c_j^\prime W_j^\prime \\
R_j^\prime =& s_j^\prime \widehat{Y}_j^\prime + c_j^\prime \mathfrak{W} \\
c_{j+1} =& H_{\texttt{ch}}(\texttt{dst}_{j} \mid \mid \ring \mid \mid \tlt \mid \mid \underline{\widehat{Y}} \mid \mid \mathfrak{W} \mid \mid \underline{W} \mid \mid \underline{\mu} \mid \mid L_{j} \mid \mid R_{j} \mid \mid \msg^\prime)
\end{align*}
Then the value $c_1 = c_{m+1}$ is set and, for $1 \leq j \leq j^*$, the same computations for $L_j^\prime, R_j^\prime, c_j^\prime$ take place. Lastly, each $s_{j^\prime, i}^\prime = d_i^\prime + \rho_i^\prime e_i^\prime + \lambda_i^\prime c_{j^\prime}^\prime y_i^\prime$ for the random scalars $d_i^\prime, e_i^\prime$, the corresponding aggregation coefficient $\rho_i^\prime$, Lagrange interpolation coefficient $\lambda_i^\prime$, and secret signing key share $y_i^\prime$.
Thus, in $E_1$, the verifier computes $\gamma^\prime$ the same as the signer, and so samples $(s_j^\prime)_{j\neq j^\prime,i}$ identically to all the signers. Moreover, for each $1 \leq j \leq m^\prime$, 
the signature nonce points satisfy $L_j^\prime = s_j^\prime G + c_j^\prime W_j^\prime$, $R_j^\prime = s_j^\prime \widehat{Y}_j^\prime + c_j^\prime \mathfrak{W}$, and the signature challenges satisfy the verification equations by construction, except $j=j^\prime$. 
% For this index and each $1 \leq i \leq \left[\left|\VK^\prime\right|\right]$, $s_{j^\prime, i}G = \lambda_i^\prime c_{j^\prime}^\prime Y_i$ and we have the following by construction. 
% \[s_{j^\prime, i} H_{\texttt{base}}(\dst_{\preproc} \mid \mid \VK \mid \mid \lt \mid \mid \msg \mid \mid \ring) = \lambda_i^\prime c_{j^\prime}^\prime \widehat{Y}_i\]

In event $E_3$, we have the following combined signatures.
\begin{align*}
\sig^\prime &= (c_1^\prime, s_1^\prime, \ldots, s_{j^\prime - 1}^\prime, \sum_i s_{j^\prime, i}^\prime, s_{j^\prime + 1}^\prime, \ldots, s_m^\prime, \lt^\prime) \\
\sig^{\prime \prime} &= (c_1^{\prime \prime}, s_1^{\prime \prime}, \ldots, s_{j^{\prime \prime} - 1}^{\prime \prime}, \sum_i s_{j^{\prime \prime}, i}^{\prime \prime}, s_{j^{\prime \prime} + 1}^{\prime \prime}, \ldots, s_m^{\prime \prime}, \lt^{\prime \prime}) \\
\sig^* &= (c_1^*, s_1^*, \ldots, s_{j^* - 1}^*, \sum_i s_{j^*, i}^*, s_{j^* + 1}^*, \ldots, s_m^*, \lt^*) 
\end{align*}
Moreover, the aggregation coefficients, aggregated ring members, point hashes of ring members' leading keys, and the seed are all computed exactly as in $\sign$ and $\verifyshare$. So, by construction, $c_1 = c_{m+1}$ and the circle of hashes pass verification.

Lastly, consider events $E_4$, $E_5$, and $E_6$. $\link$ merely compares the linking tags. Moreover, as $\link$ is a check for equality of linking tags, it is necessarily commutative.
In event $E_5$, $\sig^\prime$ and $\sig^{\prime \prime}$ are both computed from superthreshold subsets of $\VK$ with the same corresponding $\lt$, and so have identical linking tags $\mathfrak{T}$. 

In event $E_6$, on the other hand, the signatures are computed for distinct $\tvk \neq \tvk^\prime$. By  \cref{cor:key_images_are_random_oracles}, $\mathfrak{T}(\tvk) \neq \mathfrak{T}(\tvk^\prime)$ except with negligible probability, so $\link(\sig^\prime, \sig^*) = \link(\sig^*, \sig^{\prime \prime}) = 0$ except with negligible probability.\end{proof}


\subsection{Strong Unforgeability}

\begin{theorem}\label{thm:suf}
Let $\kappa_{\texttt{ch}}, \kappa_{\texttt{key}}, n_{\texttt{key}} \geq 1$ be integer parameters. For every PPT $(t, \epsilon)$-forger $\mathcal{A}$ as described in \cref{def:suf}, there exists a PPT $(t^\prime, \epsilon^\prime)$-player of the $(n_{\texttt{key}}\kappa_{\texttt{key}} - 1)$-OMDL game such that $t^\prime \in O(2t)$ and $\epsilon^\prime \in O(\frac{\epsilon^2}{\kappa_{\texttt{ch}}})$.
\end{theorem}
\begin{proof}
We solve the $\kappa$-OMDL game by constructing a tower of algorithms $\mathcal{A}_4 \rightarrow \mathcal{A}_3 \rightarrow \mathcal{A}_2 \rightarrow \mathcal{A}_1$, where $\mathcal{A}_1 = \mathcal{A}$ is a forger, $\mathcal{A}_2$ is a simulator of the unforgeability challenger for $\mathcal{A}_1$, $\mathcal{A}_3 = \texttt{Fork}_{\mathcal{A}_2}$ is the forking algorithm of \cref{def:general_forking_algorithm}, and $\mathcal{A}_4 = \mathcal{A}^\prime$ plays the $\kappa$-OMDL game. These arrows indicate $\mathcal{A}_4$ runs $\mathcal{A}_3$ as a subroutine, and so on. We discuss these in order beginning with $\mathcal{A}_1$, the forger.

\textbf{The forger.} Let $\mathcal{A}_1$ be a $(t_1, \epsilon_1)$-algorithm which is an LTM strong forger of FROSTLASS as described in \cref{def:suf} and runs with some random tape $\tau_{\mathcal{A}_1}$. $\mathcal{A}_1$ has access to the oracles $\keyOracle$, $\corruptionOracle$, and $\signingOracle$ from \cref{def:common_setup} via \cref{def:suf}, 
and all the random oracles $H_{\texttt{label}}$ with $\texttt{label} \in \left\{\texttt{base}, \texttt{seed}, (\texttt{FROST}, i), (\texttt{kb}, k), \texttt{ch}\right\}$ for $i \in \mathbb{N}$ and $k \in [d-1]$ from \cref{def:frostlass}. 

\textbf{Wrap the forger.}  We first wrap $\mathcal{A}_1$ in an algorithm $\mathcal{A}_2$. This $\mathcal{A}_2$ simulates the forgery challenger for $\mathcal{A}_1$ and is compatible with \cref{def:general_forking_algorithm}, and is a helper algorithm for playing \cref{def:omdl} which requires oracle access; we denote the oracles of \cref{def:omdl} with $\keyOracle^*$ and $\corruptionOracle^*$ to prevent confusion. $\mathcal{A}_2$ works as follows. 
\begin{enumerate}
\item Initialize empty tables $T_{\texttt{label}}$ for $\texttt{label} \in \left\{\texttt{base}, \texttt{seed}, (\texttt{FROST},i), (\lt, k), \texttt{ch}, \texttt{DL}\right\}$.
\item Run $\mathcal{A}_1$ as a subroutine. As a simulator of the game of \cref{def:suf}, $\mathcal{A}_2$ handles all oracle queries made by $\mathcal{A}_1$ as follows.
\begin{enumerate}
\item For $\texttt{label} \in \left\{\texttt{seed}, (\texttt{FROST},i), (\lt, y), (\lt, k) \mid i \in [n_{\texttt{key}}], k \in [d-1]\right\}$, when $\mathcal{A}_1$ queries $H_{\texttt{label}}$, $\mathcal{A}_3$ simulates responses using its own internal random tape, resampling in the event of a collision, and storing query-response pairs as key-value pairs in hash tables $T_{\texttt{label}}$ to maintain consistency with later queries. We assume handling these queries requires no other oracle queries, takes negligible time, and certainly succeeds. These simulations are indistinguishable from real oracles, as they are directly simulated from the random tape of $\mathcal{A}_2$.


\item When $\mathcal{A}_1$ makes some $\texttt{query}$ to $H_{\texttt{ch}}$ and $\texttt{query} \notin T_{\texttt{ch}}$, $\mathcal{A}_2$ computes $i = \left|T_{\texttt{ch}}\right|+1$, finds $h_i \in \underline{h}$, stores $T_{\texttt{ch}}[\texttt{query}] = (i, h_i)$, and responds with $h_i$.  We assume handling these queries requires no other oracle queries, takes negligible time, and succeeds with certainty.

\item When $\mathcal{A}_1$ makes some $\texttt{query}$ to $H_{\texttt{base}}$, $\mathcal{A}_2$ checks if $\texttt{query} \notin T_{\texttt{base}}$. If so, $\mathcal{A}_2$ samples $\alpha \leftarrow \Zq$, resampling in the case of a collision, and sets $T_{\texttt{base}}[\texttt{query}] = \alpha$. The response is computed in two cases.
\begin{enumerate}
\item If $\texttt{query} \notin \mathbb{G}$, then $\mathcal{A}_2$ samples $Y \leftarrow \mathbb{G}$ and responds with $\alpha Y$.
\item Otherwise, $\mathcal{A}_2$ parses $Y \leftarrow \texttt{query}$, and responds with $\alpha Y$.
\end{enumerate}
This query requires no further oracle access, takes the time to sample $\alpha \sample \Zq$, possibly the time it takes to sample $Y \in \G$, and the time it takes to compute $\alpha Y$, a scalar multiplication of a point. Thus, this query takes time at most $t_{\texttt{base}} \approx t_{q} + t_\mathbb{G} + t_{\texttt{scmul}}$, where $t_q$ is the time it takes to sample $\alpha$, $t_\mathbb{G}$ is the time it takes to sample $Y \in \mathbb{G}$, and $t_{\texttt{scmul}}$ is the time it takes to compute $\alpha Y$. This query certainly succeeds.


\item When $\mathcal{A}_1$ queries $\keyOracle$ with some pair $(n,r)$, $\mathcal{A}_2$ does the following.
\begin{enumerate}
\item If $r \notin [n]$, output a distinct failure symbol and terminate.
\item Otherwise, for each $i \in [n]$, make a query $Y_i \leftarrow \keyOracle^*(\ast)$ from \cref{def:omdl}.
\item Compute $Y = \sum_{i \in [n]} Y_i$.
\item Sample $(z_1, \ldots, z_{d-1}) \leftarrow \Zq^{d-1}$.
\item Compute each $Z_k = z_k G$ and store $T_{\texttt{DL}}[Z_k] = z_k$.
\item Simulate a query made to $H_{\texttt{base}}$ by $\mathcal{A}_1$, $\widehat{Y} = H_{\texttt{base}}(Y)$.
\item Retrieve $\alpha = T_{\texttt{base}}[Y]$; this table entry is non-empty with certainty due to the previous step.
\item Set $\mathfrak{T} = \alpha Y$, $\mathfrak{T}_i = \alpha Y_i$, and each $\mathfrak{D}_k = \alpha Z_k$ for each $i \in [n]$ and each $k \in [d-1]$.

\item Set $\tvk = (Y, Z_1, \ldots, Z_{d-1})$, $\lt = (\mathfrak{T}, \mathfrak{D}_1, \ldots, \mathfrak{D}_{d-1})$, each $\vk_i = (Y_i, Z_1, \ldots, Z_{d-1})$, and each $\lt_i = (\mathfrak{T}_i, \mathfrak{D}_1, \ldots, \mathfrak{D}_{d-1})$. 
\item Store $T_{\lt}[\vk_i] = \lt_i$ and $T_{\tlt}[\vk] = \tlt$. 
\item Respond with $(\tvk, \VK)$. 
\end{enumerate}
Handling one $\keyOracle$ query takes $n$ queries to $\keyOracle^*$ and one query to $H_{\texttt{base}}$, $n^2$ sums of points from $\G$, $(d-1)$ samples from $\Zq$, and $2d+1$ scalar multiplications against points. This query takes time at most $t_{\texttt{key}} \approx nt_{\texttt{key}}^* + n^2 t_+ + (d-1)t_{q} + (1+2d)t_{\texttt{scmul}} + t_{\texttt{base}}$, where $t_{\texttt{key}}^*$ is the time it takes to query $\keyOracle^*$, $t_+$ is the time it takes to sum two arbitrary group elements, and $t_{\texttt{base}}$ is the time it takes to simulate a query $H_{\texttt{base}}$. This query succeeds with certainty.

\item When $\mathcal{A}_1$ queries $\corruptionOracle$ with some $(i, (\tvk, \VK))$, $\mathcal{A}_2$ parses $(Y_i, Z_1, \ldots, Z_{d-1}) := \vk_i$. If this is not possible, then $\corruptionOracle$ responds with a distinct failure symbol. 
Otherwise, $\mathcal{A}_2$ looks up $z_k \leftarrow T_{\texttt{DL}}[Z_k]$ for each $k$. If this is not possible, then $\corruptionOracle$ responds with a distinct failure symbol. Otherwise, $\mathcal{A}_2$ queries $y_i \leftarrow \corruptionOracle^*(Y_i)$, sets $T_{\texttt{corrupt}}[Y_i] = y_i$, and responds with $(y_i, z_1, \ldots, z_{d-1})$.

This query takes one query to $\corruptionOracle^*$ and $d-1$ retrievals from $T_{\texttt{DL}}$. As this is a hash table, lookups are constant-time, so this query takes time at most $t_{\texttt{corrupt}} = t_{\texttt{corrupt}}^* + (d-1)O(1)$.  Moreover, this query fails if and only if $\mathcal{A}_1$ makes a query that is not a challenge key. By assumption, $\mathcal{A}_1$ prefers to fail than to make a failed corruption oracle query, so without loss of generality, this corruption oracle certainly succeeds.

This is also the only oracle which may fail if queried incorrectly, say with non-challenge data.

\item When $\mathcal{A}_1$ queries $\signingOracle$ with some $(\msg, \ring, \tvk, \VK, \vk)$, $\mathcal{A}_2$ does the following to back-patch an ostensibly valid signature.
\begin{enumerate}
\item If $(\tvk, \VK^\prime)$ does not appear as a $\keyOracle$ response to $\mathcal{A}_1$ for any $\VK \subseteq \VK^\prime$, or $\tvk \notin \ring$, output a distinct failure symbol and terminate.
\item Otherwise, there is some query made by $\mathcal{A}_1$ to $\keyOracle$ with keys matching this $\signingOracle$ query, say $(\tvk, \VK^\prime) \leftarrow \keyOracle(n,r)$ occurred for some $\VK \subseteq \VK^\prime$. If $\left|\VK\right| < r$, or $\left|\VK^\prime\right| \neq n$, output a distinct failure symbol and terminate.  
\item Otherwise, there is a superthreshold number of signers in the coalition and the correct total number of keyholders. Parse $(Y, Z_1, \ldots, Z_{d-1}) := \tvk$.
\item Retrieve $\alpha = T_{\texttt{base}}[Y]$, then compute $\mathfrak{T} = \alpha Y$ and each $\mathfrak{D}_k = \alpha Z_k$. Set $\tlt = (\mathfrak{T}, \mathfrak{D}_1, \ldots, \mathfrak{D}_{d-1})$.
% Removing dst: 
\item Simulate a query made to $H_{\texttt{base}}$ by $\mathcal{A}_1$, say $\widehat{Y}_j = H_{\texttt{base}}(Y_j)$, for each ring members' linking keys $Y_j$ (i.e.\ for each $j \in [m]$).
\item Simulate a query made to $H_{\lt}^*$ and $H_{\lt, k}$ from $\mathcal{A}_1$ for each $k \in [d-1]$ to obtain $\mu_Y = H_{\lt}^*(\ring \mid \mid \tlt \mid \mid \underline{\widehat{Y}})$ and $\mu_k = H_{\lt, k}(\ring \mid \mid \tlt \mid \mid \underline{\widehat{Y}})$ for each $k \in [d-1]$.
\item Compute $W_j = \mu_Y Y_j + \sum_k \mu_k Z_{j,k}$ for each ring member $\tvk_j = (Y_j, Z_{j,1}, \ldots, Z_{j,d-1})$.
\item Compute $\mathfrak{W} = \mu_Y \mathfrak{T} + \sum_k \mu_k \mathfrak{D}_k$.
\item Sample $s_1, \ldots, s_m \leftarrow \Zq$.
\item Retrieve $i^* = \left|T_{\texttt{ch}}\right| + 1$ and the challenges $h_{i^*}, h_{i^*+1}, \ldots, h_{i^*+m-1} \leftarrow \underline{h}$.
\item Find $j^* \in [m]$ such that $\tvk_{j^*} = \tvk$.
\item Set the following:
\begin{align*}
c_{j^*+1} &= h_{i^*} & c_m &= h_{i^* + (m-j^* - 1)} \\
c_{j^*+2} &= h_{i^*+1} & c_1 &= h_{i^* + (m-j^*)} \\
c_{j^*+3} &= h_{i^*+2} & c_2 &= h_{i^* + (m-j^*) + 1} \\
\vdots & & \vdots & \\
c_{m-2} &= h_{i^* + (m-j^*-3)} & c_{j^* - 1} &= h_{i^* + (m-2)} \\
c_{m-1} &= h_{i^* + (m-j^*-2)} & c_{j^*} &= h_{i^* + (m-1)}
\end{align*}
\item If any
$\texttt{query} = \signaturequery \in T_{\texttt{ch}}$, then output a distinct failure symbol and terminate. 
% Not removing dst because subscript seemed important
\item For $j \in [m]$, $T_{\texttt{ch}}\signaturequery := c_{j+1}$.
\item Set $\sig = (c_1, s_1, \ldots, s_m, \mathfrak{T}, \mathfrak{D}_1, \ldots, \mathfrak{D}_{d-1})$.
\item Respond with $\sig$.
\end{enumerate}
This query requires one simulated query to $H_{\lt}^*$, one query to $H_{\lt, k}$ for each $k \in [d-1]$, and one query to $H_{\texttt{base}}$ for each ring member. This query requires a lookup in $T_{\texttt{key}}$, a lookup in $T_{\texttt{key}}$, a lookup in $T_{\texttt{base}}$, and $m$ lookups in $\underline{h}$. This query takes time $t_{\texttt{\sign}} \approx t_{\lt, y} + (d-1)t_{\lt,k} + t_{\texttt{base}} + (2+m)O(1)$. The only way this algorithm fails is if $\mathcal{A}_1$ makes a poorly-formed query. By assumption, $\mathcal{A}_1$ prefers to output a distinct failure symbol than do so. That is, this simulation of $\signingOracle$ succeeds with certainty. 
\end{enumerate}


\item If $\mathcal{A}_1$ outputs a distinct failure symbol and terminates, then $\mathcal{A}_2$ outputs a distinct failure symbol and terminates.
\item Otherwise, $\texttt{out}_{1} \leftarrow \mathcal{A}_1$. In this event, $\mathcal{A}_2$ parses $(\msg, \ring, \sig) \leftarrow \texttt{out}_1$, sets $m = \left|\ring\right|$, and then does the following.
\begin{enumerate}
\item Find all queries $\texttt{query} \in T_{\texttt{ch}}$, $\ell \in [\kappa_{\texttt{ch}}]$, $j \in [m]$, $c\in\Zq$ such that $T_{\texttt{ch}}[\texttt{query}]=(\ell, c)$, $\tvk_j \in \ring$, and $c=c_{j+1}$ is used to verify $\sig$; call this set of queries $S$.
\item \label{step:whereA2fails} If $S = \emptyset$, output a distinct failure symbol and terminate.
\item Otherwise, find the argument $\texttt{query} \in S$ which minimizes $\ell \in T_{\texttt{ch}}[\texttt{query}]$; i.e.\ the index of the first $H_{\texttt{ch}}$ query used during verification.
\item Set $\texttt{tables}$ to consist of all the tables $\left\{T_{\texttt{label}}\right\}$.
\item Set $\texttt{out}_2 = (\ell, \texttt{out}_1, \texttt{tables})$ for this minimal $\ell$.
\end{enumerate}
\item Output $\texttt{out}_2$.
\end{enumerate}

The random oracles $H_{\texttt{seed}}$, $H_{\texttt{FROST}, i}$, $H_{\lt, k}$, and $H_{\texttt{base}}$ are all clearly simulated correctly, in the standard way which is indistinguishable from random oracles.  Moreover, $H_{\texttt{ch}}$ is also simulated correctly, up to the quality of the randomness used for the input $\underline{h}$. $\corruptionOracle$ is also correct, as the simulator knows the table $T_{\texttt{DL}}$ and has access to $\corruptionOracle^*$. 
Certainly $\keyOracle$ is simulated correctly, as the computation of the $Y_i$ and $Y$ points correctly simulates FROST key generation, and the remainder of the response is computed honestly.
Now, consider $\signingOracle$. By construction, the output of $\signingOracle$ is a ring signature which passes verification. Moreover, since $H_{\lt,k}$ and $H_{\texttt{base}}$ are simulated correctly, this simulation of $\signingOracle$ is correct, also, at least up to randomness used to sample $\underline{h}$.
Thus, $\mathcal{A}_2$ is a correct simulation of the unforgeability challenger for $\mathcal{A}_1$.

Now, consider the runtime of $\mathcal{A}_2$. If $t_{\texttt{label}}$ denotes the time it takes to simulate each query made to $H_{\texttt{label}}$ or $\mathcal{O}_{\texttt{label}}$, and $\kappa_{\texttt{label}}$ is the number of such queries, then $\mathcal{A}_2$ takes up to $\kappa_{\texttt{label}}t_{\texttt{label}}$ time for all these queries to $H_{\texttt{label}}$. So, $\mathcal{A}_2$ takes time
\begin{align*}
t_2 &\approx  t_1 + n_{\texttt{key}}\kappa_{\texttt{key}}t_{\texttt{key}} + \kappa_{\sign}t_{\sign} +  \kappa_{\texttt{base}}t_{\texttt{base}} + \sum_{\texttt{label} \notin \left\{\texttt{key}, \sign, \texttt{base}\right\} }\kappa_{\texttt{label}} t_{\texttt{label}} \\
&\approx  t_1 + n_{\texttt{key}}\kappa_{\texttt{key}}t_{\texttt{key}} + \kappa_{\sign}t_{\sign} +  \kappa_{\texttt{base}}t_{\texttt{base}}
\end{align*}where the times $t_{\texttt{label}}$ in the sum are assumed to be negligible, and where $\mathcal{A}_1$ makes some $\kappa_{\texttt{key}}$ queries to $\keyOracle$, and each of these is handled with up to $n_{\texttt{key}}$ queries to $\keyOracle^*$. Although we assume querying $\keyOracle^*$ takes negligible time, the wrapper used to simulate responses from $\keyOracle$ requires sampling randomness and assembling the response. 

Now consider the success probability. Certainly $\mathcal{A}_2$ succeeds at simulating all oracle queries, so $\mathcal{A}_2$ can only terminate with a distinct failure symbol if $\mathcal{A}_1$ fails, or if no index $\ell$ exists as described above.
However, we claim that if if $\mathcal{A}_1$ succeeds, then $T_{\texttt{ch}}$ contains a suitable pair $(\ell,  c)$ except with negligible probability. Indeed, if $\mathcal{A}_1$ outputs a successful forgery with ring signature $\sig = (c_1, \underline{s}, \tlt)$ which passes verification, then $\mathcal{A}_1$ selected this signature to satisfy the verification equations

% Not removing dst because these all are part of challenges
% \begin{align*}
% c_2 =& T_{\texttt{ch}}\signaturequery \\
% c_3 =& T_{\texttt{ch}}(\dst_2 \mid \mid \ring \mid \mid \lt \mid \mid s_2G + c_2 W_2 \mid \mid s_2 \widehat{Y}_2 + c_2 \mathfrak{W} \mid \mid \msg) \\
% \vdots &  \\
% c_1 =& T_{\texttt{ch}}(\dst_m \mid \mid \ring \mid \mid \lt \mid \mid s_mG + c_m W_m \mid \mid s_m \widehat{Y}_m + c_m \mathfrak{W} \mid \mid \msg)
% \end{align*}
For this forgery to pass verification, these $c_j$ must be consistent with responses from the random oracle $H_{\texttt{ch}}$ when queried by a verifier. So, $\mathcal{A}_1$ guessed or queried $H_{\texttt{ch}}$ for these $c_j$.
Guessing one $\Zq$ output of $H_{\texttt{ch}}$ successfully from amongst some $\kappa \in \N$ queries succeeds with probability at most $\prod_{i \in [\kappa]} (q-i)^{-1}$, which is negligible in $q$.  Since $q \in \polysecpar$, this probability is negligible in $\secpar$. Thus, the probability that no index $\ell$ can be found is negligible, and $\epsilon_2 \approx \epsilon_1$. 






\textbf{Fork this simulator.} 
Note that $\mathcal{A}_2$ is compatible with \cref{def:general_forking_algorithm}, leading to the forking algorithm.
Define $\mathcal{A}_3$ to be similar to $\texttt{Fork}_{\mathcal{A}_2}$ as in 
\cref{def:general_forking_algorithm}, except with oracle access to $\keyOracle^*$ and $\corruptionOracle^*$ from \cref{def:omdl}. Then \cref{lem:general_forking_lemma} implies $\mathcal{A}_3$ is a PPT $(t_3, \epsilon_3)$-algorithm where
\begin{align*}
t_3  &= 2t_2+t_2^\prime \approx 2t_2 \in \negl \\
 \epsilon_3 &= \epsilon_2\left(\frac{\epsilon_2}{\kappa_{\texttt{ch}}} - \frac{1}{q-1}\right) \in O\left(\frac{\epsilon_2^2}{\kappa_{\texttt{ch}}}\right)
\end{align*} where $t_2^\prime \in \negl$ is the additional time it takes to sample randomness in \cref{def:general_forking_algorithm}. 

\textbf{Solve OMDL.} Lastly, we build an algorithm $\mathcal{A}_4$ which has access to $\keyOracle^*$ and $\corruptionOracle^*$ as defined in \cref{def:omdl} (and which we assume take negligible time to invoke), runs $\mathcal{A}_3$ as a subroutine, and plays the $\kappa$-OMDL game in time at most $t_4$ and succeeds with probability at least $\epsilon_4$ as follows, where $\kappa =n_{\texttt{key}}\kappa_{\texttt{key}} - 1$.

Observe that $\mathcal{A}_2$ makes some $\texttt{query}$ at the fork point. Moreover, this fork point is selected so that the response appears in verification. Thus, $\texttt{query} = \signaturequery$ for some $j \in [m]$ and for some points $L_j, R_j \in \G$, and $T_{\texttt{ch}}[\texttt{query}] = c_{j+1}$. All data available to $\mathcal{A}_2$, as well as its random tape, are identical until the fork point. Thus, the queries are identical on both sides of the fork with certainty, but by the definition of \cref{def:general_forking_algorithm}, the responses vary in both transcripts except with negligible probability. In particular, the points $L_j$ and $R_j$ are common to both transcripts with certainty.  That is to say, we see the same query with two responses appear as in the following diagram.
\[
\begin{tikzcd}[row sep=large, column sep=0pt] % Adjust spacing as needed
    H_{\texttt{ch}}\signaturequery \arrow[d, shift left, shift left, shift left] \arrow[d, shift right, shift right, shift right] \\
    c_{j+1} \neq c^\prime_{j+1}
\end{tikzcd}
\]
That is to say, on one side of the fork, the same query yields the response $c_{j+1}$, and on the other side of the fork, the different response $c_{j+1}\neq c_{j+1}^\prime$.
Since $\lt = (\mathfrak{T}, \mathfrak{D}_1, \ldots, \mathfrak{D}_{d-1})$ appears in this query, these points are certainly identical on both sides of the fork. 
\begin{enumerate}
\item Run $\mathcal{A}_3$ as a subroutine, responding to $\keyOracle^*$ and $\corruptionOracle^*$ queries by consulting these oracles and responding faithfully.  If $\mathcal{A}_3$ fails, then $\mathcal{A}_4$ outputs a distinct failure symbol and terminates. This step takes time $t_3$ and succeeds (terminates without a failure symbol) with probability $\epsilon_3$.

\item Otherwise, $\texttt{out}_{\mathcal{A}_3} \leftarrow \mathcal{A}_3$. If $\mathcal{A}_4$ cannot parse $(\ell, (\texttt{out}_{\mathcal{A}_1}, \texttt{tables}), (\texttt{out}_{\mathcal{A}_1}^\prime, \texttt{tables}^\prime)) := \texttt{out}_{\mathcal{A}_3}$, then $\mathcal{A}_4$ outputs a distinct failure symbol and terminates. This takes negligible time. By construction of $\mathcal{A}_4$, this step, conditioned on the success of the previous step, certainly succeeds.

\item Otherwise, $\mathcal{A}_4$ attempts to parse the following.
\begin{align*}
(\msg, \ring, \sig) &:= \texttt{out}_{\mathcal{A}_1} & (\msg^\prime, \ring^\prime, \sig^\prime) &:= \texttt{out}_{\mathcal{A}_1}^\prime \\
(\tvk_1, \ldots, \tvk_m) &:= \ring, &
(\tvk_1^\prime, \ldots, \tvk_{m^\prime}^\prime) &:= \ring^\prime, \\
(c_1, \underline{s}, \tlt) &:= \sig &
(c_1^\prime, \underline{s}^\prime, \tlt^\prime) &:= \sig^\prime \\
(\mathfrak{T}, \underline{\mathfrak{D}}) &:= \tlt & (\mathfrak{T}^\prime, \underline{\mathfrak{D}}^\prime) &:= \tlt 
\end{align*} If $\mathcal{A}_4$ cannot parse this, then $\mathcal{A}_4$ outputs a distinct failure symbol and terminates. By construction of $\mathcal{A}_3$, this step, conditioned on the success of the previous step, certainly succeeds.

\item For each $j \in [m]$, $\mathcal{A}_4$ searches for each $\texttt{query}_j$ such that $T_{\texttt{ch}}[\texttt{query}_j]=(\ell_j, c_{j+1})$. This $\ell_j$ is the $H_{\texttt{ch}}$ query index whose response is $c_{j+1}$ used during verification. For the signature challenge $c_{j+1}$ used in verification, call this set $S_1$. Recalling that the forger makes all these queries to $H_{\texttt{ch}}$ in every successful transcript except with negligible probability, then conditioned on the success of the previous steps, this step succeeds except with negligible probability. Moreover, this step takes time at most $O(\kappa_{\texttt{ch}})$ in case we must touch every entry in $T_{\texttt{ch}}$.

\item Find the $j^* \in [m]$ such that $\texttt{query}_{j^*} \in S$ minimizes $\ell_{j^*} \in T_{\texttt{ch}}[\texttt{query}_{j^*}]$. This way, $c_{j^*}$ is the first oracle response used during verification.

\item $\mathcal{A}_4$ parses $\signaturequerystar:= \texttt{query}_{j^*}$ for some points $L_{j^*}, R_{j^*} \in \G$. If $\mathcal{A}_4$ cannot do this, then $\mathcal{A}_4$ outputs a distinct failure symbol and terminates. This takes negligible time. Say this step succeeds with probability $\epsilon^\prime$, and see below.

\item For this $j^*$, parse $(Y_{j^*}, Z_{j^*, 1}, \ldots, Z_{j^*, d-1}) := \tvk_{j^*}$ (where $\tvk_{j^*} \in \ring$). Retrieve $\mu_y \leftarrow T_{\lt}^*[\ring \mid \mid \tlt]$ and, for each $k \in [d-1]$, $\mu_k \leftarrow T_{\lt, k}[\ring \mid \mid \tlt]$.

\item $\mathcal{A}_4$ computes $\widetilde{w}_{j^*} = \frac{s_{j^*} - s_{j^*}^\prime}{c_{j^*}^\prime - c_{j^*}}$. This takes the time of two negations, two additions, an inversion, and a multiplication in $\Zq$, say $2t_{\texttt{neg}} + 2t_{+} + t_{\texttt{inv}} + t_{\texttt{mul}}$. By the construction of $\mathcal{A}_3$, and conditioned on the previous step's success, $c_{j^*}^\prime \neq c_{j^*}$ with certainty, so this step succeeds with certainty.

\item $\mathcal{A}_4$ finds a response $(\tvk, \VK) \leftarrow \keyOracle(n,r)$ to a query made by $\mathcal{A}_1$ such that $\tvk = \tvk_{j^*}$, and parses $(\vk_1, \ldots, \vk_n) := \VK$. This takes negligible time, and, conditioned on the success of the previous step, this step certainly succeeds.


\item If $\tvk_{j^*} \in \mathcal{L}^{\texttt{tot}}_{\texttt{corrupt}}$, output a distinct failure symbol and terminate. Otherwise, for this $\VK$, let $C = \VK \cap \mathcal{L}_{\texttt{corrupt}}$ be the set of corrupted public verification key shares corresponding to the total verification key ring member $\tvk_j$. This step takes negligible time. Moreover, conditioned on the success of the previous step, this step succeeds with certainty. Indeed, $\mathcal{A}_1$ would have failed (causing a cascade of failures) if too many key shares had been corrupted.


\item Otherwise, $\left|C\right| < r-1$. In this event, $\mathcal{A}_4$ selects $r-1-\left|C\right|$ uncorrupted keys associated with $\tvk_{j^*}$ to parse, say $(Y_i, Z_1, \ldots, Z_{d-1}) := \vk_i$, and corrupts them by querying $y_i \leftarrow \corruptionOracle^*(Y_i)$, until $\left|C\right| = r-1$ exactly. This step takes negligible time and certainly succeeds. 

\item Compute the indices $I \subseteq [n]$ of keys in $\VK$ corresponding to the elements in $C$, and compute the Lagrange interpolation coefficients $\lambda_i$ for $I \in [n]$.

\item $\mathcal{A}_4$ picks an uncorrupted challenge key of $\VK$, say $\vk_{i^*}$, and parses $(y_{i^*}, z_1, \ldots, z_{d-1}) :=  \vk_{i^*}$. This is the target challenge key. This step takes negligible time and succeeds with certainty.

\item $\mathcal{A}_4$ computes the Lagrange interpolation coefficients corresponding to the target challenge key $\vk_{i^*}$ and the corrupted keys $(Y_i, Z_1, \ldots, Z_{d-1})$ for each $i \in I$. Call these $\lambda_{i^*}$ for $\vk_{i^*}$, and $\lambda_i$ for each $\vk_i$ with $i \in I$. This takes $r(r-1)$ multiplications and certainly succeeds. Moreover, the Lagrange interpolation coefficients are all nonzero with certainty.


\item $\mathcal{A}_4$ computes the following.
\begin{align*}
z^* &= \sum_{k \in [d-1]} \mu_k z_k \\
\overline{y} &= \sum_{i \in I, i \neq i^*} \lambda_i y_i \\
y_{j^*,i^*} &= \lambda_{i^*}^{-1}\left(\mu_y^{-1}\left(\widetilde{w}_{j^*} - z^*\right) - \overline{y}\right)
\end{align*} Then $\mathcal{A}_4$ sets $\texttt{out}_4 = \mathcal{L}_{\texttt{corrupt}} \cup \left\{y_{j^*, i^*}\right\}$.
Computing $z^*$ takes time $(d-1)t_{\texttt{mul}} + (d-1)t_{+}$, computing $\overline{y}$ takes time $(r-1)t_{\texttt{mul}} + (r-1)t_{+}$, and computing $y_{j^*,i^*}$ from these takes time $2t_{\texttt{mul}} + 2t_{+}$. These all certainly succeed. Thus, this step takes $(r+d)(t_{\texttt{mul}} + t_+)$ time.
\end{enumerate}

Consider the correctness of the composition of all these algorithms $\mathcal{A}_4$ through $\mathcal{A}_1$ as a $\kappa$-OMDL player. Given the auxiliary keys $z_1, \ldots, z_{d-1}$ and the aggregation coefficients, the aggregated key $w$ yields the linking key $y=\mu_Y^{-1}(w-\sum_k \mu_k z_k)$, and given the $r-1$ corrupted secret signing key shares $y_i$ and the value $y$, Lagrange interpolation implies $y = \lambda_{i^*} y_{j^*,i^*} + \sum_{i \in I, i \neq i^*} \lambda_i y_{i}$. Moreover, since this is a successful forgery, all ring members are challenge keys. Solving these for $y_{j^*,i^*}$ provides correctness.

Now consider the runtime. $\mathcal{A}_4$ takes time \[t_4 \approx t_3 + (r+2)t_{+} + 3t_{\texttt{inv}} + 2t_{\texttt{scmul}} + O(\kappa_{\texttt{ch}}) + 2t_{\texttt{neg}} + r^2t_{\texttt{mul}} + (r-1-\left|C\right|)t_{\texttt{corrupt}}\] obtained by summing the times of each step above. However, $r$ is selected by $\mathcal{A}_4$ in the course of executing $\mathcal{A}_1$. Moreover, the strongest adversary corrupts no keys at all. So, we use $r \leq n \leq n_{\texttt{ch}}$ and $r-1-\left|C\right| \leq n_{\texttt{ch}}-1$ to obtain \[t_4 \approx t_3 + (n_{\texttt{ch}}+2)t_{+} + 3t_{\texttt{inv}} + 2t_{\texttt{scmul}} + O(\kappa_{\texttt{ch}}) + 2t_{\texttt{neg}} + n_{\texttt{ch}}^2t_{\texttt{mul}} + (n_{\texttt{ch}}-1)t_{\texttt{corrupt}}.\] 

However, $t_3 \approx 2t_2 \approx 2(t_1 + n_{\texttt{key}}\kappa_{\texttt{key}}t_{\texttt{key}} + \kappa_{\sign}t_{\sign} + \kappa_{\texttt{corrupt}}t_{\texttt{corrupt}} + \kappa_{\texttt{base}} t_{\texttt{base}})$, so $t_4 \approx 2t_1 + \widetilde{t}$ where
\begin{align*}
\widetilde{t} =& 2n_{\texttt{key}}\kappa_{\texttt{key}}t_{\texttt{key}} + 2\kappa_\sign t_\sign + 2\kappa_{\texttt{corrupt}}t_{\texttt{corrupt}} + 2\kappa_{\texttt{base}}t_{\texttt{base}} + \\
& (n_{\texttt{ch}}+2)t_{+} + 3t_{\texttt{inv}} + 2t_{\texttt{scmul}} + O(\kappa_{\texttt{ch}}) + 2t_{\texttt{neg}} + n_{\texttt{ch}}^2t_{\texttt{mul}} + (n_{\texttt{ch}}-1)t_{\texttt{corrupt}}
\end{align*}

Now consider success probability. If $\mathcal{A}_3$ succeeds, $\mathcal{A}_4$ can fail if the $\texttt{query}$ made to $H_{\texttt{ch}}$ such that $T_{\texttt{ch}}[\texttt{query}]=(\ell, c)$ cannot be parsed as $\signaturequery$. We denoted above the probability that $\texttt{query}$ cannot be parsed appropriately with $\epsilon^\prime$.

All that remains is to show $\epsilon^\prime$ is negligible. Indeed, the $\texttt{query}$ satisfies $T_{\texttt{ch}}[\texttt{query}]=(\ell, c)$ where $c=c_{j+1}$ is used in verification.  If $\texttt{query}$ cannot be parsed as $\signaturequery$, but $c_{j+1}$ is used in verification, then there is some other $\texttt{query}^\prime \in T_{\texttt{ch}}$ with this $c$. This is a second pre-image for $\texttt{query}$ for the random oracle $H_{\texttt{ch}}$, so this $\epsilon^\prime \in \negl$.
% On the other hand, consider $\epsilon^{\prime \prime}$. This probability describes the event that $Y_j \in \tvk_j \in \ring$ is not a challenge key from $\keyOracle^*$. This occurs if and only if $\tvk_j \in \ring$ is not a challenge key in the transcript for $\keyOracle$. Recall that $\mathcal{A}_2$ fails in \cref{step:whereA2fails} in this case, so $\epsilon^{\prime \prime} = 0$ conditioned on the event that $\mathcal{A}_2$ succeeds.
\end{proof}



\subsection{Linkability}


\begin{theorem}
FROSTLASS is linkable under \cref{def:linkable}. 
\end{theorem}
\begin{proof}
Let $\kappa \in \mathbb{N}$ and let $\mathcal{A}$ be a $\kappa$-linkability breaker. 

We build a tower of algorithms, $\mathcal{A}_3 \to \mathcal{A}_2 \to \mathcal{A}_1$, where $\mathcal{A}_1 = \mathcal{A}$, similar to those in \cref{thm:suf}, to extract the discrete logarithm of an aggregated ring member which is under adversarial control. We use this discrete logarithm to show the linkability tag is, except with negligible probability, the image of the linking key under a collision-resistant function.

First, recall $\mathcal{A}_2$ in \cref{thm:suf} was a five-step algorithm. We let $\mathcal{A}_2$ operate similarly to $\mathcal{A}_2$ from \cref{thm:suf}, modifying the following steps:
\begin{enumerate}
\setcounter{enumi}{3}
% \item As in \cref{thm:suf}.
% \item As in \cref{thm:suf}.
% \item As in \cref{thm:suf}.
\item Otherwise, $\texttt{out}_1 \leftarrow \mathcal{A}_1$. In this event, $\mathcal{A}_2$ parses the first message-ring-signature triple present, $(\msg, \ring, \sig) \leftarrow \texttt{out}_1$, sets $m = \left|\ring\right|$, and does the following.
\begin{enumerate}
\item Find all queries $\texttt{query} \in T_{\texttt{ch}}$, $\ell \in [\kappa_{\texttt{ch}}]$, $j \in [m]$, $c \in \Zq$ such that $T_{\texttt{ch}}[\texttt{query}] = (\ell, c)$, $\tvk_j \in \ring$ is not an uncorrupted challenge key, and $c=c_{j+1}$ is used to verify $\sig$; call this set of queries $S$.
% \item As in \cref{thm:suf}.
% \item As in \cref{thm:suf}.
% \item As in \cref{thm:suf}.
% \item As in \cref{thm:suf}.
\end{enumerate}
% \item As in \cref{thm:suf}.
\end{enumerate}
All other steps are otherwise similar to \cref{thm:suf}. That we fork only on queries related to uncorrupted challenge keys is critically important here, just as forking only on challenge keys was critical in \cref{thm:suf}. As before, $\mathcal{A}_2$ is compatible with the forking algorithm, so let $\mathcal{A}_3 = \texttt{Fork}_{\mathcal{A}_2}$ as in \cref{thm:suf}. Now consider a successful output of $\mathcal{A}_3$. 

The forking query  $\signaturequerystar$ is the same on both sides of the fork. Moreover, in one transcript we obtain $L_{j^*} = s_{j^*} G + c_{j^*} W_{j^*}$ where $W_{j^*}$ is the aggregation of the ring member with index $j^*$, and in the other transcript the same $L_{j^*}$ satisfies $L_{j^*} = s_{j^*}^\prime G + c_{j^*}^\prime W_{j^*}$. Thus, just as before, we obtain the discrete logarithm $W_{j^*} = \frac{s_{j^*} - s_{j^*}^\prime}{c_{j^*}^\prime - c_{j^*}}G$. However, this $W_{j^*} = \mu_Y Y_{j^*} + \sum_k \mu_k Z_{j^*,k}$. Thus, if $y_{j^*}$ is the discrete logarithm of $Y_{j^*}$ with respect to $G$ and each $z_{j^*,k}$ is the discrete logarithm of $Z_{j^*,k}$ with respect to $G$, then $\frac{s_{j^*} - s_{j^*}^\prime}{c_{j^*}^\prime - c_{j^*}} = \mu_Y y_{j^*} + \sum_k \mu_k z_{j^*,k}$. Note this equation holds for these scalars, even though we do not (and perhaps even $\mathcal{A}$ may not) know them.

Also, we obtain in one transcript $R_{j^*} = s_{j^*} \widehat{Y}_{j^*} + c_{j^*} \mathfrak{W}$ where $\mathfrak{W}$ is the aggregation of the linking tag base points $\mathfrak{T}$ and $\mathfrak{D}_k$. In the other transcript, the same $R_{j^*}$ satisfies $R_{j^*} = s_{j^*}^\prime \widehat{Y}_{j^*} + c_{j^*}^\prime \mathfrak{W}$. Thus, just as before, we obtain the discrete logarithm $\mathfrak{W} = \frac{s_{j^*} - s_{j^*}^\prime}{c_{j^*}^\prime - c_{j^*}}\widehat{Y}_{j^*}$. However, $\mathfrak{W} = \mu_Y \mathfrak{T} + \sum_k \mu_k \mathfrak{D}_k$, so $\frac{s_{j^*} - s_{j^*}^\prime}{c_{j^*}^\prime - c_{j^*}}\widehat{Y}_{j^*} = \mu_Y \mathfrak{T} + \sum_k \mu_k \mathfrak{D}_k$.
% But, as before, $\frac{s_{j^*} - s_{j^*}^\prime}{c_{j^*}^\prime - c_{j^*}} = \mu_Y y_{j^*} + \sum_k \mu_k z_{j^*,k}$, so $(\mu_Y y_{j^*} + \sum_k \mu_k z_{j^*,k})\widehat{Y}_{j^*} = \mathfrak{W} = \mu_Y \mathfrak{T} + \sum_k \mu_k \mathfrak{D}_k$. 
Re-arranging, we have $\mu_Y\left(y_{j^*} \widehat{Y}_{j^*} - \mathfrak{T}\right) + \sum_k \mu_k\left(z_{j^*,k} \widehat{Y}_{j^*} - \mathfrak{D}_k\right) = 0$, which can only be satisfied if $\mathfrak{T} = \widehat{Y}_{j^*}$, except with negligible probability. 
Since $\mathfrak{T}$ is a collision-resistant function of $y_{j^*}$, we conclude that valid signatures, except with negligible probability, have linking tags which are collision-resistant functions of some ring member which is not an uncorrupted challenge key. This prevents more linking tags than linking keys.
\end{proof}




% Partitions the Rust Implementation into a separate .tex file
% % \section{Rust Implementation}

% \subsection{Repository Structure}







% \subsection{Explanation of Code}


% \begin{table}[ht]
% \begin{center}
%     \begin{tabular}{C{1.5cm}|p{4.5cm}|p{4.5cm}}
%     \toprule
%             \multicolumn{1}{c|}{\textbf{Symbol}} & \textbf{CLSAG} & \textbf{FROST} \\ 
%     \midrule
% $\mathtt{Msg}$ & Message to be signed blah blah blah blah blah overflow blah & junk \\[2pt]
% $\mathtt{sk}$ & binary & junk \\[2pt]
% \bottomrule
% \end{tabular}
% \end{center}   
% \label{table:notation} 
% \caption{Notation employed for cryptographic components.}
% \end{table}








% Partitions Rigo's Code Corner into a separate .tex file
% \section{Rust Implementation}

In this section, we review the Rust implementation of FROSTLASS \path{/serai-dex/serai/networks/monero/} directory (the ``audit directory'') in commit \texttt{48db06f} by \texttt{KayabaNerve} of the GitHub repository at \url{github.com/serai-dex/serai}

\subsection{Overview}

The audit directory contains a Rust implementation of Monero wallet functionality, together with a new approach to using FROST for threshold signing. All files, folders, and subfolders of 
\path{/serai-dex/serai/networks/monero/} are in scope for the audit, except not the subfolder \path{/serai-dex/serai/networks/monero/verify-chain/}.

\subsection{General Findings}

This codebase is consistent, neat, clean, and efficient.

\subsection{Repository Structure}

The repository at \url{github.com/serai-dex/serai} is organized into several top-level directories, including \path{/serai-dex/serai/networks/}, which in turn contains \path{/serai-dex/serai/networks/monero/}.  Figure \ref{fig:directory_structure} describes the target directory structure, where asterisks indicate crates (named in parentheses); the only folder not in scope of this audit is \texttt{verify-chain}. 


\begin{figure}[ht]
\centering
\begin{minipage}{0.9\linewidth}
\begin{mdframed}
\begin{verbatim}
/serai-dex/serai/
|-- out_of_scope_files
|-- networks/
    |-- out_of_scope_files
    |-- monero*/ (monero-serai)
        |-- generators*/ (monero-generators)
        |-- io*/ (monero-io)
        |-- primitives*/ (monero-primitives)
        |-- ringct/
            |-- borromean*/ (monero-borromean)
            |-- bulletproofs*/ (monero-bulletproofs)
            |-- clsag*/ (monero-clsag)
            |-- mlsag*/ (monero-mlsag)
        |-- rpc*/ (monero-rpc)
            |-- simple-request*/ (monero-simple-request-rpc)
        |-- src/
        |-- tests/
        |-- verify-chain*/
        |-- wallet*/ (monero-wallet)
            |-- address*/ (monero-address)
\end{verbatim}
\end{mdframed}
\end{minipage}
\caption{Directory structure of the target repository. Everything within \protect\path{/serai-dex/serai/networks/monero} is in scope, except the subfolder \protect\path{/serai-dex/serai/networks/monero/verify-chain/}. Asterisks indicate crates. Crate names are included in parantheses.}
\label{fig:directory_structure}
\end{figure}

% \subsection{Crate Dependencies}

In the following, we omit \texttt{/serai-dex/serai/networks/monero/} from directory references for clarity, with the understanding that all directory references use this as a prefix.

The crates are evocatively named by the data they handle, and mutually depend on each other. Crate dependency is not isomorphic to the file structure. 
Figure \ref{fig:transitive_reduced_dependency_graph} displays these dependencies in the transitive reduction of the directed graph of crate dependencies. Readers should be aware that the transitive reduction of a directed graph $G$ is constructed by removing as many edges as possible without changing the \textit{reachability relation} on pairs of vertices. Thus, not all edges corresponding to direct crate dependency are displayed in Figure \ref{fig:transitive_reduced_dependency_graph}. For example, \texttt{monero-serai} depends directly on \texttt{monero-generators}, \texttt{monero-io}, and \texttt{monero-primitives}.
In Section \ref{sec:crate_summaries}, we present a look at each crate. 

\begin{figure}[ht]
\centering
\begin{tikzpicture}[
    scale=1, % Ensure proper scaling
    every node/.style={draw, rectangle, rounded corners, minimum height=1cm, font=\ttfamily, inner sep=5pt, align=center}, % Rectangular nodes
    edge/.style={-{Latex}, thick}
]

% Manually normalized coordinates for a 10x7 aspect ratio
\node (wallet) at (4, 8.5) {wallet};
\node (simple-request) at (8, 8.5) {simple-request};
\node (rpc) at (8.25, 6.5) {rpc};
\node (serai) at (2.5, 5.5) {serai};
\node (address) at (10, 3.5) {address};
\node (mlsag) at (8, 3.5) {mlsag};
\node (clsag) at (6, 3.5) {clsag};
\node (bulletproofs) at (3.5, 3.5) {bulletproofs};
\node (borromean) at (0.5, 3.5) {borromean};
\node (io) at (5, 1) {io};
\node (generators) at (2, 1) {generators};
\node (primitives) at (8, 1) {primitives};

% Edges
\draw[edge] (wallet) -- (rpc);
\draw[edge] (wallet) -- (address);
\draw[edge] (wallet) -- (clsag);
\draw[edge] (wallet) -- (serai);
\draw[edge] (simple-request) -- (rpc);
\draw[edge] (rpc) -- (serai);
\draw[edge] (rpc) -- (address);
\draw[edge] (serai) -- (borromean);
\draw[edge] (serai) -- (bulletproofs);
\draw[edge] (serai) -- (clsag);
\draw[edge] (serai) -- (mlsag);
\draw[edge] (address) -- (io);
\draw[edge] (address) -- (primitives);
\draw[edge] (borromean) -- (io);
\draw[edge] (borromean) -- (generators);
\draw[edge] (borromean) -- (primitives);
\draw[edge] (bulletproofs) -- (io);
\draw[edge] (bulletproofs) -- (generators);
\draw[edge] (bulletproofs) -- (primitives);
\draw[edge] (clsag) -- (io);
\draw[edge] (clsag) -- (generators);
\draw[edge] (clsag) -- (primitives);
\draw[edge] (mlsag) -- (io);
\draw[edge] (mlsag) -- (generators);
\draw[edge] (mlsag) -- (primitives);

\end{tikzpicture}
\caption{The transitive reduction of the graph of crates and their dependencies. Not all edges corresponding to a direct dependency are displayed in a transitive reduction. For example, \texttt{monero-serai} depends directly on \texttt{monero-generators}, \texttt{monero-io}, and \texttt{monero-primitives}, but these edges are not explicitly displayed.}
\label{fig:transitive_reduced_dependency_graph}
\end{figure}


\subsection{Functionality}

\subsubsection{Crate Details}\label{sec:crate_summaries}

In this section, we describe each crate name, version, purpose, internal dependencies, and a brief description of crate structure. We use a breadth-first, top-down approach following Figure \ref{fig:transitive_reduced_dependency_graph}. Elements of public APIs (i.e.\ with the modifier \texttt{pub}) have names which are decorated \texttt{thusly}\textsuperscript{\textdagger}, and elements exposed at the crate level (i.e.\ with the modifier \texttt{pub(crate)}) have names which are decorated \texttt{thusly}\textsuperscript{$\Delta$}. We provide links throughout the document to corresponding glossary entries.

% This is the beginning of the giant itemize section in Josh's documentation
% It's a little tricky because so many pieces are itemized that things seem to deeply nested

\begin{itemize} 

 \item \gls{monero-wallet (v0.1.0)}
 \begin{itemize}
 \item Purpose: Handle all wallet functionality.
 \item Internal Dependencies: 
 \begin{itemize}
 \item \gls{monero-address (v0.1.0)}\textsuperscript{\textdagger}
 \item \gls{monero-clsag (v0.1.0)}\textsuperscript{\textdagger}
 \item \gls{monero-rpc (v0.1.0)}\textsuperscript{\textdagger}, 
 \item \gls{monero-serai (v0.1.4-alpha)}\textsuperscript{\textdagger}
 \end{itemize}
 \item Structure: A standard library crate, with the corresponding entry point at  \gls{wallet-entry-point}. 
 
 \item Tests at \gls{wallet-tests}.
 \item The \gls{monero-wallet (v0.1.0)} crate employs the following modules.
\begin{itemize}
\item \gls{decoys-module} handles decoy selection with a publicly exposed struct \texttt{OutputWithDecoys}.
\item \gls{extra-module}\textsuperscript{\textdagger} handles the \texttt{extra} field of a transaction.
\item \gls{output-module}\textsuperscript{$\Delta$} handles transaction outputs.
\item \gls{scan-module} handles transaction scanning.
\item \gls{send-module}\textsuperscript{\textdagger} handles sending transactions. This is a directory module, and contains the following file modules:
\begin{itemize}
\item \gls{eventuality-module} handles \gls{eventualities}.
\item \gls{send-multisig-module} handles sending threshold transactions.
\item \gls{send-tx-module} handles sending transactions.
\item \gls{send-tx-keys-module} handles keys for sending transactions.
\end{itemize}
\item \gls{view-pair-module} handles pairs of keys, where one is a public spend key, and the other is a private view key.
\end{itemize}

% Rigo added this
\end{itemize}


 \item \gls{monero-simple-request-rpc (v0.1.0)}
 \begin{itemize}
 \item Purpose: Default RPC to avoid external dependencies on, e.g.\ reqwest. Only used in dev dependencies.
 \item Internal Dependencies: 
 \begin{itemize}
 \item \gls{monero-rpc (v0.1.0)}
 \end{itemize}
 \item Structure: A standard library crate,  with the corresponding entry point at  \gls{monero-simple-request-rpc-entry-point}.
\end{itemize}
 
 
 \item \gls{monero-rpc (v0.1.0)}
 \begin{itemize}
 \item Purpose: handle RPC calls for interacting on the Monero network.
 \item Internal Dependencies: 
 \begin{itemize}
 \item \gls{monero-address (v0.1.0)}  
 \item \gls{monero-serai (v0.1.4-alpha)}
 \end{itemize}
 \item Structure: A standard library crate, with the corresponding entry point at \gls{monero-rpc-entry-point}, employing no file or directory modules.
 \end{itemize}
 
\item \gls{monero-serai (v0.1.4-alpha)} 
\begin{itemize}
\item Purpose: the overall transaction library.
\item Internal dependencies: 
\begin{itemize}
\item \gls{monero-borromean (v0.1.0)}
\item \gls{monero-bulletproofs (v0.1.0)}
\item \gls{monero-clsag (v0.1.0)} 
\item \gls{monero-generators (v0.4.0)}\textsuperscript{\textdagger}
\item \gls{monero-io (v0.1.0)}\textsuperscript{\textdagger}
\item \gls{monero-mlsag (v0.1.0)}
\item \gls{monero-primitives (v0.1.0)}\textsuperscript{\textdagger}
\end{itemize}
\item Structure: A standard library crate, with the corresponding entry point at \gls{monero-serai-entry-point}.
\begin{itemize}
\item File modules:
\begin{itemize}
\item \gls{block-module}\textsuperscript{\textdagger}
\item \gls{merkle-module}
\item \gls{ring-signatures-module}\textsuperscript{\textdagger}
\item \gls{ringct-module}\textsuperscript{\textdagger}
\item \gls{transaction-module}\textsuperscript{\textdagger}
\end{itemize}
\item Tests at \gls{monero-serai-tests}
\end{itemize}

% Rigo added this
\end{itemize}

\item \gls{monero-address (v0.1.0)}
\begin{itemize}
\item Purpose: handles Monero addresses.
\item Internal dependencies: 
\begin{itemize}
\item \gls{monero-io (v0.1.0)} 
\item \gls{monero-primitives (v0.1.0)}
\end{itemize}
\item Structure: A standard library crate, with the corresponding entry point at \gls{monero-address-entry-point}.
\begin{itemize}
\item File module: \gls{base58-module}.
\item Tests at \gls{address-tests}.
\end{itemize}
\end{itemize}



   \item \gls{monero-borromean (v0.1.0)}
   \begin{itemize}
   \item Purpose: Handles Borromean signatures and Borromean range proofs.
   \item Internal dependencies:
   \begin{itemize}
   \item \gls{monero-generators (v0.4.0)}
   \item \gls{monero-io (v0.1.0)}
   \item \gls{monero-primitives (v0.1.0)}
   \end{itemize}
   \item Structure: A standard library crate, with the corresponding entry point at \gls{borromean-entry-point}. Employs no modules, and untested.
   \end{itemize}
   
   \item \gls{monero-bulletproofs (v0.1.0)}
   \begin{itemize}
   \item Purpose: Handles original bulletproofs and bulletproofs plus.
   \item Internal dependencies:
   \begin{itemize}
   \item \gls{monero-generators (v0.4.0)}
   \item \gls{monero-io (v0.1.0)}
   \item \gls{monero-primitives (v0.1.0)}
   \end{itemize}
   \item Structure: A standard library crate, with the corresponding entry point at \gls{bulletproofs-entry-point}.
   \begin{itemize}
   \item File modules:
   \begin{itemize}
   \item \gls{bp-batch-verifier-module}\textsuperscript{$\Delta$}
   \item \gls{bp-core-module}\textsuperscript{$\Delta$}
   \item \gls{bp-point-vector-module}\textsuperscript{$\Delta$}
   \item \gls{bp-scalar-vector-module}\textsuperscript{$\Delta$}
   \end{itemize}
   \item Directory modules:
   \begin{itemize}
   \item \gls{bp-original-module}\textsuperscript{$\Delta$}
   \item \gls{bp-plus-module}\textsuperscript{$\Delta$}
   \end{itemize}
   \item Tests at \gls{bp-test-module}.
   \end{itemize}
   \end{itemize}


   
   \item \gls{monero-clsag (v0.1.0)}
   \begin{itemize}
   \item Purpose: Handles CLSAG ring signatures and a FROST-like thresholdization.
   \item Internal dependencies:
   \begin{itemize}
   \item \gls{monero-generators (v0.4.0)}
   \item \gls{monero-io (v0.1.0)}
   \item \gls{monero-primitives (v0.1.0)}
   \end{itemize}
   \item Structure: A standard library crate, with the corresponding entry point at \gls{monero-clsag-entry-point}.
   \begin{itemize}
   \item File module: \gls{clsag-multisig-module}.
   \item Tests: \gls{clsag-tests}
   \end{itemize}
   \end{itemize}
   
   \item \gls{monero-mlsag (v0.1.0)}
   \begin{itemize}
   \item Purpose: Handles MLSAG ring signatures.
   \item Internal dependencies:
   \begin{itemize}
   \item \gls{monero-generators (v0.4.0)}
   \item \gls{monero-io (v0.1.0)}
   \item \gls{monero-primitives (v0.1.0)}
   \end{itemize}
   \item Structure: A standard library crate with entry point at
   
   \gls{monero-mlsag-entry-point}. Employs no modules, and untested.
   \end{itemize}


   
 \item \gls{monero-primitives (v0.1.0)}
   \begin{itemize}
   \item Purpose: Handles Pedersen Commitments and Decoys.
   \item Internal dependencies:
   \begin{itemize}
   \item  \gls{monero-io (v0.1.0)} 
   \item \gls{monero-generators (v0.4.0)}
   \end{itemize}
   \item Structure: A standard library crate with entry point at \gls{monero-primitives-entry-point}. 
   \begin{itemize}
   
   \item File module at \gls{unreduced-scalar-module}.
   \item Tests at \gls{monero-primitives-tests}.
   \end{itemize}
   \end{itemize}


 
\item \gls{monero-generators (v0.4.0)}
   \begin{itemize}
   \item Purpose: Handles hashing data to elliptic curve group elements and all fixed generators used in Monero protocol computations.
   \item Internally dependent only on \gls{monero-io (v0.1.0)}.
   \item Structure: A standard library crate with entry point at \gls{monero-generators-entry-point}. 
   \begin{itemize}
   
   \item File module at \gls{hash-to-point-module}. 
   \item Tests at \gls{monero-generators-tests}
   \end{itemize}
   \end{itemize}
   
\item \gls{monero-io (v0.1.0)} 
   \begin{itemize}
   \item Purpose: Handles reading and writing various data structures used in Monero protocol computations (e.g.\ bytes, scalars, group elements, lists whose entries are the same type).
   \item No internal dependencies.
   \item Structure: A standard library crate with entry point at \gls{monero-io-entry-point}. Employs neither modules nor tests.   
\end{itemize}

% Rigo added this
\end{itemize}


% -------------------------------------------------- %
% Below is the function section of Josh's documentation
% -------------------------------------------------- %

% -------------------------------------------------- %
\subsection{monero-io (v0.1.0)}
The \gls{monero-io (v0.1.0)} crate implements canonical serialization and deserialization operations.  The following details the supported operations.  

\begin{description}
\item[VarInt Encoding] The variable-length integer encoding employs a continuation bit scheme.  Each byte uses 7 bits for data, with the most significant bit indicating whether additional bytes follow.  % https://github.com/serai-dex/serai/blob/48db06f901952b24bb38d7c7e256f798f08512cd/networks/monero/io/src/lib.rs#L18 and https://github.com/serai-dex/serai/blob/48db06f901952b24bb38d7c7e256f798f08512cd/networks/monero/io/src/lib.rs#L60-L72
The implementation ensures canonical encoding by rejecting unnecessary leading zeros.  % https://github.com/serai-dex/serai/blob/48db06f901952b24bb38d7c7e256f798f08512cd/networks/monero/io/src/lib.rs#L139-L141

\item[Scalar Serialization] Scalars employ a fixed 32-byte encoding in little-endian representation.  % https://github.com/serai-dex/serai/blob/48db06f901952b24bb38d7c7e256f798f08512cd/networks/monero/io/src/lib.rs#L75-L77
The implementation enforces canonical form by requiring fully reduced values modulo the curve order.  % https://github.com/serai-dex/serai/blob/48db06f901952b24bb38d7c7e256f798f08512cd/networks/monero/io/src/lib.rs#L157-L160

\item[Point Serialization] Points use the compressed Edwards format in 32 bytes.  The Y-coordinate occupies bytes 0-30, while byte 31 stores the X-coordinate sign bit.  % https://github.com/serai-dex/serai/blob/48db06f901952b24bb38d7c7e256f798f08512cd/networks/monero/io/src/lib.rs#L80-L82
Validation ensures both canonical encoding and proper subgroup membership when requested.  % https://github.com/serai-dex/serai/blob/48db06f901952b24bb38d7c7e256f798f08512cd/networks/monero/io/src/lib.rs#L171-L176 and https://github.com/serai-dex/serai/blob/48db06f901952b24bb38d7c7e256f798f08512cd/networks/monero/io/src/lib.rs#L188-L193

\item[Integer Types] The implementation supports little-endian encoding of standard widths: % https://github.com/serai-dex/serai/blob/48db06f901952b24bb38d7c7e256f798f08512cd/networks/monero/io/src/lib.rs#L119-L131
\begin{itemize}
\item u16: 2 bytes
\item u32: 4 bytes  
\item u64: 8 bytes
\end{itemize}

\item[Vector Operations] The crate supports two vector serialization formats:
\begin{itemize}
\item Raw vectors with consecutive elements % https://github.com/serai-dex/serai/blob/48db06f901952b24bb38d7c7e256f798f08512cd/networks/monero/io/src/lib.rs#L85-L94
\item Length-prefixed vectors using VarInt length encoding % https://github.com/serai-dex/serai/blob/48db06f901952b24bb38d7c7e256f798f08512cd/networks/monero/io/src/lib.rs#L97-L104
\end{itemize}
\end{description}

The implementation enforces strict canonicalization through validation of:
\begin{itemize}
\item VarInt minimal encoding % https://github.com/serai-dex/serai/blob/48db06f901952b24bb38d7c7e256f798f08512cd/networks/monero/io/src/lib.rs#L139-L141
\item Scalar reduction state % https://github.com/serai-dex/serai/blob/48db06f901952b24bb38d7c7e256f798f08512cd/networks/monero/io/src/lib.rs#L157-L160
\item Point coordinate canonical form % https://github.com/serai-dex/serai/blob/48db06f901952b24bb38d7c7e256f798f08512cd/networks/monero/io/src/lib.rs#L171-L176
\item Optional prime-order subgroup membership % https://github.com/serai-dex/serai/blob/48db06f901952b24bb38d7c7e256f798f08512cd/networks/monero/io/src/lib.rs#L188-L193
\end{itemize}


% -------------------------------------------------- %
\subsection{monero-generators (v0.1.0)}

The \texttt{hash\_to\_point} function implements Monero's \texttt{hash\_to\_ec} operation to deterministically maps 32 bytes to a point on the Ed25519 curve as in:

\begin{description}
\item[Input] \hfill 

\begin{itemize}
\item Takes exactly 32 bytes.  % https://github.com/serai-dex/serai/blob/48db06f901952b24bb38d7c7e256f798f08512cd/networks/monero/generators/src/hash_to_point.rs#L13
\end{itemize}

\item[Processing] \hfill 

\begin{enumerate}
\item Performs Keccak-256 hash of input bytes.  % https://github.com/serai-dex/serai/blob/48db06f901952b24bb38d7c7e256f798f08512cd/networks/monero/generators/src/hash_to_point.rs#L17 
\item Squares the hash result and doubles it to get value \texttt{v}.  % "
\item Computes intermediate values:
  \begin{itemize}
  \item \texttt{w = v + 1}.  % https://github.com/serai-dex/serai/blob/48db06f901952b24bb38d7c7e256f798f08512cd/networks/monero/generators/src/hash_to_point.rs#L18
  \item \texttt{x = w² - A²v} where A = 486662 (curve parameter).  % https://github.com/serai-dex/serai/blob/48db06f901952b24bb38d7c7e256f798f08512cd/networks/monero/generators/src/hash_to_point.rs#L15 and https://github.com/serai-dex/serai/blob/48db06f901952b24bb38d7c7e256f798f08512cd/networks/monero/generators/src/hash_to_point.rs#L19
  \item Calculates a partial X coordinate through a series of field operations.  % https://github.com/serai-dex/serai/blob/48db06f901952b24bb38d7c7e256f798f08512cd/networks/monero/generators/src/hash_to_point.rs#L25-L33
  \end{itemize}
\item Derives Y coordinate through:
  \begin{itemize}
  \item Calculates \texttt{y = w - x}.  % https://github.com/serai-dex/serai/blob/48db06f901952b24bb38d7c7e256f798f08512cd/networks/monero/generators/src/hash_to_point.rs#L36
  \item Sets sign bit based on zero checks.  % https://github.com/serai-dex/serai/blob/48db06f901952b24bb38d7c7e256f798f08512cd/networks/monero/generators/src/hash_to_point.rs#L37-L39
  \item Performs field inversions and multiplications to get final Y value.  % https://github.com/serai-dex/serai/blob/48db06f901952b24bb38d7c7e256f798f08512cd/networks/monero/generators/src/hash_to_point.rs#L41-L48
  \end{itemize}
\item Sets the high bit of byte 31 based on calculated sign.  % https://github.com/serai-dex/serai/blob/48db06f901952b24bb38d7c7e256f798f08512cd/networks/monero/generators/src/hash_to_point.rs#L50
\end{enumerate}

\item[Output] \hfill 

\begin{itemize}
\item Returns an EdwardsPoint in compressed 32-byte format:  % https://github.com/serai-dex/serai/blob/48db06f901952b24bb38d7c7e256f798f08512cd/networks/monero/generators/src/hash_to_point.rs#L52
  \begin{itemize}
  \item Bytes 0-30: Y coordinate.  
  \item Byte 31: Sign bit of X coordinate in MSB.  % https://github.com/serai-dex/serai/blob/48db06f901952b24bb38d7c7e256f798f08512cd/networks/monero/generators/src/hash_to_point.rs#L50
  \end{itemize}
\item Multiplies result by cofactor (8) to ensure point is in the prime-order subgroup.  % https://github.com/serai-dex/serai/blob/48db06f901952b24bb38d7c7e256f798f08512cd/networks/monero/generators/src/hash_to_point.rs#L52 (mul_by_cofactor)
\end{itemize}

\end{description}

The function uses the \texttt{FieldElement} type from dalek-ff-group for constant-time field arithmetic operations over the Ed25519 field of order $2^{255} - 19$.  All operations are performed in constant time as a mitigation against timing-based side-channel attacks.  

\subsubsection{monero-generators Operations}

The \texttt{monero-generators} crate provides generator points for Pedersen commitments and Bulletproofs in Monero as in:

\subsubsection{Core Generator H}

\begin{description}
\item[Generation] \hfill \\
The base generator point H is computed as in:
\begin{enumerate}
\item Take ED25519\_BASEPOINT\_POINT (the standard Ed25519 base point G).  % https://github.com/serai-dex/serai/blob/48db06f901952b24bb38d7c7e256f798f08512cd/networks/monero/generators/src/lib.rs#L10 and https://github.com/serai-dex/serai/blob/48db06f901952b24bb38d7c7e256f798f08512cd/networks/monero/generators/src/lib.rs#L29-L31
\item Compress it to 32 bytes.  % https://github.com/serai-dex/serai/blob/48db06f901952b24bb38d7c7e256f798f08512cd/networks/monero/generators/src/lib.rs#L30
\item Compute Keccak-256 hash of these bytes.  % "
\item Map the hash to a curve point using \texttt{hash\_to\_point}.  % "
\item Multiply by the cofactor (8).  % https://github.com/serai-dex/serai/blob/48db06f901952b24bb38d7c7e256f798f08512cd/networks/monero/generators/src/lib.rs#L32
\end{enumerate}

\item[Powers Table] \hfill \\
Precomputes powers of H for efficient amount commitments as in:
\begin{itemize}
\item Creates array of 64 points $[H, H*2, H*4, ..., H*2^{63}]$. \free{should be exponents?}
% https://github.com/serai-dex/serai/blob/48db06f901952b24bb38d7c7e256f798f08512cd/networks/monero/generators/src/lib.rs#L35-L41
\item Each entry is double the previous entry.  % https://github.com/serai-dex/serai/blob/48db06f901952b24bb38d7c7e256f798f08512cd/networks/monero/generators/src/lib.rs#L38
\item Used for efficient decomposition of u64 amounts.  % https://github.com/serai-dex/serai/blob/48db06f901952b24bb38d7c7e256f798f08512cd/networks/monero/generators/src/lib.rs#L42-L48
\end{itemize}
\end{description}

\subsubsection{Bulletproofs Generator Vector Creation}

\begin{description}
\item[Constants] \hfill \\
\begin{itemize}
\item MAX\_COMMITMENTS: 16 (maximum provable commitments per range proof).  % https://github.com/serai-dex/serai/blob/48db06f901952b24bb38d7c7e256f798f08512cd/networks/monero/generators/src/lib.rs#L51
\item COMMITMENT\_BITS: 64 (bits per commitment value).  % https://github.com/serai-dex/serai/blob/48db06f901952b24bb38d7c7e256f798f08512cd/networks/monero/generators/src/lib.rs#L53
\item MAX\_MN: MAX\_COMMITMENTS * COMMITMENT\_BITS (total bits in proof).  % https://github.com/serai-dex/serai/blob/48db06f901952b24bb38d7c7e256f798f08512cd/networks/monero/generators/src/lib.rs#L72
\end{itemize}

\item[Generator Creation] \hfill \\
Takes a domain separation tag (\texttt{dst}) and produces two vectors of points: % https://github.com/serai-dex/serai/blob/48db06f901952b24bb38d7c7e256f798f08512cd/networks/monero/generators/src/lib.rs#L70
\begin{enumerate}
\item Create preimage = H.compress() || dst.  % https://github.com/serai-dex/serai/blob/48db06f901952b24bb38d7c7e256f798f08512cd/networks/monero/generators/src/lib.rs#L74-L75
\item For i from 0 to MAX\_MN - 1: % https://github.com/serai-dex/serai/blob/48db06f901952b24bb38d7c7e256f798f08512cd/networks/monero/generators/src/lib.rs#L78-L89
  \begin{itemize}
  \item Let j = 2*i.  % https://github.com/serai-dex/serai/blob/48db06f901952b24bb38d7c7e256f798f08512cd/networks/monero/generators/src/lib.rs#L80
  \item H[i] = hash\_to\_point(Keccak256(preimage || varint(j))).  % https://github.com/serai-dex/serai/blob/48db06f901952b24bb38d7c7e256f798f08512cd/networks/monero/generators/src/lib.rs#L82-L84
  \item G[i] = hash\_to\_point(Keccak256(preimage || varint(j+1))).  % https://github.com/serai-dex/serai/blob/48db06f901952b24bb38d7c7e256f798f08512cd/networks/monero/generators/src/lib.rs#L86-L88
  \end{itemize}
\end{enumerate}

\item[Output] \hfill \\
Returns \texttt{Generators} struct containing: % https://github.com/serai-dex/serai/blob/48db06f901952b24bb38d7c7e256f798f08512cd/networks/monero/generators/src/lib.rs#L58-L64
\begin{itemize}
\item G: Vector of MAX\_MN points for value commitments.  % https://github.com/serai-dex/serai/blob/48db06f901952b24bb38d7c7e256f798f08512cd/networks/monero/generators/src/lib.rs#L61
\item H: Vector of MAX\_MN points for mask commitments.  % https://github.com/serai-dex/serai/blob/48db06f901952b24bb38d7c7e256f798f08512cd/networks/monero/generators/src/lib.rs#L63
\end{itemize}
\end{description}

All operations maintain constant-time properties through:
\begin{itemize}
\item Use of \texttt{LazyLock} for thread-safe lazy initialization.  % https://github.com/serai-dex/serai/blob/48db06f901952b24bb38d7c7e256f798f08512cd/networks/monero/generators/src/lib.rs#L6 and https://github.com/serai-dex/serai/blob/48db06f901952b24bb38d7c7e256f798f08512cd/networks/monero/generators/src/lib.rs#L29 and https://github.com/serai-dex/serai/blob/48db06f901952b24bb38d7c7e256f798f08512cd/networks/monero/generators/src/lib.rs#L35
\item Constant-time point operations from curve25519-dalek.  % https://github.com/serai-dex/serai/blob/48db06f901952b24bb38d7c7e256f798f08512cd/networks/monero/generators/src/lib.rs#L10
\item Constant-time hash-to-point mapping.  % https://github.com/serai-dex/serai/blob/48db06f901952b24bb38d7c7e256f798f08512cd/networks/monero/generators/src/hash_to_point.rs#L1
\end{itemize}


% --------------------------------------------------- %
\subsection{monero-primitives (v0.1.0)}

\subsubsection{Overview}
The \texttt{monero-primitives} crate provides the core cryptographic operations required by Monero's protocol.  The crate operates in both \texttt{std} and \texttt{no-std} environments, with certain optimizations available when \texttt{std} is enabled.


% ---------- nesting UnreducedScalar, Commitment, Decoys into Core Types

\subsubsection{Core Types}

% Replacing the subsubsection function with the paragraph function
\subsubsubsection{ UnreducedScalar} 
A structure representing an unreduced scalar value, primarily used to handle legacy Monero code sections where scalar reduction was not enforced: 

% https://github.com/serai-dex/serai/blob/48db06f901952b24bb38d7c7e256f798f08512cd/networks/monero/primitives/src/unreduced_scalar.rs#L22-L30 


\begin{verbatim}
pub struct UnreducedScalar(pub [u8; 32]);
\end{verbatim}

Key operations:
\begin{itemize}
  \item \texttt{recover\_monero\_slide\_scalar}: Recovers the scalar that would have been incorrectly interpreted by Monero's \texttt{slide} function due to lack of reduction checks in Borromean range proofs. % https://github.com/serai-dex/serai/blob/48db06f901952b24bb38d7c7e256f798f08512cd/networks/monero/primitives/src/unreduced_scalar.rs#L110-L136
  \item \texttt{non\_adjacent\_form}: Computes the width-5 non-adjacent form (NAF) representation, intentionally matching Monero's potentially incorrect behavior. % https://github.com/serai-dex/serai/blob/48db06f901952b24bb38d7c7e256f798f08512cd/networks/monero/primitives/src/unreduced_scalar.rs#L52-L108
\end{itemize}

% Replacing the subsubsection function with the paragraph function
\subsubsubsection{ Commitment}
Represents a transparent Pedersen commitment with the following structure: % https://github.com/serai-dex/serai/blob/48db06f901952b24bb38d7c7e256f798f08512cd/networks/monero/primitives/src/lib.rs#L80-L88

\begin{verbatim}
pub struct Commitment {
    pub mask: Scalar,
    pub amount: u64
}
\end{verbatim}

Operations:
\begin{itemize}
  \item \texttt{calculate}: Computes the Pedersen commitment point as $amount \cdot G + mask \cdot H$.  % https://github.com/serai-dex/serai/blob/48db06f901952b24bb38d7c7e256f798f08512cd/networks/monero/primitives/src/lib.rs#L107-L110
  \item \texttt{zero}: Creates a commitment to zero using a mask value of 1.  % https://github.com/serai-dex/serai/blob/48db06f901952b24bb38d7c7e256f798f08512cd/networks/monero/primitives/src/lib.rs#L97-L100
  \item Serialization and deserialization functionality.
\end{itemize}

% Replacing the subsubsection function with the paragraph function
\subsubsubsection{ Decoys}
Manages ring signature data with the structure: % https://github.com/serai-dex/serai/blob/48db06f901952b24bb38d7c7e256f798f08512cd/networks/monero/primitives/src/lib.rs#L140-L146

\begin{verbatim}
pub struct Decoys {
    offsets: Vec<u64>,
    signer_index: u8,
    ring: Vec<[EdwardsPoint; 2]>
}
\end{verbatim}

Features:
\begin{itemize}
  \item Stores ring member positions as offsets from previous positions.  % https://github.com/serai-dex/serai/blob/48db06f901952b24bb38d7c7e256f798f08512cd/networks/monero/primitives/src/lib.rs#L184-L192
  \item Maintains key-commitment pairs for each ring member.  % https://github.com/serai-dex/serai/blob/48db06f901952b24bb38d7c7e256f798f08512cd/networks/monero/primitives/src/lib.rs#L145
  \item Provides access to the signer's position and ring members. % https://github.com/serai-dex/serai/blob/48db06f901952b24bb38d7c7e256f798f08512cd/networks/monero/primitives/src/lib.rs#L194-L207
\end{itemize}

\subsubsection{Core Operations}


% Replacing the subsubsection function with the paragraph function
\subsubsubsection{ Hash Operations}
\begin{itemize}
  \item \texttt{keccak256}: Computes Keccak-256 hash of input data.  % https://github.com/serai-dex/serai/blob/48db06f901952b24bb38d7c7e256f798f08512cd/networks/monero/primitives/src/lib.rs#L62-L65
  \item \texttt{keccak256\_to\_scalar}: Maps input to a scalar via $keccak256(\text{data}) \bmod l$, where $l$ is Ed25519's prime order. % https://github.com/serai-dex/serai/blob/48db06f901952b24bb38d7c7e256f798f08512cd/networks/monero/primitives/src/lib.rs#L67-L78
\end{itemize}

% Replacing the subsubsection function with the paragraph function
\subsubsubsection{ Cached Computations}
When the \texttt{std} feature is enabled:
\begin{itemize}
  \item \texttt{INV\_EIGHT}: Caches $8^{-1} \bmod l$.  % https://github.com/serai-dex/serai/blob/48db06f901952b24bb38d7c7e256f798f08512cd/networks/monero/primitives/src/lib.rs#L29-L44
  \item \texttt{G\_PRECOMP}: Maintains precomputed tables for the Ed25519 base point. % https://github.com/serai-dex/serai/blob/48db06f901952b24bb38d7c7e256f798f08512cd/networks/monero/primitives/src/lib.rs#L46-L60
\end{itemize}

\subsubsection{Implementation Details}

% Replacing the subsubsection function with the paragraph function
\subsubsubsection{ NAF Implementation}
% https://github.com/serai-dex/serai/blob/48db06f901952b24bb38d7c7e256f798f08512cd/networks/monero/primitives/src/unreduced_scalar.rs#L52-L105
\begin{itemize}
  \item Non-constant time operation, explicitly matching Monero's implementation.  % Documented at https://github.com/serai-dex/serai/blob/48db06f901952b24bb38d7c7e256f798f08512cd/networks/monero/primitives/src/unreduced_scalar.rs#L57
  \item Window size is fixed at 5 bits.  % https://github.com/serai-dex/serai/blob/48db06f901952b24bb38d7c7e256f798f08512cd/networks/monero/primitives/src/unreduced_scalar.rs#L69
  \item Edge cases:
    \begin{itemize}
      \item Handles overflow beyond 256 bits by truncating.  % https://github.com/serai-dex/serai/blob/48db06f901952b24bb38d7c7e256f798f08512cd/networks/monero/primitives/src/unreduced_scalar.rs#L70-L74
      \item Uses special carry propagation when the sum exceeds 15 or the difference goes below -15.  % https://github.com/serai-dex/serai/blob/48db06f901952b24bb38d7c7e256f798f08512cd/networks/monero/primitives/src/unreduced_scalar.rs#L78-L101
      \item Matches Monero's behavior even when cryptographically suboptimal.% 
    \end{itemize}
\end{itemize}

The NAF computation in \texttt{non\_adjacent\_form} follows these steps:
\begin{enumerate}
  \item Convert input to a bit array.  % https://github.com/serai-dex/serai/blob/48db06f901952b24bb38d7c7e256f798f08512cd/networks/monero/primitives/src/unreduced_scalar.rs#L43-L50
  \item Process bits sequentially.  % https://github.com/serai-dex/serai/blob/48db06f901952b24bb38d7c7e256f798f08512cd/networks/monero/primitives/src/unreduced_scalar.rs#L61-L63
  \item For each non-zero bit: % https://github.com/serai-dex/serai/blob/48db06f901952b24bb38d7c7e256f798f08512cd/networks/monero/primitives/src/unreduced_scalar.rs#L66-L104
    \begin{itemize}
      \item Examine the next 5 bits (window).  % https://github.com/serai-dex/serai/blob/48db06f901952b24bb38d7c7e256f798f08512cd/networks/monero/primitives/src/unreduced_scalar.rs#L69-L74
      \item Combine bits when sum $\leq 15$ or difference $\geq -15$.  % https://github.com/serai-dex/serai/blob/48db06f901952b24bb38d7c7e256f798f08512cd/networks/monero/primitives/src/unreduced_scalar.rs#L79-L84
      \item Handle carry propagation.  % https://github.com/serai-dex/serai/blob/48db06f901952b24bb38d7c7e256f798f08512cd/networks/monero/primitives/src/unreduced_scalar.rs#L91-L98
    \end{itemize}
\end{enumerate}

This intentionally matches Monero's implementation, including specific edge cases.

% Replacing the subsubsection function with the paragraph function
\subsubsubsection{ Scalar Recovery}
The \texttt{recover\_monero\_slide\_scalar} function:
\begin{itemize}
  \item Returns direct reduction if the high bit is not set.  % https://github.com/serai-dex/serai/blob/48db06f901952b24bb38d7c7e256f798f08512cd/networks/monero/primitives/src/unreduced_scalar.rs#L119-L124
  \item Otherwise reconstructs the scalar from its NAF representation.  % https://github.com/serai-dex/serai/blob/48db06f901952b24bb38d7c7e256f798f08512cd/networks/monero/primitives/src/unreduced_scalar.rs#L127-L134
  \item Uses precomputed odd scalars for efficiency.  % https://github.com/serai-dex/serai/blob/48db06f901952b24bb38d7c7e256f798f08512cd/networks/monero/primitives/src/unreduced_scalar.rs#L13-L20
\end{itemize}

\subsubsection{Security Considerations}

% Replacing the subsubsection function with the paragraph function
\subsubsubsection{ Critical Assumptions}
\begin{itemize}
  \item A zero hash result in \texttt{keccak256\_to\_scalar} causes a panic.  % https://github.com/serai-dex/serai/blob/48db06f901952b24bb38d7c7e256f798f08512cd/networks/monero/primitives/src/lib.rs#L76
  \item \texttt{UnreducedScalar} operations do not guarantee constant-time execution.  % https://github.com/serai-dex/serai/blob/48db06f901952b24bb38d7c7e256f798f08512cd/networks/monero/primitives/src/lib.rs#L55-L60
  \item Commitment masking relies on the discrete logarithm assumption in the Ed25519 group.  % https://github.com/serai-dex/serai/blob/48db06f901952b24bb38d7c7e256f798f08512cd/networks/monero/primitives/src/lib.rs#L80-L88
\end{itemize}

% Replacing the subsubsection function with the paragraph function
\subsubsubsection{ Implementation Notes}
\begin{itemize}
  \item \textbf{Serialization}:
    \begin{itemize}
      \item Uses custom formats, distinct from Monero protocol standards.  % https://github.com/serai-dex/serai/blob/48db06f901952b24bb38d7c7e256f798f08512cd/networks/monero/primitives/src/lib.rs#L112-L137 and https://github.com/serai-dex/serai/blob/48db06f901952b24bb38d7c7e256f798f08512cd/networks/monero/primitives/src/lib.rs#L209-L248
      \item Provides no backwards compatibility guarantees.  % 
      \item Limits \texttt{Decoys} serialization to Edwards points only.  % https://github.com/serai-dex/serai/blob/48db06f901952b24bb38d7c7e256f798f08512cd/networks/monero/primitives/src/lib.rs#L216-L223
    \end{itemize}
  \item \textbf{Memory Safety}:
    \begin{itemize}
      \item Implements \texttt{Zeroize} and \texttt{ZeroizeOnDrop} for sensitive data.  % eg. https://github.com/serai-dex/serai/blob/48db06f901952b24bb38d7c7e256f798f08512cd/networks/monero/primitives/src/lib.rs#L82
      \item Performs careful bounds checking on ring operations. % https://github.com/serai-dex/serai/blob/48db06f901952b24bb38d7c7e256f798f08512cd/networks/monero/primitives/src/lib.rs#L165-L167
    \end{itemize}
  \item \textbf{Compatibility Choices}:
    \begin{itemize}
      \item Matches Monero behavior even when suboptimal.  % Documented at https://github.com/serai-dex/serai/blob/48db06f901952b24bb38d7c7e256f798f08512cd/networks/monero/primitives/src/unreduced_scalar.rs#L52-L57
      \item Preserves specific edge cases for protocol compatibility.  % https://github.com/serai-dex/serai/blob/48db06f901952b24bb38d7c7e256f798f08512cd/networks/monero/primitives/src/unreduced_scalar.rs#L119-L124
      \item Retains variable-time operations where Monero uses them. % Documented at https://github.com/serai-dex/serai/blob/48db06f901952b24bb38d7c7e256f798f08512cd/networks/monero/primitives/src/unreduced_scalar.rs#L52-L57
    \end{itemize}
\end{itemize}

\begin{itemize}
  \item Implements \texttt{Zeroize} and \texttt{ZeroizeOnDrop} for sensitive data.
  \item Matches Monero's behavior for compatibility, even if suboptimal.
  \item Employs crate-specific serialization formats, not the Monero protocol standard.
\end{itemize}


% --------------------------------------------- %
\subsection{monero-rpc (v0.1.0)}
\label{sec:monero-rpc-crate}

The \texttt{monero-rpc} crate provides abstractions for
communicating with a Monero daemon, retrieving various data (blocks, transactions,
fee estimates, \emph{etc.}), and publishing transactions.  The crate relies on data
structures from \texttt{monero-serai}, using them to parse and serialize Monero
primitives (such as transactions and blocks).  It offers a specialized trait for
decoy selection.

\subsubsection{Overview}
\label{sec:monero-rpc-overview}

The \texttt{monero-rpc} crate defines:

\begin{enumerate}
    \item \textbf{\texttt{RpcError}}: An enumeration capturing errors that may
    arise when performing RPC calls (see
    \S\ref{sec:monero-rpc-rpcerror}). % https://github.com/serai-dex/serai/blob/48db06f901952b24bb38d7c7e256f798f08512cd/networks/monero/rpc/src/lib.rs#L44-L74

    \item \textbf{\texttt{Rpc} trait}: The primary abstraction for interacting
    with a Monero daemon.  It defines a set of asynchronous methods for
    retrieving blocks, transactions, and other chain data, as well as
    publishing transactions (see \S\ref{sec:monero-rpc-rpc-trait}). % https://github.com/serai-dex/serai/blob/48db06f901952b24bb38d7c7e256f798f08512cd/networks/monero/rpc/src/lib.rs#L244-L252

    \item \textbf{\texttt{DecoyRpc} trait}: A higher-level trait extending the
    concept of retrieving outputs for constructing ring signatures (see
    \S\ref{sec:monero-rpc-decoy-rpc}). % https://github.com/serai-dex/serai/blob/48db06f901952b24bb38d7c7e256f798f08512cd/networks/monero/rpc/src/lib.rs#L1001-L1028 and https://github.com/serai-dex/serai/blob/48db06f901952b24bb38d7c7e256f798f08512cd/networks/monero/rpc/src/lib.rs#L1049-L1269

    \item \textbf{Various helpers and supporting items}, such as:
    \begin{itemize}
      \item \texttt{ScannableBlock}, bundling a \emph{Monero} block together
      with its non-miner transactions (in pruned form), plus metadata used when
      scanning outputs. % https://github.com/serai-dex/serai/blob/48db06f901952b24bb38d7c7e256f798f08512cd/networks/monero/rpc/src/lib.rs#L76-L87
      \item \texttt{FeeRate} and \texttt{FeePriority}, providing abstractions
      for Monero fee estimation. % https://github.com/serai-dex/serai/blob/48db06f901952b24bb38d7c7e256f798f08512cd/networks/monero/rpc/src/lib.rs#L76-L87
      \item Utility functions to parse certain specialized binary responses from
      a node, such as \texttt{get\_o\_indexes.bin}. % https://github.com/serai-dex/serai/blob/48db06f901952b24bb38d7c7e256f798f08512cd/networks/monero/rpc/src/lib.rs#L812-L999
    \end{itemize}
\end{enumerate}

Implementors of \texttt{Rpc} are expected to supply the low-level HTTP or
transport logic, including authentication if applicable.  The actual requests and
responses are shaped to match Monero’s JSON-RPC and specialized “binary”
endpoints.  This allows various backends (e.g., over Tor/i2p, local node,
dedicated HTTP crates) to be slotted in.

\subsubsection{\texttt{RpcError} Enumeration}
\label{sec:monero-rpc-rpcerror}

The \texttt{RpcError} enum defines all error conditions encountered by the RPC
layer: % https://github.com/serai-dex/serai/blob/48db06f901952b24bb38d7c7e256f798f08512cd/networks/monero/rpc/src/lib.rs#L43-L74

\begin{itemize}
    \item \textbf{\texttt{InternalError(String)}} --- Typically signals some
    logical problem unrelated to standard node replies (for instance,
    constructing a request with out-of-range parameters).

    \item \textbf{\texttt{ConnectionError(String)}} --- Indicates an inability
    to reach or properly communicate with the daemon (e.g.\ timeouts, malformed
    responses, or connectivity disruptions).

    \item \textbf{\texttt{InvalidNode(String)}} --- The remote node returned
    data not conforming to the Monero protocol, or otherwise gave impossible or
    contradictory information (suspecting a malicious or misconfigured node).

    \item \textbf{\texttt{TransactionsNotFound(Vec<[u8; 32]>)}} --- One or more
    requested transactions could not be retrieved by the node.

    \item \textbf{\texttt{PrunedTransaction}} --- A transaction was retrieved in
    a pruned form when unpruned data was required.  In the current code, pruned
    transactions are not considered usable for certain operations.

    \item \textbf{\texttt{InvalidTransaction([u8; 32])}} --- The node claims a
    transaction is valid, but it fails local parsing or verification.

    \item \textbf{\texttt{InvalidFee}} --- The fee returned by the node was
    nonsensical or out-of-range for safe usage.

    \item \textbf{\texttt{InvalidPriority}} --- The given fee priority could not
    be honored or mapped (e.g.\ out of the valid \texttt{1--4} range for
    Monero's known fee multipliers).
\end{itemize}

Such typed error handling allows the upper layers or other consumers of the
\texttt{monero-rpc} crate to correctly distinguish user-facing issues (like
\texttt{TransactionsNotFound}) from transport or malicious-node issues
(\texttt{ConnectionError} or \texttt{InvalidNode}).

\subsubsection{\texttt{Rpc} Trait}
\label{sec:monero-rpc-rpc-trait}

The \texttt{Rpc} trait is the centerpiece of the crate, defining the primary
asynchronous calls for interacting with a Monero daemon.  Its methods can be
broken into three categories: \emph{primitive calls}, \emph{block/transaction
retrieval}, and \emph{transaction publishing}. % https://github.com/serai-dex/serai/blob/48db06f901952b24bb38d7c7e256f798f08512cd/networks/monero/rpc/src/lib.rs#L244-L999

\subsubsection{\texttt{post}}
\label{sec:monero-rpc-rpc-trait-post}

\begin{verbatim}
fn post(
    &self,
    route: &str,
    body: Vec<u8>
) -> impl Future<Output = Result<Vec<u8>, RpcError>> + Send;
\end{verbatim}
% https://github.com/serai-dex/serai/blob/48db06f901952b24bb38d7c7e256f798f08512cd/networks/monero/rpc/src/lib.rs#L253-L260

At the lowest level, \texttt{post} takes an HTTP route string (e.g.:
\texttt{"get\_transactions"}) and a raw byte vector containing the request body.
It returns a future resolving to a \texttt{Result} with the raw byte vector for
the response (or an error).  Implementors must handle:

\begin{itemize}
    \item HTTP or HTTPS connections.
    \item Digest or basic authentication.
    \item Connection pooling or caching.
\end{itemize}

Nothing in \texttt{monero-rpc} enforces a particular method of connection;
\texttt{post} abstracts this detail away.

\subsubsection{\texttt{rpc\_call} and \texttt{json\_rpc\_call}}
\label{sec:monero-rpc-rpc-trait-jsonrpc}

\begin{verbatim}
fn rpc_call<Params: Serialize + Debug,
            Response: DeserializeOwned + Debug>(
    &self,
    route: &str,
    params: Option<Params>
) -> impl Future<Output = Result<Response, RpcError>> + Send;

fn json_rpc_call<Response: DeserializeOwned + Debug>(
    &self,
    method: &str,
    params: Option<Value>
) -> impl Future<Output = Result<Response, RpcError>> + Send;
\end{verbatim}
% https://github.com/serai-dex/serai/blob/48db06f901952b24bb38d7c7e256f798f08512cd/networks/monero/rpc/src/lib.rs#L262-L287 and https://github.com/serai-dex/serai/blob/48db06f901952b24bb38d7c7e256f798f08512cd/networks/monero/rpc/src/lib.rs#L289-L302

These two methods use \texttt{post} internally, providing:

\begin{enumerate}
    \item \texttt{rpc\_call}: For “raw” RPC calls, typically when a daemon
    endpoint does not follow the JSON-RPC 2.0 standard but instead uses
    POST-based JSON data (for example, \texttt{get\_transactions}).

    \item \texttt{json\_rpc\_call}: For endpoints explicitly requiring
    \texttt{json\_rpc} protocol usage (e.g. \\\texttt{get\_block\_header\_by\_height}).
    The parameters and method name are placed into the JSON-RPC structure, sent,
    and the result is deserialized back to a typed \texttt{Response}.
\end{enumerate}

Both automatically handle JSON deserialization into strongly typed Rust
structures.  If the response is malformed or the node returns an error-like
structure, \texttt{RpcError::InvalidNode} or \texttt{RpcError::ConnectionError}
may be produced.

\subsubsection{\texttt{get\_height}, \texttt{get\_transactions}, and Block-Related Methods}
\rigo{not sure why Block-Related Methods is a different font. Is it a function? Because then it can be the same font}

The crate includes convenience methods for:
\begin{itemize}
    \item \texttt{get\_height()}: Returns the current chain height (the number
    of blocks, with the genesis block counted as height 1). % https://github.com/serai-dex/serai/blob/48db06f901952b24bb38d7c7e256f798f08512cd/networks/monero/rpc/src/lib.rs#L338-L354
    \item \texttt{get\_block}, \texttt{get\_block\_by\_number}, and
    \texttt{get\_block\_hash}: Retrieve and verify specific blocks, either by
    zero-indexed block number or by 32-byte hash.  The code ensures the block’s
    serialized hash matches the node’s reported identifier, guarding against
    untrusted or corrupted data. % https://github.com/serai-dex/serai/blob/48db06f901952b24bb38d7c7e256f798f08512cd/networks/monero/rpc/src/lib.rs#L504-L584
    \item \texttt{get\_transactions} and \texttt{get\_pruned\_transactions}: % https://github.com/serai-dex/serai/blob/48db06f901952b24bb38d7c7e256f798f08512cd/networks/monero/rpc/src/lib.rs#L483
    Fetch zero or more transactions by their 32-byte hash, verifying they match
    local expectations (checking the computed transaction hash).  Pruned or
    unpruned forms are selectively parsed via the relevant entry point.
    \item \texttt{get\_transaction} (unpruned) and
    \texttt{get\_pruned\_transaction}: Single-transaction convenience wrappers. % https://github.com/serai-dex/serai/blob/48db06f901952b24bb38d7c7e256f798f08512cd/networks/monero/rpc/src/lib.rs#L485-L494
\end{itemize}

Block retrieval can optionally return a \texttt{ScannableBlock}, packaging a
\texttt{Block} and its pruned non-coinbase transactions, allowing scanning for
RingCT outputs. % https://github.com/serai-dex/serai/blob/48db06f901952b24bb38d7c7e256f798f08512cd/networks/monero/rpc/src/lib.rs#L586-L655

\subsubsection{\texttt{publish\_transaction}}
\label{sec:monero-rpc-rpc-trait-publish}

\begin{verbatim}
fn publish_transaction(
    &self,
    tx: &Transaction
) -> impl Future<Output = Result<(), RpcError>> + Send;
\end{verbatim}
% https://github.com/serai-dex/serai/blob/48db06f901952b24bb38d7c7e256f798f08512cd/networks/monero/rpc/src/lib.rs#L742-L777

Publishes a \texttt{Transaction} to the network.  It leverages the
\texttt{send\_raw\_transaction} route (or equivalent) on the Monero daemon.  
Should the daemon reject the transaction (for instance, if it conflicts with a
recently seen double spend), an appropriate \texttt{RpcError} is returned.

\subsubsection{\texttt{get\_fee\_rate}}
\label{sec:monero-rpc-rpc-trait-fee-estimate}

\begin{verbatim}
fn get_fee_rate(
    &self,
    priority: FeePriority
) -> impl Future<Output = Result<FeeRate, RpcError>> + Send;
\end{verbatim}
% https://github.com/serai-dex/serai/blob/48db06f901952b24bb38d7c7e256f798f08512cd/networks/monero/rpc/src/lib.rs#L675-L740

Requests fee estimates for a particular \texttt{FeePriority} (one of
\texttt{Unimportant}, \texttt{Normal}, \texttt{Elevated}, \texttt{Priority}, or
\texttt{Custom}).  Internally, it interprets the JSON response to obtain the fee
in “per-weight” units and a quantization mask.  The resulting \texttt{FeeRate}
structure:

\begin{itemize}
    \item \texttt{per\_weight}: The base fee per weight unit.
    \item \texttt{mask}: A mask used to round the final transaction fee upward.
\end{itemize}

The consumer can compute the final fee for a transaction based on the
transaction weight and this \texttt{FeeRate}.  If an unrecognized or invalid
priority is requested, \texttt{RpcError::InvalidPriority} is raised.

\subsubsection{\texttt{generate\_blocks}}
\label{sec:monero-rpc-rpc-trait-generate-blocks}

While not part of a typical production Monero daemon (where block generation is
not done via RPC in mainnet environments), \texttt{generate\_blocks} is useful
in a local testing context (e.g.\ in a private regtest setup).  It creates
blocks, awarding the block reward to the specified \texttt{Address}.

\begin{verbatim}
fn generate_blocks<const ADDR_BYTES: u128>(
    &self,
    address: &Address<ADDR_BYTES>,
    block_count: usize
) -> impl Future<Output = Result<(Vec<[u8; 32]>, usize), RpcError>> + Send;
\end{verbatim}
% https://github.com/serai-dex/serai/blob/48db06f901952b24bb38d7c7e256f798f08512cd/networks/monero/rpc/src/lib.rs#L779-L810

This returns a list of the newly mined blocks’ hashes and the resulting height.

\subsubsection{\texttt{DecoyRpc} Trait}
\label{sec:monero-rpc-decoy-rpc}

The \texttt{DecoyRpc} trait extends the base RPC for selecting decoy outputs.  
When constructing Monero transactions, ring signatures typically reference
random “decoy” outputs on-chain, enabling privacy.  \texttt{DecoyRpc} adds
specialized queries for zero-amount RingCT outputs:

\begin{itemize}
    \item \texttt{get\_output\_distribution\_end\_height()}: Reports the maximum
    block index used in the distribution.  Typically this equals the chain
    height. % https://github.com/serai-dex/serai/blob/48db06f901952b24bb38d7c7e256f798f08512cd/networks/monero/rpc/src/lib.rs#L1007-L1013
    \item \texttt{get\_output\_distribution(range: Range<usize>)}: Returns a
    vector of cumulative output counts for that block range.  By focusing on
    zero-amount outputs, it leverages the \texttt{get\_output\_distribution}
    endpoint. % https://github.com/serai-dex/serai/blob/48db06f901952b24bb38d7c7e256f798f08512cd/networks/monero/rpc/src/lib.rs#L1015-L1022
    \item \texttt{get\_outs(indexes: \&[u64])}: Fetches details for a batch of
    RingCT outputs, including the block height, unlocking status, compressed key
    (\texttt{C\_i}), and the commitment (\texttt{mask}). % https://github.com/serai-dex/serai/blob/48db06f901952b24bb38d7c7e256f798f08512cd/networks/monero/rpc/src/lib.rs#L1024-L1028
    \item \texttt{get\_unlocked\_outputs(indexes, height, fingerprintable)}:
    Returns the subset of outputs that are actually unlocked by the node’s
    current chain state, or by a purely deterministic check (when
    \texttt{fingerprintable\_deterministic} is true).  This helps ensure
    decoys represent valid, spendable outputs from the node’s perspective. % https://github.com/serai-dex/serai/blob/48db06f901952b24bb38d7c7e256f798f08512cd/networks/monero/rpc/src/lib.rs#L1030-L1046
\end{itemize}

In typical usage, one might maintain a local database of zero-amount outputs
(built incrementally), but \texttt{DecoyRpc} allows direct on-demand retrieval
from a node if required.  Implementing \texttt{DecoyRpc} on top of \texttt{Rpc}
is straightforward, as the crate includes a default blanket implementation for
any type satisfying \texttt{Rpc}.

\subsubsection{Supporting Types and Structures}
\label{sec:monero-rpc-supporting-types}


% Replacing the subsubsection function with the paragraph function
\subsubsubsection{ \texttt{ScannableBlock}}
\label{sec:monero-rpc-supporting-types-scannableblock}

\begin{verbatim}
pub struct ScannableBlock {
    pub block: Block,
    pub transactions: Vec<Transaction<Pruned>>,
    pub output_index_for_first_ringct_output: Option<u64>,
}
\end{verbatim}
% https://github.com/serai-dex/serai/blob/48db06f901952b24bb38d7c7e256f798f08512cd/networks/monero/rpc/src/lib.rs#L76-L87

\texttt{ScannableBlock} encapsulates a fully read, unpruned coinbase
(\textit{miner}) transaction plus the pruned forms of other transactions, along
with an optional \texttt{output\_index\_for\_first\_ringct\_output}.  This index
helps to avoid repeated queries for the output index of every single
RingCT output in the block, which can be a privacy leak.

% Replacing the subsubsection function with the paragraph function
\subsubsubsection{ \texttt{FeeRate}}
\label{sec:monero-rpc-supporting-types-feerate}

\begin{verbatim}
pub struct FeeRate {
    per_weight: u64,
    mask: u64,
}
\end{verbatim}
% https://github.com/serai-dex/serai/blob/48db06f901952b24bb38d7c7e256f798f08512cd/networks/monero/rpc/src/lib.rs#L89-L152

This data structure is an interpreted result from the node’s fee estimate.  
The 

\texttt{FeeRate::calculate\_fee\_from\_weight} method performs final rounding:
\[
\text{fee} = \left\lceil \frac{\text{per\_weight}
             \times \text{tx\_weight}}{\text{mask}} \right\rceil
             \times \text{mask}.
\]
% https://github.com/serai-dex/serai/blob/48db06f901952b24bb38d7c7e256f798f08512cd/networks/monero/rpc/src/lib.rs#L138-L151


% Replacing the subsubsection function with the paragraph function
\subsubsubsection{ \texttt{FeePriority}}
\label{sec:monero-rpc-supporting-types-feepriority}

\begin{verbatim}
pub enum FeePriority {
    Unimportant,
    Normal,
    Elevated,
    Priority,
    Custom { priority: u32 },
}
\end{verbatim}
% https://github.com/serai-dex/serai/blob/48db06f901952b24bb38d7c7e256f798f08512cd/networks/monero/rpc/src/lib.rs#L154-L173

Monero typically recognizes four distinct priority levels: “unimportant,”
“normal,” “elevated,” and “priority.”  Each internally maps to a numeric
multiplier for the base fee.  The \texttt{Custom} variant allows direct numeric
control, though the node might reject unknown or extreme values.

\subsubsection{Security Considerations}
\label{sec:monero-rpc-security}

\begin{enumerate}
    \item \textbf{Node Trust and Validation}: Since
    \texttt{monero-rpc} can operate on untrusted nodes, it attempts some
    validation (for example, verifying that a transaction’s hash matches the
    requested hash).  However, in general, the node is relied upon for essential
    data about chain state.  The \texttt{InvalidNode} error signals a mismatch
    from protocol expectations.
    % https://github.com/serai-dex/serai/blob/48db06f901952b24bb38d7c7e256f798f08512cd/networks/monero/rpc/src/lib.rs#L418-L428 for example.
    
    \item \textbf{Confidentiality of Queries}: Repeatedly querying the node
    about certain output indexes or unconfirmed transactions can leak usage
    patterns.  Future or external solutions may integrate local caches or batch
    requests to reduce fingerprinting.
    
    \item \textbf{Pruned Transactions}: If the node can only supply pruned
    data for certain legacy or low-mixin transactions, the library raises
    \texttt{RpcError::PrunedTransaction} if unpruned data is strictly required.  
    Users must ensure they only proceed when they can meaningfully handle a
    pruned transaction (e.g.\ scanning coinbase outputs).
\end{enumerate}

\subsubsection{Summary of the \texttt{monero-rpc} Crate}
\label{sec:monero-rpc-conclusion}

The \texttt{monero-rpc} crate provides a clear, typed interface for interacting
with Monero daemons, either trusted or untrusted.  By defining the \texttt{Rpc}
trait, it ensures a uniform approach to retrieving, verifying, and publishing
transactions, while the \texttt{DecoyRpc} trait extends it to fetch random ring
outputs in a convenient, standardized manner.  This design integrates into
higher layers of the \texttt{monero-serai} stack, making it easy to develop new
functionality (e.g.\ bridging, multi-signature orchestrations, or custom
wallets) in a modular and verifiable way.


% ----------------------------------------------------%
\subsection{monero-bulletproofs (v0.1.0)}
\label{sec:monero-bulletproofs-impl}

The \texttt{monero-bulletproofs} crate implements both the original Bulletproofs range proof scheme and the newer Bulletproofs+ scheme for the Monero protocol.  The library is located within the Serai repository under \texttt{\small networks/monero/ringct/bulletproofs}.  Its purpose is to generate and verify range proofs over amounts in confidential transactions, ensuring that the amounts are in a valid range (specifically $[0, 2^{64})$) without revealing the amounts themselves.  Below is an overview of its structure, methods, and verification flow, referencing the relevant source files.

\subsubsection{Structure and Entry Points}
The crate exposes its main functionality via:
\begin{itemize}
    \item \texttt{Bulletproof}: An enum representing either the original Bulletproof or a Bulletproof+ proof. % https://github.com/serai-dex/serai/blob/48db06f901952b24bb38d7c7e256f798f08512cd/networks/monero/ringct/bulletproofs/src/lib.rs#L59-L69
    \item \texttt{prove} and \texttt{prove\_plus}: Functions that construct the respective Bulletproof or Bulletproof+ proofs for a list of commitments. % https://github.com/serai-dex/serai/blob/48db06f901952b24bb38d7c7e256f798f08512cd/networks/monero/ringct/bulletproofs/src/lib.rs#L108-L146
    \item \texttt{verify} and \texttt{batch\_verify}: Functions that verify these proofs, either in a standalone or batched manner. % https://github.com/serai-dex/serai/blob/48db06f901952b24bb38d7c7e256f798f08512cd/networks/monero/ringct/bulletproofs/src/lib.rs#L148-L173
\end{itemize}

Internally, the crate is separated into several modules:
\begin{itemize}
    \item \texttt{core}: Provides low-level operations such as multi-exponentiation routines and challenge product computation.  
    \item \texttt{batch\_verifier}: Holds accumulators that enable batching multiple Bulletproof or Bulletproof+ verifications into a single multi-exponentiation.  
    \item \texttt{original}: Implements the original Bulletproofs approach, including the inner-product proof (\texttt{IpProof}) used by \texttt{AggregateRangeProof}.  
    \item \texttt{plus}: Implements Bulletproofs+ as described by the Monero project.  It includes the improved weighted inner product proof (\texttt{WipProof}) and the aggregated range proof mechanism that uses it.  
    \item \texttt{scalar\_vector} and \texttt{point\_vector}: Utility types for vectors of \texttt{Scalar} and \texttt{EdwardsPoint}, offering safe indexing and algebraic operations.  
    \item \texttt{tests}: Internal tests that validate correctness across both schemes, employing batched and single verifications.  
\end{itemize}

\subsubsection{Range Proof Representation}
The top-level enum \texttt{Bulletproof} is:
\begin{itemize}
    \item \texttt{Original(OriginalProof)}: The classic Bulletproof structure using the original protocol design.  
    \item \texttt{Plus(PlusProof)}: The Bulletproof+ structure, notably smaller and with a slightly different internal proving mechanism.  
\end{itemize}

Each structure internally contains the group elements and scalars required for proof verification.  For instance, \texttt{OriginalProof} holds:
\begin{itemize}
    \item \texttt{(A, S, T1, T2)}: Points used in the range proof polynomial commitment.  
    \item \texttt{tau\_x, mu, t\_hat}: Challenge-related scalars for the polynomial identity check.  
    \item \texttt{ip}: An \texttt{IpProof} implementing the inner product argument.  
\end{itemize}
% https://github.com/serai-dex/serai/blob/48db06f901952b24bb38d7c7e256f798f08512cd/networks/monero/ringct/bulletproofs/src/original/mod.rs#L29-L39

Likewise, \texttt{PlusProof} holds:
\begin{itemize}
    \item \texttt{A}: The aggregated vector commitment.  
    \item \texttt{wip (WipProof)}: A specialized weighted inner product proof, the core cryptographic novelty of Bulletproofs+.  
\end{itemize}
% https://github.com/serai-dex/serai/blob/48db06f901952b24bb38d7c7e256f798f08512cd/networks/monero/ringct/bulletproofs/src/plus/aggregate_range_proof.rs#L41-L47

\subsubsection{Aggregated Range Proof Generation}
Both Bulletproof and Bulletproof+ provide \emph{aggregated} proofs that combine multiple range proofs for multiple outputs into a single proof whose size grows only logarithmically in the number of outputs.  Each scheme follows a ``commit-and-challenge'' approach similar to standard Bulletproof protocols, but they differ in their choice of transcript hashing and the particular steps for the inner product argument.  

\paragraph{Original Bulletproof (\texttt{original} module)}
The original code is in \texttt{original/mod.rs} and \\
\texttt{original/inner\_product.rs}, closely mirroring the \textit{Bulletproofs} paper.  Key steps:
\begin{enumerate}
    \item \emph{Precomputation:} Generators \texttt{G} and \texttt{H} are allocated, one for each bit of each commitment.  The maximum number of commitments is specified by \texttt{MAX\_COMMITMENTS}. % MAX_COMMITMENTS imported at https://github.com/serai-dex/serai/blob/48db06f901952b24bb38d7c7e256f798f08512cd/networks/monero/ringct/bulletproofs/src/original/mod.rs#L9 and generators at https://github.com/serai-dex/serai/blob/48db06f901952b24bb38d7c7e256f798f08512cd/networks/monero/ringct/bulletproofs/src/original/mod.rs#L17 (generators generated in https://github.com/serai-dex/serai/blob/48db06f901952b24bb38d7c7e256f798f08512cd/networks/monero/ringct/bulletproofs/build.rs)
    \item \emph{Polynomial Commitments:} The protocol encodes bits of the amounts in vectors $(a_L, a_R)$, from which it forms commitments $(A, S)$. % https://github.com/serai-dex/serai/blob/48db06f901952b24bb38d7c7e256f798f08512cd/networks/monero/ringct/bulletproofs/src/original/mod.rs#L120-L166
    \item \emph{Challenges:} The code uses Keccak-256 (converted to a \texttt{Scalar} via \texttt{keccak256\_to\_scalar}) to derive random challenges $y, z, x$. % https://github.com/serai-dex/serai/blob/48db06f901952b24bb38d7c7e256f798f08512cd/networks/monero/ringct/bulletproofs/src/original/mod.rs#L65-L82 transcript_A_S for y,z challenges, transcript_T12 for x challenge
    \item \emph{Constrained Polynomial Identity:} The aggregator constructs $(T_1, T_2)$ to bind the polynomial terms, and merges everything with the challenge $x$ to produce $t_{\hat{}}$, $\tau_x$, and $\mu$. %
    \item \emph{Inner Product Proof:} The \texttt{IpProof} is generated by recursively splitting vectors in half and committing to partial inner products, storing the commitments $(L_i, R_i)$. % See https://github.com/serai-dex/serai/blob/48db06f901952b24bb38d7c7e256f798f08512cd/networks/monero/ringct/bulletproofs/src/original/mod.rs#L184-L222 for T1 and T2 commitments, transcript/challenge x, t_hat, tau_x, and mu.
\end{enumerate}

|paragraph{Bulletproof+ (\texttt{plus} module)}
Bulletproof+ is provided by the \texttt{plus/} folder, primarily in \texttt{weighted\_inner\_product.rs} and \texttt{aggregate\_range\_proof.rs}.  The official Monero code slightly alters the structure of the proof to reduce proof size, while still maintaining logarithmic size.  Notable elements:
\begin{enumerate}
    \item The \emph{Weighted Inner Product} (\texttt{WipStatement}) generalizes the classic inner product.  It incorporates certain transformations by weighting vectors with powers of $y$ and rearranging final group elements. % https://github.com/serai-dex/serai/blob/48db06f901952b24bb38d7c7e256f798f08512cd/networks/monero/ringct/bulletproofs/src/plus/weighted_inner_product.rs#L15-L28
    \item The \emph{Generators} are re-labeled for Monero's usage: \texttt{g\_bold} and \texttt{h\_bold} in code, though the actual roles are reversed (Monero uses $H$ as the value basepoint and $G$ for mask derivation). % https://github.com/serai-dex/serai/blob/48db06f901952b24bb38d7c7e256f798f08512cd/networks/monero/ringct/bulletproofs/src/plus/mod.rs#L25-L35
    \item Aggregation is performed similarly with powers of $z$ and $y$, but the final step uses a more efficient \texttt{WipProof}.  
    \item The function \texttt{prove\_plus} sets up the aggregated statement, computes an $A$ commitment, then calls \texttt{WipStatement::prove} to produce the final weighted inner product proof. % See https://github.com/serai-dex/serai/blob/48db06f901952b24bb38d7c7e256f798f08512cd/networks/monero/ringct/bulletproofs/src/plus/aggregate_range_proof.rs#L186-L203 for z powers and https://github.com/serai-dex/serai/blob/48db06f901952b24bb38d7c7e256f798f08512cd/networks/monero/ringct/bulletproofs/src/plus/aggregate_range_proof.rs#L213-L234 for y powers and aggregation.
\end{enumerate}

At verification time, \texttt{aggregate\_range\_proof.rs} re-derives the same hashed challenges from the set of commitments and checks consistency with the \texttt{WipProof}.  As with the original proof, a batched approach merges all terms into a single multi-scalar multiplication check.  

\subsubsection{Core Routines and Utilities}

\paragraph{\texttt{core/mod.rs}}
Contains low-level primitives:
\begin{itemize}
    \item \texttt{multiexp} and \texttt{multiexp\_vartime}: Perform (variable-time) multi-exponentiation to accumulate the proof checks. % https://github.com/serai-dex/serai/blob/48db06f901952b24bb38d7c7e256f798f08512cd/networks/monero/ringct/bulletproofs/src/core.rs#L31-L74
    \item \texttt{challenge\_products}: Generates the partial products of the challenges $(e_i, e_i^{-1})$ for the recursive proof steps, optimizing repeated multiplications. % https://github.com/serai-dex/serai/blob/48db06f901952b24bb38d7c7e256f798f08512cd/networks/monero/ringct/bulletproofs/src/core.rs#L50-L74
\end{itemize}

These functions are the primary means of combining scalar-point pairs for cryptographic checks throughout the library.  

\paragraph{\texttt{batch\_verifier.rs}}
Implements the \texttt{BatchVerifier} structure, storing:
\begin{itemize}
    \item \texttt{original}: An accumulator for the original Bulletproof scheme. % https://github.com/serai-dex/serai/blob/48db06f901952b24bb38d7c7e256f798f08512cd/networks/monero/ringct/bulletproofs/src/batch_verifier.rs#L55-L62
    \item \texttt{plus}: An accumulator for Bulletproof+. % https://github.com/serai-dex/serai/blob/48db06f901952b24bb38d7c7e256f798f08512cd/networks/monero/ringct/bulletproofs/src/batch_verifier.rs#L64-L73
\end{itemize}

Each accumulator collects scalar multipliers for fixed and variable basepoints (respectively \texttt{G}, \texttt{H}, per-commitment bases, etc.).  Finally, \texttt{verify} executes a \emph{single} call to \texttt{vartime\_multiscalar\_mul} to confirm all queued proofs in batch.  

\paragraph{\texttt{scalar\_vector.rs} and \texttt{point\_vector.rs}}
Define lightweight wrappers around \texttt{Vec<Scalar>} and \texttt{Vec<EdwardsPoint>} to simplify:
\begin{itemize}
    \item Indexing with bounds checks (in debug mode). % https://github.com/serai-dex/serai/blob/48db06f901952b24bb38d7c7e256f798f08512cd/networks/monero/ringct/bulletproofs/src/scalar_vector.rs#L16-L26
    \item Element-wise operations: addition, subtraction, or multiplication on each vector element. % https://github.com/serai-dex/serai/blob/48db06f901952b24bb38d7c7e256f798f08512cd/networks/monero/ringct/bulletproofs/src/scalar_vector.rs#L28-L85
    \item Inner product computation: a straightforward $(\mathbf{a},\mathbf{b}) \mapsto \sum a_i b_i$. % https://github.com/serai-dex/serai/blob/48db06f901952b24bb38d7c7e256f798f08512cd/networks/monero/ringct/bulletproofs/src/scalar_vector.rs#L124-L130
\end{itemize}

These wrappers help maintain correctness by preventing mistakes in indexing or length mismatches between vectors.  

\subsubsection{Transcript and Fiat--Shamir Challenges}
Throughout the protocol, challenges ($y, z, x, \dots$) are computed by hashing various group elements and scalars with Keccak-256.  This is done via:
\begin{itemize}
    \item \texttt{keccak256\_to\_scalar(...)}: Converts a Keccak-256 hash into a \texttt{Scalar}. % https://github.com/serai-dex/serai/blob/48db06f901952b24bb38d7c7e256f798f08512cd/networks/monero/ringct/bulletproofs/src/plus/transcript.rs#L13-L17
    \item Combined with domain separators and partial transcripts (e.g.\ $\texttt{A}$, $\texttt{S}$, $\texttt{T1}$, $\texttt{T2}$ in the original scheme). % https://github.com/serai-dex/serai/blob/48db06f901952b24bb38d7c7e256f798f08512cd/networks/monero/ringct/bulletproofs/src/plus/transcript.rs#L8-L11
\end{itemize}

In \texttt{plus/transcript.rs}, Bulletproof+ uses an additional domain separation constant \\
\verb|bulletproof_plus_transcript|, plus Monero's \texttt{hash\_to\_point} trick, though its necessity is not entirely clear from the code.  

\subsubsection{Security Observations}
\begin{itemize}
    \item \emph{Cofactor clearing}: By multiplying all external inputs (\texttt{EdwardsPoint} commitments) by \texttt{INV\_EIGHT} then re-multiplying by 8, the code ensures those points lie in the primary subgroup.  This is critical to avoid torsion-based forgeries. % See https://github.com/serai-dex/serai/blob/48db06f901952b24bb38d7c7e256f798f08512cd/networks/monero/ringct/bulletproofs/src/original/mod.rs#L61-L62 for INV_EIGHT multiplication and https://github.com/serai-dex/serai/blob/48db06f901952b24bb38d7c7e256f798f08512cd/networks/monero/ringct/bulletproofs/src/batch_verifier.rs#L23-L53 for is_identity check.
    \item \emph{Batch verification correctness}: Accumulated scalars and points are carefully updated to ensure the final check equals the identity only if all aggregated proofs are valid.  If any proof is malformed, the final multi-exponentiation will yield a non-identity point. % See latter reference above.
    \item \emph{Randomness and transcript:} The crate depends on externally provided randomness (via \texttt{rand\_core} or through \texttt{OsRng} in tests).  The Fiat--Shamir heuristic is used for non-interactive challenge derivation. % See https://github.com/serai-dex/serai/blob/48db06f901952b24bb38d7c7e256f798f08512cd/networks/monero/ringct/bulletproofs/src/plus/transcript.rs for Fiat-Shamir.
\end{itemize}


% --------------------------------------------------- %
\subsection{\texttt{monero-address (v0.1.0)}}

\subsubsection{Purpose}

The \texttt{monero-address} crate provides functionality for handling Monero addresses, including standard, subaddress, integrated, and featured addresses.  It also supports encoding and decoding Monero addresses in Base58Check format.  

\subsubsection{Internal Dependencies}

\begin{itemize}
    \item \texttt{monero-io (v0.1.0)}.  
    \item \texttt{monero-primitives (v0.1.0)}.  
\end{itemize}

\subsubsection{Structure}

A standard library crate, with the corresponding entry point at \path{/wallet/address/src/lib.rs}.  

\begin{itemize}
    \item \texttt{base58check} module.  
    \item Tests at \path{/wallet/address/src/tests.rs}.  
\end{itemize}

\subsubsection{Functionality}

The \texttt{monero-address} crate defines several types of Monero addresses and provides encoding, decoding, and validation functionality.  

\paragraph{Address Types}

The \texttt{monero-address} crate defines the following address types.  

\begin{itemize}
    \item \textbf{Legacy:} A standard Monero address (public spend and view keys).  
    \item \textbf{Legacy Integrated:} A legacy address with an embedded 8-byte payment ID.  
    \item \textbf{Subaddress:} A derived address that allows the receiver to differentiate sources of funds.  
    \item \textbf{Featured Address:} An extended address format supporting subaddresses, embedded payment IDs, and a guarantee against the burning bug.  
\end{itemize}
% https://github.com/serai-dex/serai/blob/48db06f901952b24bb38d7c7e256f798f08512cd/networks/monero/wallet/address/src/lib.rs#L24-L58

Each address type is represented using the \texttt{AddressType} enum, which provides methods for checking whether an address is a subaddress, retrieving its embedded payment ID, and determining if it is guaranteed. % https://github.com/serai-dex/serai/blob/48db06f901952b24bb38d7c7e256f798f08512cd/networks/monero/wallet/address/src/lib.rs#L60-L86

\paragraph{Network Identification}

The \texttt{Network} enum defines three possible Monero network types.  

\begin{itemize}
    \item \textbf{Mainnet:} The primary Monero blockchain.  
    \item \textbf{Stagenet:} A staging network with the same rules as mainnet, used for testing.  
    \item \textbf{Testnet:} A network used to test new features before they are deployed.  
\end{itemize}
% https://github.com/serai-dex/serai/blob/48db06f901952b24bb38d7c7e256f798f08512cd/networks/monero/wallet/address/src/lib.rs#L186-L199

Each network is associated with specific address prefix bytes to distinguish them from each other. % https://github.com/serai-dex/serai/blob/48db06f901952b24bb38d7c7e256f798f08512cd/networks/monero/wallet/address/src/lib.rs#L339-L347

\paragraph{Base58Check Encoding and Decoding}

The \texttt{base58check} module implements Base58Check encoding and decoding, ensuring that Monero addresses include a checksum for error detection. % https://github.com/serai-dex/serai/blob/48db06f901952b24bb38d7c7e256f798f08512cd/networks/monero/wallet/address/src/base58check.rs#L86-L106

\begin{itemize}
    \item \texttt{encode}: Encodes a byte array into Base58Check format.  
    \item \texttt{decode}: Decodes a Base58Check string into raw bytes.  
    \item \texttt{encode\_check}: Computes a checksum and encodes a byte array into Base58Check.  
    \item \texttt{decode\_check}: Decodes a Base58Check string and verifies its checksum.  
\end{itemize}

\paragraph{Address Parsing and Generation}

Monero addresses can be created from public spend and view keys, or parsed from strings. % https://github.com/serai-dex/serai/blob/48db06f901952b24bb38d7c7e256f798f08512cd/networks/monero/wallet/address/src/lib.rs#L401-L461

\begin{itemize}
    \item \texttt{MoneroAddress::new}: Creates a new Monero address given a network, address type, and key pair.  
    \item \texttt{MoneroAddress::from\_str}: Parses an address string for a given network.  
    \item \texttt{MoneroAddress::from\_str\_with\_unchecked\_network}: Parses an address string without verifying the network.  
    \item \texttt{MoneroAddress::to\_string}: Serializes an address to a Base58Check-encoded string.  
\end{itemize}

\paragraph{Address Validation}

The crate ensures that parsed addresses meet the expected format. % https://github.com/serai-dex/serai/blob/48db06f901952b24bb38d7c7e256f798f08512cd/networks/monero/wallet/address/src/lib.rs#L419-L440

\begin{itemize}
    \item \textbf{Length validation:} Ensures addresses are the correct length based on their type.  
    \item \textbf{Checksum verification:} Ensures that decoded addresses pass the Base58Check validation.  
    \item \textbf{Key validation:} Ensures that the spend and view keys are valid Edwards25519 points.  
\end{itemize}

\subsubsection{Serialization Details}

The \texttt{monero-address} crate serializes addresses into a compact binary format before encoding them into Base58Check.  The serialization structure is as follows. % https://github.com/serai-dex/serai/blob/48db06f901952b24bb38d7c7e256f798f08512cd/networks/monero/wallet/address/src/lib.rs#L381-L399

\begin{enumerate}
    \item \textbf{Network Byte:} A single byte indicating the network and address type.
    \item \textbf{Spend Key:} A 32-byte compressed Edwards25519 public key.
    \item \textbf{View Key:} A 32-byte compressed Edwards25519 public key.
    \item \textbf{Optional Fields:}
    \begin{itemize}
        \item \textbf{Integrated Payment ID:} An 8-byte payment ID (only for integrated addresses).
        \item \textbf{Feature Flags:} A variable-length integer encoding subaddress, payment ID, and guarantee status.
    \end{itemize}
    \item \textbf{Checksum:} A 4-byte Keccak-256 hash of the serialized data.
\end{enumerate}
% See https://github.com/serai-dex/serai/blob/48db06f901952b24bb38d7c7e256f798f08512cd/networks/monero/wallet/address/src/lib.rs#L381-L399 and https://github.com/serai-dex/serai/blob/48db06f901952b24bb38d7c7e256f798f08512cd/networks/monero/wallet/address/src/base58check.rs#L86-L91 for checksum details.

\subsubsection{Testing}

The \texttt{monero-address} crate includes comprehensive tests to validate the correctness of its encoding, decoding, and parsing functions. % https://github.com/serai-dex/serai/blob/48db06f901952b24bb38d7c7e256f798f08512cd/networks/monero/wallet/address/src/tests.rs

\begin{itemize}
    \item \textbf{Base58Check Encoding Tests:} Ensures that encoded addresses correctly round-trip through decoding.  
    \item \textbf{Standard Address Tests:} Validates correct parsing and serialization of standard Monero addresses.  
    \item \textbf{Integrated Address Tests:} Ensures proper handling of embedded payment IDs.  
    \item \textbf{Subaddress Tests:} Validates the ability to generate and recognize subaddresses.  
    \item \textbf{Featured Address Tests:} Ensures correct parsing and serialization of extended features.  
    \item \textbf{Vector-Based Tests:} Uses predefined address vectors to ensure consistency with expected values.  
\end{itemize}

The test suite ensures that all address types are handled correctly and provides coverage for both expected and edge cases.  


% --------------------------------------- %
\subsection{monero-clsag (v0.1.0)}
The \texttt{monero-clsag} crate implements Compact Linkable Spontaneous Anonymous Group (CLSAG) signatures and provides a FROST-inspired threshold signing mechanism.  
The implementation consists of two main components:

\subsubsection{Core CLSAG Implementation}
\begin{description}
\item[ClsagContext] \hfill  % https://github.com/serai-dex/serai/blob/48db06f901952b24bb38d7c7e256f798f08512cd/networks/monero/ringct/clsag/src/lib.rs#L65-L72

Holds the context needed for signing:
\begin{itemize}
\item A commitment opening (mask and amount)
\item Selected decoy positions and ring
\end{itemize}

\item[Clsag Signature] \hfill % https://github.com/serai-dex/serai/blob/48db06f901952b24bb38d7c7e256f798f08512cd/networks/monero/ringct/clsag/src/lib.rs#L222-L231

A signature consisting of:
\begin{itemize}
\item D: Difference of commitment randomness scaling the key image generator
\item s: Vector of responses for each ring member
\item c1: First challenge in the ring
\end{itemize}

\item[Core Algorithm] \hfill  % https://github.com/serai-dex/serai/blob/48db06f901952b24bb38d7c7e256f798f08512cd/networks/monero/ringct/clsag/src/lib.rs#L96-L220

The core signing/verification algorithm:
\begin{enumerate}
\item Takes ring members, key image I, pseudo-out P, and message m
\item For signing:
  \begin{itemize}
  \item Generates random nonces for the real signer
  \item Computes the key image generator from the public key
  \item Calculates initial commitments A and AH
  \end{itemize}
\item For both signing and verifying:
  \begin{itemize}
  \item Computes challenges using Keccak256
  \item Performs ring calculations over the Ed25519 group
  \item Validates signature components
  \end{itemize}
\end{enumerate}
\end{description}

\subsubsection{FROST-Inspired Threshold Signing}
\begin{description}
\item[ClsagMultisig] \hfill % https://github.com/serai-dex/serai/blob/48db06f901952b24bb38d7c7e256f798f08512cd/networks/monero/ringct/clsag/src/multisig.rs#L115-L137

Implements threshold signing for CLSAG:
\begin{itemize}
\item Uses FROST key generation and coordination
\item Extends CLSAG to support threshold key images
\item Preserves CLSAG's linkability property
\end{itemize}

\item[Key Components] \hfill  % See https://github.com/serai-dex/serai/blob/48db06f901952b24bb38d7c7e256f798f08512cd/networks/monero/ringct/clsag/src/multisig.rs#L57-L68 and https://github.com/serai-dex/serai/blob/48db06f901952b24bb38d7c7e256f798f08512cd/networks/monero/ringct/clsag/src/multisig.rs#L86-L97
\begin{itemize}
\item \textbf{ClsagMultisigMaskSender}: A channel for communicating commitment masks.  
\item \textbf{ClsagAddendum}: Key image shares produced during signing.  
\item \textbf{Interim}: Stores partial signature data during the protocol.  
\end{itemize}

\item[Protocol Flow] \hfill  % https://github.com/serai-dex/serai/blob/48db06f901952b24bb38d7c7e256f798f08512cd/networks/monero/ringct/clsag/src/multisig.rs#L172-L245
\begin{enumerate}
\item Initialize with a transcript and CLSAG context
\item Share and aggregate key image contributions
\item Generate and share nonces via FROST
\item Produce partial signatures
\item Verify shares and construct the final signature
\end{enumerate}

\item[Security Properties]  \hfill % https://github.com/serai-dex/serai/blob/48db06f901952b24bb38d7c7e256f798f08512cd/networks/monero/ringct/clsag/src/multisig.rs#L315-L378

\begin{itemize}
\item Maintains unforgeability of CLSAG
\item Preserves one-time key image property
\item Requires a threshold of signers to complete
\item Uses transcript-based challenge generation
\end{itemize}
\end{description}

The implementation leverages the \texttt{curve25519-dalek} library for Ed25519 operations and includes comprehensive test vectors covering both single-signer and threshold signing scenarios.  
All operations maintain constant-time properties to prevent timing side-channels.  


% -------------------------------------- %
\subsection{monero-simple-request-rpc (v0.1.0)}

\paragraph{Purpose}
The \texttt{monero-simple-request-rpc} crate provides an HTTP(S)-based transport layer for performing RPC calls to a Monero daemon.  This crate is designed to be minimal and efficient, avoiding dependencies on larger libraries like \texttt{reqwest}, while supporting both authenticated and unauthenticated connections.  The crate implements the \texttt{Rpc} trait from the \texttt{monero-rpc} crate.

\paragraph{Internal Dependencies}
\begin{itemize}
    \item \texttt{monero-rpc (v0.1.0)}: Provides the \texttt{Rpc} trait and associated error types.  
    \item \texttt{simple-request}: Handles HTTP(S) request and response processing.  
    \item \texttt{digest-auth}: Implements HTTP Digest Authentication for authenticated connections.  
    \item \texttt{tokio}: Used for asynchronous operations and synchronization primitives.  
    \item \texttt{hex}: Used for encoding and decoding hexadecimal data.
\end{itemize}

\paragraph{Structure}
The crate is a standard library crate with its entry point at \path{/rpc/simple-request/src/lib.rs}.  The main component is the \texttt{SimpleRequestRpc} struct, which encapsulates the connection logic, authentication handling, and request processing.  The implementation includes:
\begin{itemize}
    \item \textbf{Authentication Handling}: Supports both authenticated and unauthenticated RPC connections.  Authentication uses HTTP Digest Authentication when credentials are provided in the URL. % See Authentication enum at https://github.com/serai-dex/serai/blob/48db06f901952b24bb38d7c7e256f798f08512cd/networks/monero/rpc/simple-request/src/lib.rs#L20-L33
    \item \textbf{Thread Safety}: Uses \texttt{Arc<Mutex>} for safe concurrent access to authentication state and nonce management. % See Arc<Mutex> usage at https://github.com/serai-dex/serai/blob/48db06f901952b24bb38d7c7e256f798f08512cd/networks/monero/rpc/simple-request/src/lib.rs#L27-L32
    \item \textbf{Request Processing}: Sends RPC requests and processes responses, including retries for stale authentication challenges. % See inner_post in https://github.com/serai-dex/serai/blob/48db06f901952b24bb38d7c7e256f798f08512cd/networks/monero/rpc/simple-request/src/lib.rs#L130-L280
    \item \textbf{Timeout Management}: Allows for customizable request timeouts, defaulting to 30 seconds. % See declaration at https://github.com/serai-dex/serai/blob/48db06f901952b24bb38d7c7e256f798f08512cd/networks/monero/rpc/simple-request/src/lib.rs#L18 and usage in https://github.com/serai-dex/serai/blob/48db06f901952b24bb38d7c7e256f798f08512cd/networks/monero/rpc/simple-request/src/lib.rs#L62-L77 and after https://github.com/serai-dex/serai/blob/48db06f901952b24bb38d7c7e256f798f08512cd/networks/monero/rpc/simple-request/src/lib.rs#L129
\end{itemize}
% See Struct at https://github.com/serai-dex/serai/blob/48db06f901952b24bb38d7c7e256f798f08512cd/networks/monero/rpc/simple-request/src/lib.rs#L35-L43 and implementation at https://github.com/serai-dex/serai/blob/48db06f901952b24bb38d7c7e256f798f08512cd/networks/monero/rpc/simple-request/src/lib.rs#L45-L127

\paragraph{Detailed Functionality}
\begin{description}
    \item[\texttt{SimpleRequestRpc::new}]  
    Creates a new \texttt{SimpleRequestRpc} instance with a default timeout of 30 seconds.  If the provided URL contains credentials, they are parsed and used for Digest Authentication. % https://github.com/serai-dex/serai/blob/48db06f901952b24bb38d7c7e256f798f08512cd/networks/monero/rpc/simple-request/src/lib.rs#L62-L68

    \item[\texttt{SimpleRequestRpc::with\_custom\_timeout}]  
    Similar to \texttt{new}, but allows specifying a custom timeout duration.  It parses the URL for authentication details and initializes the appropriate client. % https://github.com/serai-dex/serai/blob/48db06f901952b24bb38d7c7e256f798f08512cd/networks/monero/rpc/simple-request/src/lib.rs#L70-L126

    \item[\texttt{SimpleRequestRpc::post}]  
    Implements the \texttt{post} method from the \texttt{Rpc} trait.  It sends a POST request to the specified RPC route with the provided payload, handling retries for authentication failures due to stale challenges. % https://github.com/serai-dex/serai/blob/48db06f901952b24bb38d7c7e256f798f08512cd/networks/monero/rpc/simple-request/src/lib.rs#L283-L293

    \item[\texttt{SimpleRequestRpc::inner\_post}]  
    An internal method that performs the actual request logic.  It handles both authenticated and unauthenticated requests and processes the response to extract the body or detect errors. % https://github.com/serai-dex/serai/blob/48db06f901952b24bb38d7c7e256f798f08512cd/networks/monero/rpc/simple-request/src/lib.rs#L130-L135

    \item[\texttt{SimpleRequestRpc::digest\_auth\_challenge}]  
    Extracts and parses the \texttt{WWW-Authenticate} header from the server response to initialize or update the Digest Authentication state. % https://github.com/serai-dex/serai/blob/48db06f901952b24bb38d7c7e256f798f08512cd/networks/monero/rpc/simple-request/src/lib.rs#L46-L60

    \item[\texttt{Authentication Retry Logic}]  
    Implements a two-attempt retry mechanism for authentication failures, with specific handling for stale nonces and connection errors.  This retry logic is important for maintaining connection stability during authentication state changes. % https://github.com/serai-dex/serai/blob/48db06f901952b24bb38d7c7e256f798f08512cd/networks/monero/rpc/simple-request/src/lib.rs#L253-L275
\end{description}

\paragraph{Authentication State Management}
The authentication state is managed through an internal \texttt{Authentication} enum with two variants:
\begin{itemize}
    \item \texttt{Unauthenticated}: Contains a single \texttt{Client} instance for all requests.  
    \item \texttt{Authenticated}: Contains a username, password, and a thread-safe connection state managed through \texttt{Arc<Mutex<(Option<(WwwAuthenticateHeader, u64)>, Client)>>}.
\end{itemize}

\paragraph{Testing}
Integration tests are located in \path{/rpc/simple-request/tests/tests.rs}.  These tests validate:
\begin{itemize}
    \item RPC functionality for retrieving blockchain height, blocks, and hardfork versions.  
    \item Block generation with different amounts of blocks (1 and 5).  
    \item Authentication handling for valid and stale challenges.  
    \item Decoy RPC functionality and output distribution queries.  
    \item Sequential test execution using \texttt{LazyLock<Mutex<()>>} to prevent test interference.  
    \item Error handling for various failure scenarios.
\end{itemize}
% https://github.com/serai-dex/serai/blob/48db06f901952b24bb38d7c7e256f798f08512cd/networks/monero/rpc/simple-request/tests/tests.rs#L16-L65

\paragraph{Error Handling}
The implementation provides comprehensive error handling for:
\begin{itemize}
    \item Connection failures with detailed error messages.  
    \item Authentication failures, including stale nonce detection.  
    \item Invalid or malformed responses.  
    \item Timeout errors with a configurable duration.  
    \item URL parsing and validation errors.
\end{itemize}

\paragraph{Summary}
The \texttt{monero-simple-request-rpc} crate is a lightweight, efficient solution for connecting to a Monero daemon via HTTP(S).  It adheres to the \texttt{Rpc} trait interface, enabling seamless integration with the broader \texttt{monero-rpc} ecosystem.  Its design prioritizes thread safety, robust authentication handling, and comprehensive error management, making it suitable for both development and production use cases.



% --------------------------------------------- %
\subsection{\texttt{monero-borromean (v0.1.0)}}
\label{sec:monero_borromean}

This crate provides a Borromean ring signature--based approach to a 64-bit range proof within the Monero protocol. It defines two primary types:

\begin{itemize}
  \item \texttt{BorromeanSignatures} % https://github.com/serai-dex/serai/blob/48db06f901952b24bb38d7c7e256f798f08512cd/networks/monero/ringct/borromean/src/lib.rs#L18-L28
  \item \texttt{BorromeanRange} % https://github.com/serai-dex/serai/blob/48db06f901952b24bb38d7c7e256f798f08512cd/networks/monero/ringct/borromean/src/lib.rs#L74-L79
\end{itemize}

Both types are specialized to operate with 64-bit range proofs and rely on Curve25519-based group operations to construct and verify Borromean ring signatures.

\paragraph{Data Structures.}
\begin{itemize}
  \item \texttt{BorromeanSignatures}
  \begin{itemize}
    \item Stores exactly 64 Borromean ring signatures in two arrays of scalars, \texttt{s0} and \texttt{s1}, each of length 64.
    \item Holds the final challenge scalar \texttt{ee}.
    \item Implements custom \texttt{read} and \texttt{write} methods using the \texttt{monero-io} crate to produce or consume canonical byte encodings. % https://github.com/serai-dex/serai/blob/48db06f901952b24bb38d7c7e256f798f08512cd/networks/monero/ringct/borromean/src/lib.rs#L31-L49
    \item Its \texttt{verify} function checks correctness by: % https://github.com/serai-dex/serai/blob/48db06f901952b24bb38d7c7e256f798f08512cd/networks/monero/ringct/borromean/src/lib.rs#L51-L72
      \begin{enumerate}[label=(\alph*)]
        \item Performing an iterative double-scalar multiplication on each signature component. % See section referenced above, especially https://github.com/serai-dex/serai/blob/48db06f901952b24bb38d7c7e256f798f08512cd/networks/monero/ringct/borromean/src/lib.rs#L56-L60 and https://github.com/serai-dex/serai/blob/48db06f901952b24bb38d7c7e256f798f08512cd/networks/monero/ringct/borromean/src/lib.rs#L62-L66
        \item Accumulating hash outputs in a transcript. % https://github.com/serai-dex/serai/blob/48db06f901952b24bb38d7c7e256f798f08512cd/networks/monero/ringct/borromean/src/lib.rs#L67
        \item Ensuring the final hash matches \texttt{ee}. % https://github.com/serai-dex/serai/blob/48db06f901952b24bb38d7c7e256f798f08512cd/networks/monero/ringct/borromean/src/lib.rs#L70
      \end{enumerate}
  \end{itemize}

  \item \texttt{BorromeanRange} % https://github.com/serai-dex/serai/blob/48db06f901952b24bb38d7c7e256f798f08512cd/networks/monero/ringct/borromean/src/lib.rs#L74-L79
  \begin{itemize}
    \item Contains a \texttt{BorromeanSignatures} field and 64 \texttt{bit\_commitments}, each an \texttt{EdwardsPoint}.
    \item Like \texttt{BorromeanSignatures}, it has \texttt{read} and \texttt{write} functions for serialized I/O. % https://github.com/serai-dex/serai/blob/48db06f901952b24bb38d7c7e256f798f08512cd/networks/monero/ringct/borromean/src/lib.rs#L82-L94
    \item Its \texttt{verify} method checks that: % https://github.com/serai-dex/serai/blob/48db06f901952b24bb38d7c7e256f798f08512cd/networks/monero/ringct/borromean/src/lib.rs#L96-L111
      \begin{enumerate}[label=(\alph*)]
        \item The sum of all \texttt{bit\_commitments} equals the provided \texttt{commitment}. % https://github.com/serai-dex/serai/blob/48db06f901952b24bb38d7c7e256f798f08512cd/networks/monero/ringct/borromean/src/lib.rs#L99-L101
        \item For each bit, subtracting \(\texttt{H\_pow\_2}[i]\) from \texttt{bit\_commitments}[i] produces a second set of points. % https://github.com/serai-dex/serai/blob/48db06f901952b24bb38d7c7e256f798f08512cd/networks/monero/ringct/borromean/src/lib.rs#L103-L108
        \item The embedded \texttt{BorromeanSignatures} correctly verifies under those two sets of points. % https://github.com/serai-dex/serai/blob/48db06f901952b24bb38d7c7e256f798f08512cd/networks/monero/ringct/borromean/src/lib.rs#L110
      \end{enumerate}
  \end{itemize}
\end{itemize}

\paragraph{Purpose and Scope.} 
This crate provides an older, yet still relevant, Monero-based range proof system.  While newer systems like Bulletproofs have largely supplanted Borromean proofs in many contexts, the \texttt{monero-borromean} crate is crucial for maintaining backwards compatibility with legacy proofs and for verifying archived transactions that used Borromean range proofs. It accepts both \texttt{std} and \texttt{no\_std} environments, and depends on a few of the core Monero crates:
\begin{itemize}
  \item \texttt{monero-io} for reading and writing typed data.
  \item \texttt{monero-generators} for fixed-base scalar multiplication with \(H^2\).
  \item \texttt{monero-primitives} for cryptographic primitives such as \texttt{UnreducedScalar}.
\end{itemize}

\paragraph{Verification Logic.}
\begin{enumerate}[label=(\roman*)]
  \item \texttt{BorromeanSignatures::verify} loops over 64 “bits” (i.e., the separate ring signatures). For each bit: % https://github.com/serai-dex/serai/blob/48db06f901952b24bb38d7c7e256f798f08512cd/networks/monero/ringct/borromean/src/lib.rs#L51-L72
  \begin{itemize}
    \item It checks the challenge scalar by computing intermediate elliptic-curve multiplications. % https://github.com/serai-dex/serai/blob/48db06f901952b24bb38d7c7e256f798f08512cd/networks/monero/ringct/borromean/src/lib.rs#L56-L59
    \item It accumulates partial results into a 2048-byte transcript. Z% https://github.com/serai-dex/serai/blob/48db06f901952b24bb38d7c7e256f798f08512cd/networks/monero/ringct/borromean/src/lib.rs#L67
  \end{itemize}
  \item Finally, if the scalar derived from hashing the transcript matches the stored scalar \texttt{ee}, the signature is valid % https://github.com/serai-dex/serai/blob/48db06f901952b24bb38d7c7e256f798f08512cd/networks/monero/ringct/borromean/src/lib.rs#L70
\end{enumerate}

\paragraph{Conclusion.} 
The \texttt{monero-borromean (v0.1.0)} crate implements older-style Borromean range proofs used historically in Monero transactions.  It is straightforward, cleanly implemented, line 3 enforces \texttt{missing\_docs} denial to encourage thorough documentation and offers \texttt{no\_std} support.  Its modular APIs for reading, writing, and verifying signatures make it an easy integration point for other projects that might require Borromean signature functionality or need to validate historical transactions.




% --------------------------------------- %
\subsection{monero-mlsag (v0.1.0)}

The \texttt{monero-mlsag} crate provides functionality for Multilayered Linkable Spontaneous Anonymous Group (MLSAG) signatures as used in the Monero protocol.  
The crate is organized around two primary data structures: \texttt{RingMatrix} and \texttt{Mlsag}.

\subsubsection{Ring Matrix}

\begin{description}
\item[Structure] \hfill 

The \texttt{RingMatrix} type encapsulates a matrix of Edwards points used for MLSAG verification:  
\begin{itemize}
\item Internal representation: \texttt{Vec<Vec<EdwardsPoint>>}.  
\item Zeroizes on drop for security.  
\item Must contain at least 2 ring members.  
\item All members must have equal length.  
\end{itemize}
% https://github.com/serai-dex/serai/blob/48db06f901952b24bb38d7c7e256f798f08512cd/networks/monero/ringct/mlsag/src/lib.rs#L42-L46

\item[Construction Methods] \hfill 
\begin{enumerate}
\item \texttt{new}: Creates a ring matrix from a pre-formatted vector of vectors.  
  \begin{itemize}
  \item Validates matrix dimensions.  
  \item Ensures minimum ring size of 2.  
  \item Ensures consistent member lengths.  
  \end{itemize}
% https://github.com/serai-dex/serai/blob/48db06f901952b24bb38d7c7e256f798f08512cd/networks/monero/ringct/mlsag/src/lib.rs#L49-L64

\item \texttt{individual}: Constructs a ring matrix for single output verification.  
  \begin{itemize}
  \item Takes a ring of \texttt{[EdwardsPoint; 2]} arrays.  
  \item Takes a pseudo-output point.  
  \item Subtracts the pseudo-output from the second column.  
  \end{itemize}
\end{enumerate}
% https://github.com/serai-dex/serai/blob/48db06f901952b24bb38d7c7e256f798f08512cd/networks/monero/ringct/mlsag/src/lib.rs#L66-L76

\item[Utilities] \hfill 
\begin{itemize}
\item \texttt{members()}: Returns the count of ring members. % https://github.com/serai-dex/serai/blob/48db06f901952b24bb38d7c7e256f798f08512cd/networks/monero/ringct/mlsag/src/lib.rs#L83-L86
\item \texttt{member\_len()}: Returns the length of each member vector. % https://github.com/serai-dex/serai/blob/48db06f901952b24bb38d7c7e256f798f08512cd/networks/monero/ringct/mlsag/src/lib.rs#L83-L86
\item \texttt{iter()}: Provides an iterator over matrix members as slices. % https://github.com/serai-dex/serai/blob/48db06f901952b24bb38d7c7e256f798f08512cd/networks/monero/ringct/mlsag/src/lib.rs#L78-L81
\end{itemize}
\end{description}

\subsubsection{MLSAG Signature}

\begin{description}
\item[Structure] \hfill 

The \texttt{Mlsag} type represents a complete MLSAG signature:
\begin{itemize}
\item \texttt{ss}: A matrix of response scalars (\texttt{Vec<Vec<Scalar>>}).  
\item \texttt{cc}: A challenge scalar.  
\item Implements zeroization for security.  
\end{itemize}
% https://github.com/serai-dex/serai/blob/48db06f901952b24bb38d7c7e256f798f08512cd/networks/monero/ringct/mlsag/src/lib.rs#L98-L103

\item[Serialization] \hfill

Provides binary serialization methods.  
\begin{itemize}
\item \texttt{write}: Serializes to a writer. % https://github.com/serai-dex/serai/blob/48db06f901952b24bb38d7c7e256f798f08512cd/networks/monero/ringct/mlsag/src/lib.rs#L106-L112
  \begin{itemize}
  \item Writes the \texttt{ss} matrix elements.  
  \item Writes the \texttt{cc} challenge.  
  \end{itemize}
\item \texttt{read}: Deserializes from a reader. % https://github.com/serai-dex/serai/blob/48db06f901952b24bb38d7c7e256f798f08512cd/networks/monero/ringct/mlsag/src/lib.rs#L114-L122
  \begin{itemize}
  \item Takes the expected mixin count.  
  \item Takes the expected width of the \texttt{ss} matrix.  
  \item Reconstructs the signature structure.  
  \end{itemize}
\end{itemize}

\item[Verification] \hfill 

The \texttt{verify} method validates an MLSAG signature.  
\begin{enumerate}
\item Input validation.  
  \begin{itemize}
  \item Validates that the key image count matches the ring member length minus 1.  
  \item Ensures consistent matrix dimensions.  
  \item Validates key image properties (non-identity, torsion-free).  
  \end{itemize}
% https://github.com/serai-dex/serai/blob/48db06f901952b24bb38d7c7e256f798f08512cd/networks/monero/ringct/mlsag/src/lib.rs#L124-L186

\item Challenge reconstruction.  
  \begin{itemize}
  \item Maintains a message buffer for hash computation. % https://github.com/serai-dex/serai/blob/48db06f901952b24bb38d7c7e256f798f08512cd/networks/monero/ringct/mlsag/src/lib.rs#L137
  \item Iterates through ring members and key images. % https://github.com/serai-dex/serai/blob/48db06f901952b24bb38d7c7e256f798f08512cd/networks/monero/ringct/mlsag/src/lib.rs#L150-L179
  \item Computes \(L = sG + (c_i \times P)\) for each entry. % https://github.com/serai-dex/serai/blob/48db06f901952b24bb38d7c7e256f798f08512cd/networks/monero/ringct/mlsag/src/lib.rs#L157
  \item For linkable layers, computes \(R = s \times \mathrm{Hp}(P) + (c_i \times I)\). % https://github.com/serai-dex/serai/blob/48db06f901952b24bb38d7c7e256f798f08512cd/networks/monero/ringct/mlsag/src/lib.rs#L171
  \item Updates the challenge using Keccak256. % https://github.com/serai-dex/serai/blob/48db06f901952b24bb38d7c7e256f798f08512cd/networks/monero/ringct/mlsag/src/lib.rs#L176
  \end{itemize}

\item Final verification.  
  \begin{itemize}
  \item Checks that the reconstructed challenge matches the signature. % https://github.com/serai-dex/serai/blob/48db06f901952b24bb38d7c7e256f798f08512cd/networks/monero/ringct/mlsag/src/lib.rs#L181-L184
  \item Returns a \texttt{Result} indicating validity.  
  \end{itemize}
\end{enumerate}
\end{description}

\subsubsection{Aggregate Ring Matrix Builder}

\begin{description}
\item[Purpose] \hfill

The \texttt{AggregateRingMatrixBuilder} facilitates the construction of ring matrices for aggregate signatures.  
\begin{itemize}
\item Manages key ring vectors.  
\item Tracks amount commitments.  
\item Handles pseudo-output calculations.  
\end{itemize}
% https://github.com/serai-dex/serai/blob/48db06f901952b24bb38d7c7e256f798f08512cd/networks/monero/ringct/mlsag/src/lib.rs#L188-L196

\item[Construction] \hfill 

Created with transaction outputs and a fee. % https://github.com/serai-dex/serai/blob/48db06f901952b24bb38d7c7e256f798f08512cd/networks/monero/ringct/mlsag/src/lib.rs#L199-L208
\begin{itemize}
\item Takes a slice of output commitment points.  
\item Takes the fee amount as a \texttt{u64}.  
\item Computes the initial sum of outputs. % https://github.com/serai-dex/serai/blob/48db06f901952b24bb38d7c7e256f798f08512cd/networks/monero/ringct/mlsag/src/lib.rs#L206
\end{itemize}

\item[Ring Addition] \hfill 

The \texttt{push\_ring} method builds the matrix incrementally. % https://github.com/serai-dex/serai/blob/48db06f901952b24bb38d7c7e256f798f08512cd/networks/monero/ringct/mlsag/src/lib.rs#L210-L229
\begin{itemize}
\item Validates ring dimensions.  
\item Separates key and amount components.  
\item Updates running sums.  
\end{itemize}

\item[Finalization] \hfill 

The \texttt{build} method produces the final \texttt{RingMatrix}. % https://github.com/serai-dex/serai/blob/48db06f901952b24bb38d7c7e256f798f08512cd/networks/monero/ringct/mlsag/src/lib.rs#L231-L237
\begin{itemize}
\item Combines key and amount components.  
\item Validates the final matrix structure.  
\item Returns the complete ring matrix.  
\end{itemize}
\end{description}

\subsubsection{Error Handling}

The crate defines the \texttt{MlsagError} enum for various failure modes.  
\begin{itemize}
\item \texttt{InvalidRing}: Ring size or structure issues.  
\item \texttt{InvalidAmountOfKeyImages}: Incorrect key image count.  
\item \texttt{InvalidSs}: Response matrix dimension mismatch.  
\item \texttt{InvalidKeyImage}: Invalid key image properties.  
\item \texttt{InvalidCi}: Challenge verification failure.  
\end{itemize}
% https://github.com/serai-dex/serai/blob/48db06f901952b24bb38d7c7e256f798f08512cd/networks/monero/ringct/mlsag/src/lib.rs#L21-L40


































\bibliographystyle{amsplain}
\bibliography{sources.bib}




\end{document}
