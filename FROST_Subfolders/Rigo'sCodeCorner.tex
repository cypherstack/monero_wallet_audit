\section{Rust Implementation}

In this section, we review the Rust implementation of FROSTLASS \path{/serai-dex/serai/networks/monero/} directory (the ``audit directory'') in commit \texttt{48db06f} by \texttt{KayabaNerve} of the GitHub repository at \url{github.com/serai-dex/serai}

\subsection{Overview}

The audit directory contains a Rust implementation of Monero wallet functionality, together with a new approach to using FROST for threshold signing. All files, folders, and subfolders of 
\path{/serai-dex/serai/networks/monero/} are in scope for the audit, except not the subfolder \path{/serai-dex/serai/networks/monero/verify-chain/}.

\subsection{General Findings}

This codebase is consistent, neat, clean, and efficient.

\subsection{Repository Structure}

The repository at \url{github.com/serai-dex/serai} is organized into several top-level directories, including \path{/serai-dex/serai/networks/}, which in turn contains \path{/serai-dex/serai/networks/monero/}.  Figure \ref{fig:directory_structure} describes the target directory structure, where asterisks indicate crates (named in parentheses); the only folder not in scope of this audit is \texttt{verify-chain}. 


\begin{figure}[ht]
\centering
\begin{minipage}{0.9\linewidth}
\begin{mdframed}
\begin{verbatim}
/serai-dex/serai/
|-- out_of_scope_files
|-- networks/
    |-- out_of_scope_files
    |-- monero*/ (monero-serai)
        |-- generators*/ (monero-generators)
        |-- io*/ (monero-io)
        |-- primitives*/ (monero-primitives)
        |-- ringct/
            |-- borromean*/ (monero-borromean)
            |-- bulletproofs*/ (monero-bulletproofs)
            |-- clsag*/ (monero-clsag)
            |-- mlsag*/ (monero-mlsag)
        |-- rpc*/ (monero-rpc)
            |-- simple-request*/ (monero-simple-request-rpc)
        |-- src/
        |-- tests/
        |-- verify-chain*/
        |-- wallet*/ (monero-wallet)
            |-- address*/ (monero-address)
\end{verbatim}
\end{mdframed}
\end{minipage}
\caption{Directory structure of the target repository. Everything within \protect\path{/serai-dex/serai/networks/monero} is in scope, except the subfolder \protect\path{/serai-dex/serai/networks/monero/verify-chain/}. Asterisks indicate crates. Crate names are included in parantheses.}
\label{fig:directory_structure}
\end{figure}

% \subsection{Crate Dependencies}

In the following, we omit \texttt{/serai-dex/serai/networks/monero/} from directory references for clarity, with the understanding that all directory references use this as a prefix.

The crates are evocatively named by the data they handle, and mutually depend on each other. Crate dependency is not isomorphic to the file structure. 
Figure \ref{fig:transitive_reduced_dependency_graph} displays these dependencies in the transitive reduction of the directed graph of crate dependencies. Readers should be aware that the transitive reduction of a directed graph $G$ is constructed by removing as many edges as possible without changing the \textit{reachability relation} on pairs of vertices. Thus, not all edges corresponding to direct crate dependency are displayed in Figure \ref{fig:transitive_reduced_dependency_graph}. For example, \texttt{monero-serai} depends directly on \texttt{monero-generators}, \texttt{monero-io}, and \texttt{monero-primitives}.
In Section \ref{sec:crate_summaries}, we present a look at each crate. 

\begin{figure}[ht]
\centering
\begin{tikzpicture}[
    scale=1, % Ensure proper scaling
    every node/.style={draw, rectangle, rounded corners, minimum height=1cm, font=\ttfamily, inner sep=5pt, align=center}, % Rectangular nodes
    edge/.style={-{Latex}, thick}
]

% Manually normalized coordinates for a 10x7 aspect ratio
\node (wallet) at (4, 8.5) {wallet};
\node (simple-request) at (8, 8.5) {simple-request};
\node (rpc) at (8.25, 6.5) {rpc};
\node (serai) at (2.5, 5.5) {serai};
\node (address) at (10, 3.5) {address};
\node (mlsag) at (8, 3.5) {mlsag};
\node (clsag) at (6, 3.5) {clsag};
\node (bulletproofs) at (3.5, 3.5) {bulletproofs};
\node (borromean) at (0.5, 3.5) {borromean};
\node (io) at (5, 1) {io};
\node (generators) at (2, 1) {generators};
\node (primitives) at (8, 1) {primitives};

% Edges
\draw[edge] (wallet) -- (rpc);
\draw[edge] (wallet) -- (address);
\draw[edge] (wallet) -- (clsag);
\draw[edge] (wallet) -- (serai);
\draw[edge] (simple-request) -- (rpc);
\draw[edge] (rpc) -- (serai);
\draw[edge] (rpc) -- (address);
\draw[edge] (serai) -- (borromean);
\draw[edge] (serai) -- (bulletproofs);
\draw[edge] (serai) -- (clsag);
\draw[edge] (serai) -- (mlsag);
\draw[edge] (address) -- (io);
\draw[edge] (address) -- (primitives);
\draw[edge] (borromean) -- (io);
\draw[edge] (borromean) -- (generators);
\draw[edge] (borromean) -- (primitives);
\draw[edge] (bulletproofs) -- (io);
\draw[edge] (bulletproofs) -- (generators);
\draw[edge] (bulletproofs) -- (primitives);
\draw[edge] (clsag) -- (io);
\draw[edge] (clsag) -- (generators);
\draw[edge] (clsag) -- (primitives);
\draw[edge] (mlsag) -- (io);
\draw[edge] (mlsag) -- (generators);
\draw[edge] (mlsag) -- (primitives);

\end{tikzpicture}
\caption{The transitive reduction of the graph of crates and their dependencies. Not all edges corresponding to a direct dependency are displayed in a transitive reduction. For example, \texttt{monero-serai} depends directly on \texttt{monero-generators}, \texttt{monero-io}, and \texttt{monero-primitives}, but these edges are not explicitly displayed.}
\label{fig:transitive_reduced_dependency_graph}
\end{figure}


\subsection{Functionality}

\subsubsection{Crate Details}\label{sec:crate_summaries}

In this section, we describe each crate name, version, purpose, internal dependencies, and a brief description of crate structure. We use a breadth-first, top-down approach following Figure \ref{fig:transitive_reduced_dependency_graph}. Elements of public APIs (i.e.\ with the modifier \texttt{pub}) have names which are decorated \texttt{thusly}\textsuperscript{\textdagger}, and elements exposed at the crate level (i.e.\ with the modifier \texttt{pub(crate)}) have names which are decorated \texttt{thusly}\textsuperscript{$\Delta$}. We provide links throughout the document to corresponding glossary entries.

% This is the beginning of the giant itemize section in Josh's documentation
% It's a little tricky because so many pieces are itemized that things seem to deeply nested

\begin{itemize} 

 \item \gls{monero-wallet (v0.1.0)}
 \begin{itemize}
 \item Purpose: Handle all wallet functionality.
 \item Internal Dependencies: 
 \begin{itemize}
 \item \gls{monero-address (v0.1.0)}\textsuperscript{\textdagger}
 \item \gls{monero-clsag (v0.1.0)}\textsuperscript{\textdagger}
 \item \gls{monero-rpc (v0.1.0)}\textsuperscript{\textdagger}, 
 \item \gls{monero-serai (v0.1.4-alpha)}\textsuperscript{\textdagger}
 \end{itemize}
 \item Structure: A standard library crate, with the corresponding entry point at  \gls{wallet-entry-point}. 
 
 \item Tests at \gls{wallet-tests}.
 \item The \gls{monero-wallet (v0.1.0)} crate employs the following modules.
\begin{itemize}
\item \gls{decoys-module} handles decoy selection with a publicly exposed struct \texttt{OutputWithDecoys}.
\item \gls{extra-module}\textsuperscript{\textdagger} handles the \texttt{extra} field of a transaction.
\item \gls{output-module}\textsuperscript{$\Delta$} handles transaction outputs.
\item \gls{scan-module} handles transaction scanning.
\item \gls{send-module}\textsuperscript{\textdagger} handles sending transactions. This is a directory module, and contains the following file modules:
\begin{itemize}
\item \gls{eventuality-module} handles \gls{eventualities}.
\item \gls{send-multisig-module} handles sending threshold transactions.
\item \gls{send-tx-module} handles sending transactions.
\item \gls{send-tx-keys-module} handles keys for sending transactions.
\end{itemize}
\item \gls{view-pair-module} handles pairs of keys, where one is a public spend key, and the other is a private view key.
\end{itemize}

% Rigo added this
\end{itemize}


 \item \gls{monero-simple-request-rpc (v0.1.0)}
 \begin{itemize}
 \item Purpose: Default RPC to avoid external dependencies on, e.g.\ reqwest. Only used in dev dependencies.
 \item Internal Dependencies: 
 \begin{itemize}
 \item \gls{monero-rpc (v0.1.0)}
 \end{itemize}
 \item Structure: A standard library crate,  with the corresponding entry point at  \gls{monero-simple-request-rpc-entry-point}.
\end{itemize}
 
 
 \item \gls{monero-rpc (v0.1.0)}
 \begin{itemize}
 \item Purpose: handle RPC calls for interacting on the Monero network.
 \item Internal Dependencies: 
 \begin{itemize}
 \item \gls{monero-address (v0.1.0)}  
 \item \gls{monero-serai (v0.1.4-alpha)}
 \end{itemize}
 \item Structure: A standard library crate, with the corresponding entry point at \gls{monero-rpc-entry-point}, employing no file or directory modules.
 \end{itemize}
 
\item \gls{monero-serai (v0.1.4-alpha)} 
\begin{itemize}
\item Purpose: the overall transaction library.
\item Internal dependencies: 
\begin{itemize}
\item \gls{monero-borromean (v0.1.0)}
\item \gls{monero-bulletproofs (v0.1.0)}
\item \gls{monero-clsag (v0.1.0)} 
\item \gls{monero-generators (v0.4.0)}\textsuperscript{\textdagger}
\item \gls{monero-io (v0.1.0)}\textsuperscript{\textdagger}
\item \gls{monero-mlsag (v0.1.0)}
\item \gls{monero-primitives (v0.1.0)}\textsuperscript{\textdagger}
\end{itemize}
\item Structure: A standard library crate, with the corresponding entry point at \gls{monero-serai-entry-point}.
\begin{itemize}
\item File modules:
\begin{itemize}
\item \gls{block-module}\textsuperscript{\textdagger}
\item \gls{merkle-module}
\item \gls{ring-signatures-module}\textsuperscript{\textdagger}
\item \gls{ringct-module}\textsuperscript{\textdagger}
\item \gls{transaction-module}\textsuperscript{\textdagger}
\end{itemize}
\item Tests at \gls{monero-serai-tests}
\end{itemize}

% Rigo added this
\end{itemize}

\item \gls{monero-address (v0.1.0)}
\begin{itemize}
\item Purpose: handles Monero addresses.
\item Internal dependencies: 
\begin{itemize}
\item \gls{monero-io (v0.1.0)} 
\item \gls{monero-primitives (v0.1.0)}
\end{itemize}
\item Structure: A standard library crate, with the corresponding entry point at \gls{monero-address-entry-point}.
\begin{itemize}
\item File module: \gls{base58-module}.
\item Tests at \gls{address-tests}.
\end{itemize}
\end{itemize}



   \item \gls{monero-borromean (v0.1.0)}
   \begin{itemize}
   \item Purpose: Handles Borromean signatures and Borromean range proofs.
   \item Internal dependencies:
   \begin{itemize}
   \item \gls{monero-generators (v0.4.0)}
   \item \gls{monero-io (v0.1.0)}
   \item \gls{monero-primitives (v0.1.0)}
   \end{itemize}
   \item Structure: A standard library crate, with the corresponding entry point at \gls{borromean-entry-point}. Employs no modules, and untested.
   \end{itemize}
   
   \item \gls{monero-bulletproofs (v0.1.0)}
   \begin{itemize}
   \item Purpose: Handles original bulletproofs and bulletproofs plus.
   \item Internal dependencies:
   \begin{itemize}
   \item \gls{monero-generators (v0.4.0)}
   \item \gls{monero-io (v0.1.0)}
   \item \gls{monero-primitives (v0.1.0)}
   \end{itemize}
   \item Structure: A standard library crate, with the corresponding entry point at \gls{bulletproofs-entry-point}.
   \begin{itemize}
   \item File modules:
   \begin{itemize}
   \item \gls{bp-batch-verifier-module}\textsuperscript{$\Delta$}
   \item \gls{bp-core-module}\textsuperscript{$\Delta$}
   \item \gls{bp-point-vector-module}\textsuperscript{$\Delta$}
   \item \gls{bp-scalar-vector-module}\textsuperscript{$\Delta$}
   \end{itemize}
   \item Directory modules:
   \begin{itemize}
   \item \gls{bp-original-module}\textsuperscript{$\Delta$}
   \item \gls{bp-plus-module}\textsuperscript{$\Delta$}
   \end{itemize}
   \item Tests at \gls{bp-test-module}.
   \end{itemize}
   \end{itemize}


   
   \item \gls{monero-clsag (v0.1.0)}
   \begin{itemize}
   \item Purpose: Handles CLSAG ring signatures and a FROST-like thresholdization.
   \item Internal dependencies:
   \begin{itemize}
   \item \gls{monero-generators (v0.4.0)}
   \item \gls{monero-io (v0.1.0)}
   \item \gls{monero-primitives (v0.1.0)}
   \end{itemize}
   \item Structure: A standard library crate, with the corresponding entry point at \gls{monero-clsag-entry-point}.
   \begin{itemize}
   \item File module: \gls{clsag-multisig-module}.
   \item Tests: \gls{clsag-tests}
   \end{itemize}
   \end{itemize}
   
   \item \gls{monero-mlsag (v0.1.0)}
   \begin{itemize}
   \item Purpose: Handles MLSAG ring signatures.
   \item Internal dependencies:
   \begin{itemize}
   \item \gls{monero-generators (v0.4.0)}
   \item \gls{monero-io (v0.1.0)}
   \item \gls{monero-primitives (v0.1.0)}
   \end{itemize}
   \item Structure: A standard library crate with entry point at
   
   \gls{monero-mlsag-entry-point}. Employs no modules, and untested.
   \end{itemize}


   
 \item \gls{monero-primitives (v0.1.0)}
   \begin{itemize}
   \item Purpose: Handles Pedersen Commitments and Decoys.
   \item Internal dependencies:
   \begin{itemize}
   \item  \gls{monero-io (v0.1.0)} 
   \item \gls{monero-generators (v0.4.0)}
   \end{itemize}
   \item Structure: A standard library crate with entry point at \gls{monero-primitives-entry-point}. 
   \begin{itemize}
   
   \item File module at \gls{unreduced-scalar-module}.
   \item Tests at \gls{monero-primitives-tests}.
   \end{itemize}
   \end{itemize}


 
\item \gls{monero-generators (v0.4.0)}
   \begin{itemize}
   \item Purpose: Handles hashing data to elliptic curve group elements and all fixed generators used in Monero protocol computations.
   \item Internally dependent only on \gls{monero-io (v0.1.0)}.
   \item Structure: A standard library crate with entry point at \gls{monero-generators-entry-point}. 
   \begin{itemize}
   
   \item File module at \gls{hash-to-point-module}. 
   \item Tests at \gls{monero-generators-tests}
   \end{itemize}
   \end{itemize}
   
\item \gls{monero-io (v0.1.0)} 
   \begin{itemize}
   \item Purpose: Handles reading and writing various data structures used in Monero protocol computations (e.g.\ bytes, scalars, group elements, lists whose entries are the same type).
   \item No internal dependencies.
   \item Structure: A standard library crate with entry point at \gls{monero-io-entry-point}. Employs neither modules nor tests.   
\end{itemize}

% Rigo added this
\end{itemize}


% -------------------------------------------------- %
% Below is the function section of Josh's documentation
% -------------------------------------------------- %

% -------------------------------------------------- %
\subsection{monero-io (v0.1.0)}
The \gls{monero-io (v0.1.0)} crate implements canonical serialization and deserialization operations.  The following details the supported operations.  

\begin{description}
\item[VarInt Encoding] The variable-length integer encoding employs a continuation bit scheme.  Each byte uses 7 bits for data, with the most significant bit indicating whether additional bytes follow.  % https://github.com/serai-dex/serai/blob/48db06f901952b24bb38d7c7e256f798f08512cd/networks/monero/io/src/lib.rs#L18 and https://github.com/serai-dex/serai/blob/48db06f901952b24bb38d7c7e256f798f08512cd/networks/monero/io/src/lib.rs#L60-L72
The implementation ensures canonical encoding by rejecting unnecessary leading zeros.  % https://github.com/serai-dex/serai/blob/48db06f901952b24bb38d7c7e256f798f08512cd/networks/monero/io/src/lib.rs#L139-L141

\item[Scalar Serialization] Scalars employ a fixed 32-byte encoding in little-endian representation.  % https://github.com/serai-dex/serai/blob/48db06f901952b24bb38d7c7e256f798f08512cd/networks/monero/io/src/lib.rs#L75-L77
The implementation enforces canonical form by requiring fully reduced values modulo the curve order.  % https://github.com/serai-dex/serai/blob/48db06f901952b24bb38d7c7e256f798f08512cd/networks/monero/io/src/lib.rs#L157-L160

\item[Point Serialization] Points use the compressed Edwards format in 32 bytes.  The Y-coordinate occupies bytes 0-30, while byte 31 stores the X-coordinate sign bit.  % https://github.com/serai-dex/serai/blob/48db06f901952b24bb38d7c7e256f798f08512cd/networks/monero/io/src/lib.rs#L80-L82
Validation ensures both canonical encoding and proper subgroup membership when requested.  % https://github.com/serai-dex/serai/blob/48db06f901952b24bb38d7c7e256f798f08512cd/networks/monero/io/src/lib.rs#L171-L176 and https://github.com/serai-dex/serai/blob/48db06f901952b24bb38d7c7e256f798f08512cd/networks/monero/io/src/lib.rs#L188-L193

\item[Integer Types] The implementation supports little-endian encoding of standard widths: % https://github.com/serai-dex/serai/blob/48db06f901952b24bb38d7c7e256f798f08512cd/networks/monero/io/src/lib.rs#L119-L131
\begin{itemize}
\item u16: 2 bytes
\item u32: 4 bytes  
\item u64: 8 bytes
\end{itemize}

\item[Vector Operations] The crate supports two vector serialization formats:
\begin{itemize}
\item Raw vectors with consecutive elements % https://github.com/serai-dex/serai/blob/48db06f901952b24bb38d7c7e256f798f08512cd/networks/monero/io/src/lib.rs#L85-L94
\item Length-prefixed vectors using VarInt length encoding % https://github.com/serai-dex/serai/blob/48db06f901952b24bb38d7c7e256f798f08512cd/networks/monero/io/src/lib.rs#L97-L104
\end{itemize}
\end{description}

The implementation enforces strict canonicalization through validation of:
\begin{itemize}
\item VarInt minimal encoding % https://github.com/serai-dex/serai/blob/48db06f901952b24bb38d7c7e256f798f08512cd/networks/monero/io/src/lib.rs#L139-L141
\item Scalar reduction state % https://github.com/serai-dex/serai/blob/48db06f901952b24bb38d7c7e256f798f08512cd/networks/monero/io/src/lib.rs#L157-L160
\item Point coordinate canonical form % https://github.com/serai-dex/serai/blob/48db06f901952b24bb38d7c7e256f798f08512cd/networks/monero/io/src/lib.rs#L171-L176
\item Optional prime-order subgroup membership % https://github.com/serai-dex/serai/blob/48db06f901952b24bb38d7c7e256f798f08512cd/networks/monero/io/src/lib.rs#L188-L193
\end{itemize}


% -------------------------------------------------- %
\subsection{monero-generators (v0.1.0)}

The \texttt{hash\_to\_point} function implements Monero's \texttt{hash\_to\_ec} operation to deterministically maps 32 bytes to a point on the Ed25519 curve as in:

\begin{description}
\item[Input] \hfill 

\begin{itemize}
\item Takes exactly 32 bytes.  % https://github.com/serai-dex/serai/blob/48db06f901952b24bb38d7c7e256f798f08512cd/networks/monero/generators/src/hash_to_point.rs#L13
\end{itemize}

\item[Processing] \hfill 

\begin{enumerate}
\item Performs Keccak-256 hash of input bytes.  % https://github.com/serai-dex/serai/blob/48db06f901952b24bb38d7c7e256f798f08512cd/networks/monero/generators/src/hash_to_point.rs#L17 
\item Squares the hash result and doubles it to get value \texttt{v}.  % "
\item Computes intermediate values:
  \begin{itemize}
  \item \texttt{w = v + 1}.  % https://github.com/serai-dex/serai/blob/48db06f901952b24bb38d7c7e256f798f08512cd/networks/monero/generators/src/hash_to_point.rs#L18
  \item \texttt{x = w² - A²v} where A = 486662 (curve parameter).  % https://github.com/serai-dex/serai/blob/48db06f901952b24bb38d7c7e256f798f08512cd/networks/monero/generators/src/hash_to_point.rs#L15 and https://github.com/serai-dex/serai/blob/48db06f901952b24bb38d7c7e256f798f08512cd/networks/monero/generators/src/hash_to_point.rs#L19
  \item Calculates a partial X coordinate through a series of field operations.  % https://github.com/serai-dex/serai/blob/48db06f901952b24bb38d7c7e256f798f08512cd/networks/monero/generators/src/hash_to_point.rs#L25-L33
  \end{itemize}
\item Derives Y coordinate through:
  \begin{itemize}
  \item Calculates \texttt{y = w - x}.  % https://github.com/serai-dex/serai/blob/48db06f901952b24bb38d7c7e256f798f08512cd/networks/monero/generators/src/hash_to_point.rs#L36
  \item Sets sign bit based on zero checks.  % https://github.com/serai-dex/serai/blob/48db06f901952b24bb38d7c7e256f798f08512cd/networks/monero/generators/src/hash_to_point.rs#L37-L39
  \item Performs field inversions and multiplications to get final Y value.  % https://github.com/serai-dex/serai/blob/48db06f901952b24bb38d7c7e256f798f08512cd/networks/monero/generators/src/hash_to_point.rs#L41-L48
  \end{itemize}
\item Sets the high bit of byte 31 based on calculated sign.  % https://github.com/serai-dex/serai/blob/48db06f901952b24bb38d7c7e256f798f08512cd/networks/monero/generators/src/hash_to_point.rs#L50
\end{enumerate}

\item[Output] \hfill 

\begin{itemize}
\item Returns an EdwardsPoint in compressed 32-byte format:  % https://github.com/serai-dex/serai/blob/48db06f901952b24bb38d7c7e256f798f08512cd/networks/monero/generators/src/hash_to_point.rs#L52
  \begin{itemize}
  \item Bytes 0-30: Y coordinate.  
  \item Byte 31: Sign bit of X coordinate in MSB.  % https://github.com/serai-dex/serai/blob/48db06f901952b24bb38d7c7e256f798f08512cd/networks/monero/generators/src/hash_to_point.rs#L50
  \end{itemize}
\item Multiplies result by cofactor (8) to ensure point is in the prime-order subgroup.  % https://github.com/serai-dex/serai/blob/48db06f901952b24bb38d7c7e256f798f08512cd/networks/monero/generators/src/hash_to_point.rs#L52 (mul_by_cofactor)
\end{itemize}

\end{description}

The function uses the \texttt{FieldElement} type from dalek-ff-group for constant-time field arithmetic operations over the Ed25519 field of order $2^{255} - 19$.  All operations are performed in constant time as a mitigation against timing-based side-channel attacks.  

\subsubsection{monero-generators Operations}

The \texttt{monero-generators} crate provides generator points for Pedersen commitments and Bulletproofs in Monero as in:

\subsubsection{Core Generator H}

\begin{description}
\item[Generation] \hfill \\
The base generator point H is computed as in:
\begin{enumerate}
\item Take ED25519\_BASEPOINT\_POINT (the standard Ed25519 base point G).  % https://github.com/serai-dex/serai/blob/48db06f901952b24bb38d7c7e256f798f08512cd/networks/monero/generators/src/lib.rs#L10 and https://github.com/serai-dex/serai/blob/48db06f901952b24bb38d7c7e256f798f08512cd/networks/monero/generators/src/lib.rs#L29-L31
\item Compress it to 32 bytes.  % https://github.com/serai-dex/serai/blob/48db06f901952b24bb38d7c7e256f798f08512cd/networks/monero/generators/src/lib.rs#L30
\item Compute Keccak-256 hash of these bytes.  % "
\item Map the hash to a curve point using \texttt{hash\_to\_point}.  % "
\item Multiply by the cofactor (8).  % https://github.com/serai-dex/serai/blob/48db06f901952b24bb38d7c7e256f798f08512cd/networks/monero/generators/src/lib.rs#L32
\end{enumerate}

\item[Powers Table] \hfill \\
Precomputes powers of H for efficient amount commitments as in:
\begin{itemize}
\item Creates array of 64 points $[H, H*2, H*4, ..., H*2^{63}]$. \free{should be exponents?}
% https://github.com/serai-dex/serai/blob/48db06f901952b24bb38d7c7e256f798f08512cd/networks/monero/generators/src/lib.rs#L35-L41
\item Each entry is double the previous entry.  % https://github.com/serai-dex/serai/blob/48db06f901952b24bb38d7c7e256f798f08512cd/networks/monero/generators/src/lib.rs#L38
\item Used for efficient decomposition of u64 amounts.  % https://github.com/serai-dex/serai/blob/48db06f901952b24bb38d7c7e256f798f08512cd/networks/monero/generators/src/lib.rs#L42-L48
\end{itemize}
\end{description}

\subsubsection{Bulletproofs Generator Vector Creation}

\begin{description}
\item[Constants] \hfill \\
\begin{itemize}
\item MAX\_COMMITMENTS: 16 (maximum provable commitments per range proof).  % https://github.com/serai-dex/serai/blob/48db06f901952b24bb38d7c7e256f798f08512cd/networks/monero/generators/src/lib.rs#L51
\item COMMITMENT\_BITS: 64 (bits per commitment value).  % https://github.com/serai-dex/serai/blob/48db06f901952b24bb38d7c7e256f798f08512cd/networks/monero/generators/src/lib.rs#L53
\item MAX\_MN: MAX\_COMMITMENTS * COMMITMENT\_BITS (total bits in proof).  % https://github.com/serai-dex/serai/blob/48db06f901952b24bb38d7c7e256f798f08512cd/networks/monero/generators/src/lib.rs#L72
\end{itemize}

\item[Generator Creation] \hfill \\
Takes a domain separation tag (\texttt{dst}) and produces two vectors of points: % https://github.com/serai-dex/serai/blob/48db06f901952b24bb38d7c7e256f798f08512cd/networks/monero/generators/src/lib.rs#L70
\begin{enumerate}
\item Create preimage = H.compress() || dst.  % https://github.com/serai-dex/serai/blob/48db06f901952b24bb38d7c7e256f798f08512cd/networks/monero/generators/src/lib.rs#L74-L75
\item For i from 0 to MAX\_MN - 1: % https://github.com/serai-dex/serai/blob/48db06f901952b24bb38d7c7e256f798f08512cd/networks/monero/generators/src/lib.rs#L78-L89
  \begin{itemize}
  \item Let j = 2*i.  % https://github.com/serai-dex/serai/blob/48db06f901952b24bb38d7c7e256f798f08512cd/networks/monero/generators/src/lib.rs#L80
  \item H[i] = hash\_to\_point(Keccak256(preimage || varint(j))).  % https://github.com/serai-dex/serai/blob/48db06f901952b24bb38d7c7e256f798f08512cd/networks/monero/generators/src/lib.rs#L82-L84
  \item G[i] = hash\_to\_point(Keccak256(preimage || varint(j+1))).  % https://github.com/serai-dex/serai/blob/48db06f901952b24bb38d7c7e256f798f08512cd/networks/monero/generators/src/lib.rs#L86-L88
  \end{itemize}
\end{enumerate}

\item[Output] \hfill \\
Returns \texttt{Generators} struct containing: % https://github.com/serai-dex/serai/blob/48db06f901952b24bb38d7c7e256f798f08512cd/networks/monero/generators/src/lib.rs#L58-L64
\begin{itemize}
\item G: Vector of MAX\_MN points for value commitments.  % https://github.com/serai-dex/serai/blob/48db06f901952b24bb38d7c7e256f798f08512cd/networks/monero/generators/src/lib.rs#L61
\item H: Vector of MAX\_MN points for mask commitments.  % https://github.com/serai-dex/serai/blob/48db06f901952b24bb38d7c7e256f798f08512cd/networks/monero/generators/src/lib.rs#L63
\end{itemize}
\end{description}

All operations maintain constant-time properties through:
\begin{itemize}
\item Use of \texttt{LazyLock} for thread-safe lazy initialization.  % https://github.com/serai-dex/serai/blob/48db06f901952b24bb38d7c7e256f798f08512cd/networks/monero/generators/src/lib.rs#L6 and https://github.com/serai-dex/serai/blob/48db06f901952b24bb38d7c7e256f798f08512cd/networks/monero/generators/src/lib.rs#L29 and https://github.com/serai-dex/serai/blob/48db06f901952b24bb38d7c7e256f798f08512cd/networks/monero/generators/src/lib.rs#L35
\item Constant-time point operations from curve25519-dalek.  % https://github.com/serai-dex/serai/blob/48db06f901952b24bb38d7c7e256f798f08512cd/networks/monero/generators/src/lib.rs#L10
\item Constant-time hash-to-point mapping.  % https://github.com/serai-dex/serai/blob/48db06f901952b24bb38d7c7e256f798f08512cd/networks/monero/generators/src/hash_to_point.rs#L1
\end{itemize}


% --------------------------------------------------- %
\subsection{monero-primitives (v0.1.0)}

\subsubsection{Overview}
The \texttt{monero-primitives} crate provides the core cryptographic operations required by Monero's protocol.  The crate operates in both \texttt{std} and \texttt{no-std} environments, with certain optimizations available when \texttt{std} is enabled.


% ---------- nesting UnreducedScalar, Commitment, Decoys into Core Types

\subsubsection{Core Types}

% Replacing the subsubsection function with the paragraph function
\subsubsubsection{ UnreducedScalar} 
A structure representing an unreduced scalar value, primarily used to handle legacy Monero code sections where scalar reduction was not enforced: 

% https://github.com/serai-dex/serai/blob/48db06f901952b24bb38d7c7e256f798f08512cd/networks/monero/primitives/src/unreduced_scalar.rs#L22-L30 


\begin{verbatim}
pub struct UnreducedScalar(pub [u8; 32]);
\end{verbatim}

Key operations:
\begin{itemize}
  \item \texttt{recover\_monero\_slide\_scalar}: Recovers the scalar that would have been incorrectly interpreted by Monero's \texttt{slide} function due to lack of reduction checks in Borromean range proofs. % https://github.com/serai-dex/serai/blob/48db06f901952b24bb38d7c7e256f798f08512cd/networks/monero/primitives/src/unreduced_scalar.rs#L110-L136
  \item \texttt{non\_adjacent\_form}: Computes the width-5 non-adjacent form (NAF) representation, intentionally matching Monero's potentially incorrect behavior. % https://github.com/serai-dex/serai/blob/48db06f901952b24bb38d7c7e256f798f08512cd/networks/monero/primitives/src/unreduced_scalar.rs#L52-L108
\end{itemize}

% Replacing the subsubsection function with the paragraph function
\subsubsubsection{ Commitment}
Represents a transparent Pedersen commitment with the following structure: % https://github.com/serai-dex/serai/blob/48db06f901952b24bb38d7c7e256f798f08512cd/networks/monero/primitives/src/lib.rs#L80-L88

\begin{verbatim}
pub struct Commitment {
    pub mask: Scalar,
    pub amount: u64
}
\end{verbatim}

Operations:
\begin{itemize}
  \item \texttt{calculate}: Computes the Pedersen commitment point as $amount \cdot G + mask \cdot H$.  % https://github.com/serai-dex/serai/blob/48db06f901952b24bb38d7c7e256f798f08512cd/networks/monero/primitives/src/lib.rs#L107-L110
  \item \texttt{zero}: Creates a commitment to zero using a mask value of 1.  % https://github.com/serai-dex/serai/blob/48db06f901952b24bb38d7c7e256f798f08512cd/networks/monero/primitives/src/lib.rs#L97-L100
  \item Serialization and deserialization functionality.
\end{itemize}

% Replacing the subsubsection function with the paragraph function
\subsubsubsection{ Decoys}
Manages ring signature data with the structure: % https://github.com/serai-dex/serai/blob/48db06f901952b24bb38d7c7e256f798f08512cd/networks/monero/primitives/src/lib.rs#L140-L146

\begin{verbatim}
pub struct Decoys {
    offsets: Vec<u64>,
    signer_index: u8,
    ring: Vec<[EdwardsPoint; 2]>
}
\end{verbatim}

Features:
\begin{itemize}
  \item Stores ring member positions as offsets from previous positions.  % https://github.com/serai-dex/serai/blob/48db06f901952b24bb38d7c7e256f798f08512cd/networks/monero/primitives/src/lib.rs#L184-L192
  \item Maintains key-commitment pairs for each ring member.  % https://github.com/serai-dex/serai/blob/48db06f901952b24bb38d7c7e256f798f08512cd/networks/monero/primitives/src/lib.rs#L145
  \item Provides access to the signer's position and ring members. % https://github.com/serai-dex/serai/blob/48db06f901952b24bb38d7c7e256f798f08512cd/networks/monero/primitives/src/lib.rs#L194-L207
\end{itemize}

\subsubsection{Core Operations}


% Replacing the subsubsection function with the paragraph function
\subsubsubsection{ Hash Operations}
\begin{itemize}
  \item \texttt{keccak256}: Computes Keccak-256 hash of input data.  % https://github.com/serai-dex/serai/blob/48db06f901952b24bb38d7c7e256f798f08512cd/networks/monero/primitives/src/lib.rs#L62-L65
  \item \texttt{keccak256\_to\_scalar}: Maps input to a scalar via $keccak256(\text{data}) \bmod l$, where $l$ is Ed25519's prime order. % https://github.com/serai-dex/serai/blob/48db06f901952b24bb38d7c7e256f798f08512cd/networks/monero/primitives/src/lib.rs#L67-L78
\end{itemize}

% Replacing the subsubsection function with the paragraph function
\subsubsubsection{ Cached Computations}
When the \texttt{std} feature is enabled:
\begin{itemize}
  \item \texttt{INV\_EIGHT}: Caches $8^{-1} \bmod l$.  % https://github.com/serai-dex/serai/blob/48db06f901952b24bb38d7c7e256f798f08512cd/networks/monero/primitives/src/lib.rs#L29-L44
  \item \texttt{G\_PRECOMP}: Maintains precomputed tables for the Ed25519 base point. % https://github.com/serai-dex/serai/blob/48db06f901952b24bb38d7c7e256f798f08512cd/networks/monero/primitives/src/lib.rs#L46-L60
\end{itemize}

\subsubsection{Implementation Details}

% Replacing the subsubsection function with the paragraph function
\subsubsubsection{ NAF Implementation}
% https://github.com/serai-dex/serai/blob/48db06f901952b24bb38d7c7e256f798f08512cd/networks/monero/primitives/src/unreduced_scalar.rs#L52-L105
\begin{itemize}
  \item Non-constant time operation, explicitly matching Monero's implementation.  % Documented at https://github.com/serai-dex/serai/blob/48db06f901952b24bb38d7c7e256f798f08512cd/networks/monero/primitives/src/unreduced_scalar.rs#L57
  \item Window size is fixed at 5 bits.  % https://github.com/serai-dex/serai/blob/48db06f901952b24bb38d7c7e256f798f08512cd/networks/monero/primitives/src/unreduced_scalar.rs#L69
  \item Edge cases:
    \begin{itemize}
      \item Handles overflow beyond 256 bits by truncating.  % https://github.com/serai-dex/serai/blob/48db06f901952b24bb38d7c7e256f798f08512cd/networks/monero/primitives/src/unreduced_scalar.rs#L70-L74
      \item Uses special carry propagation when the sum exceeds 15 or the difference goes below -15.  % https://github.com/serai-dex/serai/blob/48db06f901952b24bb38d7c7e256f798f08512cd/networks/monero/primitives/src/unreduced_scalar.rs#L78-L101
      \item Matches Monero's behavior even when cryptographically suboptimal.% 
    \end{itemize}
\end{itemize}

The NAF computation in \texttt{non\_adjacent\_form} follows these steps:
\begin{enumerate}
  \item Convert input to a bit array.  % https://github.com/serai-dex/serai/blob/48db06f901952b24bb38d7c7e256f798f08512cd/networks/monero/primitives/src/unreduced_scalar.rs#L43-L50
  \item Process bits sequentially.  % https://github.com/serai-dex/serai/blob/48db06f901952b24bb38d7c7e256f798f08512cd/networks/monero/primitives/src/unreduced_scalar.rs#L61-L63
  \item For each non-zero bit: % https://github.com/serai-dex/serai/blob/48db06f901952b24bb38d7c7e256f798f08512cd/networks/monero/primitives/src/unreduced_scalar.rs#L66-L104
    \begin{itemize}
      \item Examine the next 5 bits (window).  % https://github.com/serai-dex/serai/blob/48db06f901952b24bb38d7c7e256f798f08512cd/networks/monero/primitives/src/unreduced_scalar.rs#L69-L74
      \item Combine bits when sum $\leq 15$ or difference $\geq -15$.  % https://github.com/serai-dex/serai/blob/48db06f901952b24bb38d7c7e256f798f08512cd/networks/monero/primitives/src/unreduced_scalar.rs#L79-L84
      \item Handle carry propagation.  % https://github.com/serai-dex/serai/blob/48db06f901952b24bb38d7c7e256f798f08512cd/networks/monero/primitives/src/unreduced_scalar.rs#L91-L98
    \end{itemize}
\end{enumerate}

This intentionally matches Monero's implementation, including specific edge cases.

% Replacing the subsubsection function with the paragraph function
\subsubsubsection{ Scalar Recovery}
The \texttt{recover\_monero\_slide\_scalar} function:
\begin{itemize}
  \item Returns direct reduction if the high bit is not set.  % https://github.com/serai-dex/serai/blob/48db06f901952b24bb38d7c7e256f798f08512cd/networks/monero/primitives/src/unreduced_scalar.rs#L119-L124
  \item Otherwise reconstructs the scalar from its NAF representation.  % https://github.com/serai-dex/serai/blob/48db06f901952b24bb38d7c7e256f798f08512cd/networks/monero/primitives/src/unreduced_scalar.rs#L127-L134
  \item Uses precomputed odd scalars for efficiency.  % https://github.com/serai-dex/serai/blob/48db06f901952b24bb38d7c7e256f798f08512cd/networks/monero/primitives/src/unreduced_scalar.rs#L13-L20
\end{itemize}

\subsubsection{Security Considerations}

% Replacing the subsubsection function with the paragraph function
\subsubsubsection{ Critical Assumptions}
\begin{itemize}
  \item A zero hash result in \texttt{keccak256\_to\_scalar} causes a panic.  % https://github.com/serai-dex/serai/blob/48db06f901952b24bb38d7c7e256f798f08512cd/networks/monero/primitives/src/lib.rs#L76
  \item \texttt{UnreducedScalar} operations do not guarantee constant-time execution.  % https://github.com/serai-dex/serai/blob/48db06f901952b24bb38d7c7e256f798f08512cd/networks/monero/primitives/src/lib.rs#L55-L60
  \item Commitment masking relies on the discrete logarithm assumption in the Ed25519 group.  % https://github.com/serai-dex/serai/blob/48db06f901952b24bb38d7c7e256f798f08512cd/networks/monero/primitives/src/lib.rs#L80-L88
\end{itemize}

% Replacing the subsubsection function with the paragraph function
\subsubsubsection{ Implementation Notes}
\begin{itemize}
  \item \textbf{Serialization}:
    \begin{itemize}
      \item Uses custom formats, distinct from Monero protocol standards.  % https://github.com/serai-dex/serai/blob/48db06f901952b24bb38d7c7e256f798f08512cd/networks/monero/primitives/src/lib.rs#L112-L137 and https://github.com/serai-dex/serai/blob/48db06f901952b24bb38d7c7e256f798f08512cd/networks/monero/primitives/src/lib.rs#L209-L248
      \item Provides no backwards compatibility guarantees.  % 
      \item Limits \texttt{Decoys} serialization to Edwards points only.  % https://github.com/serai-dex/serai/blob/48db06f901952b24bb38d7c7e256f798f08512cd/networks/monero/primitives/src/lib.rs#L216-L223
    \end{itemize}
  \item \textbf{Memory Safety}:
    \begin{itemize}
      \item Implements \texttt{Zeroize} and \texttt{ZeroizeOnDrop} for sensitive data.  % eg. https://github.com/serai-dex/serai/blob/48db06f901952b24bb38d7c7e256f798f08512cd/networks/monero/primitives/src/lib.rs#L82
      \item Performs careful bounds checking on ring operations. % https://github.com/serai-dex/serai/blob/48db06f901952b24bb38d7c7e256f798f08512cd/networks/monero/primitives/src/lib.rs#L165-L167
    \end{itemize}
  \item \textbf{Compatibility Choices}:
    \begin{itemize}
      \item Matches Monero behavior even when suboptimal.  % Documented at https://github.com/serai-dex/serai/blob/48db06f901952b24bb38d7c7e256f798f08512cd/networks/monero/primitives/src/unreduced_scalar.rs#L52-L57
      \item Preserves specific edge cases for protocol compatibility.  % https://github.com/serai-dex/serai/blob/48db06f901952b24bb38d7c7e256f798f08512cd/networks/monero/primitives/src/unreduced_scalar.rs#L119-L124
      \item Retains variable-time operations where Monero uses them. % Documented at https://github.com/serai-dex/serai/blob/48db06f901952b24bb38d7c7e256f798f08512cd/networks/monero/primitives/src/unreduced_scalar.rs#L52-L57
    \end{itemize}
\end{itemize}

\begin{itemize}
  \item Implements \texttt{Zeroize} and \texttt{ZeroizeOnDrop} for sensitive data.
  \item Matches Monero's behavior for compatibility, even if suboptimal.
  \item Employs crate-specific serialization formats, not the Monero protocol standard.
\end{itemize}


% --------------------------------------------- %
\subsection{monero-rpc (v0.1.0)}
\label{sec:monero-rpc-crate}

The \texttt{monero-rpc} crate provides abstractions for
communicating with a Monero daemon, retrieving various data (blocks, transactions,
fee estimates, \emph{etc.}), and publishing transactions.  The crate relies on data
structures from \texttt{monero-serai}, using them to parse and serialize Monero
primitives (such as transactions and blocks).  It offers a specialized trait for
decoy selection.

\subsubsection{Overview}
\label{sec:monero-rpc-overview}

The \texttt{monero-rpc} crate defines:

\begin{enumerate}
    \item \textbf{\texttt{RpcError}}: An enumeration capturing errors that may
    arise when performing RPC calls (see
    \S\ref{sec:monero-rpc-rpcerror}). % https://github.com/serai-dex/serai/blob/48db06f901952b24bb38d7c7e256f798f08512cd/networks/monero/rpc/src/lib.rs#L44-L74

    \item \textbf{\texttt{Rpc} trait}: The primary abstraction for interacting
    with a Monero daemon.  It defines a set of asynchronous methods for
    retrieving blocks, transactions, and other chain data, as well as
    publishing transactions (see \S\ref{sec:monero-rpc-rpc-trait}). % https://github.com/serai-dex/serai/blob/48db06f901952b24bb38d7c7e256f798f08512cd/networks/monero/rpc/src/lib.rs#L244-L252

    \item \textbf{\texttt{DecoyRpc} trait}: A higher-level trait extending the
    concept of retrieving outputs for constructing ring signatures (see
    \S\ref{sec:monero-rpc-decoy-rpc}). % https://github.com/serai-dex/serai/blob/48db06f901952b24bb38d7c7e256f798f08512cd/networks/monero/rpc/src/lib.rs#L1001-L1028 and https://github.com/serai-dex/serai/blob/48db06f901952b24bb38d7c7e256f798f08512cd/networks/monero/rpc/src/lib.rs#L1049-L1269

    \item \textbf{Various helpers and supporting items}, such as:
    \begin{itemize}
      \item \texttt{ScannableBlock}, bundling a \emph{Monero} block together
      with its non-miner transactions (in pruned form), plus metadata used when
      scanning outputs. % https://github.com/serai-dex/serai/blob/48db06f901952b24bb38d7c7e256f798f08512cd/networks/monero/rpc/src/lib.rs#L76-L87
      \item \texttt{FeeRate} and \texttt{FeePriority}, providing abstractions
      for Monero fee estimation. % https://github.com/serai-dex/serai/blob/48db06f901952b24bb38d7c7e256f798f08512cd/networks/monero/rpc/src/lib.rs#L76-L87
      \item Utility functions to parse certain specialized binary responses from
      a node, such as \texttt{get\_o\_indexes.bin}. % https://github.com/serai-dex/serai/blob/48db06f901952b24bb38d7c7e256f798f08512cd/networks/monero/rpc/src/lib.rs#L812-L999
    \end{itemize}
\end{enumerate}

Implementors of \texttt{Rpc} are expected to supply the low-level HTTP or
transport logic, including authentication if applicable.  The actual requests and
responses are shaped to match Monero’s JSON-RPC and specialized “binary”
endpoints.  This allows various backends (e.g., over Tor/i2p, local node,
dedicated HTTP crates) to be slotted in.

\subsubsection{\texttt{RpcError} Enumeration}
\label{sec:monero-rpc-rpcerror}

The \texttt{RpcError} enum defines all error conditions encountered by the RPC
layer: % https://github.com/serai-dex/serai/blob/48db06f901952b24bb38d7c7e256f798f08512cd/networks/monero/rpc/src/lib.rs#L43-L74

\begin{itemize}
    \item \textbf{\texttt{InternalError(String)}} --- Typically signals some
    logical problem unrelated to standard node replies (for instance,
    constructing a request with out-of-range parameters).

    \item \textbf{\texttt{ConnectionError(String)}} --- Indicates an inability
    to reach or properly communicate with the daemon (e.g.\ timeouts, malformed
    responses, or connectivity disruptions).

    \item \textbf{\texttt{InvalidNode(String)}} --- The remote node returned
    data not conforming to the Monero protocol, or otherwise gave impossible or
    contradictory information (suspecting a malicious or misconfigured node).

    \item \textbf{\texttt{TransactionsNotFound(Vec<[u8; 32]>)}} --- One or more
    requested transactions could not be retrieved by the node.

    \item \textbf{\texttt{PrunedTransaction}} --- A transaction was retrieved in
    a pruned form when unpruned data was required.  In the current code, pruned
    transactions are not considered usable for certain operations.

    \item \textbf{\texttt{InvalidTransaction([u8; 32])}} --- The node claims a
    transaction is valid, but it fails local parsing or verification.

    \item \textbf{\texttt{InvalidFee}} --- The fee returned by the node was
    nonsensical or out-of-range for safe usage.

    \item \textbf{\texttt{InvalidPriority}} --- The given fee priority could not
    be honored or mapped (e.g.\ out of the valid \texttt{1--4} range for
    Monero's known fee multipliers).
\end{itemize}

Such typed error handling allows the upper layers or other consumers of the
\texttt{monero-rpc} crate to correctly distinguish user-facing issues (like
\texttt{TransactionsNotFound}) from transport or malicious-node issues
(\texttt{ConnectionError} or \texttt{InvalidNode}).

\subsubsection{\texttt{Rpc} Trait}
\label{sec:monero-rpc-rpc-trait}

The \texttt{Rpc} trait is the centerpiece of the crate, defining the primary
asynchronous calls for interacting with a Monero daemon.  Its methods can be
broken into three categories: \emph{primitive calls}, \emph{block/transaction
retrieval}, and \emph{transaction publishing}. % https://github.com/serai-dex/serai/blob/48db06f901952b24bb38d7c7e256f798f08512cd/networks/monero/rpc/src/lib.rs#L244-L999

\subsubsection{\texttt{post}}
\label{sec:monero-rpc-rpc-trait-post}

\begin{verbatim}
fn post(
    &self,
    route: &str,
    body: Vec<u8>
) -> impl Future<Output = Result<Vec<u8>, RpcError>> + Send;
\end{verbatim}
% https://github.com/serai-dex/serai/blob/48db06f901952b24bb38d7c7e256f798f08512cd/networks/monero/rpc/src/lib.rs#L253-L260

At the lowest level, \texttt{post} takes an HTTP route string (e.g.:
\texttt{"get\_transactions"}) and a raw byte vector containing the request body.
It returns a future resolving to a \texttt{Result} with the raw byte vector for
the response (or an error).  Implementors must handle:

\begin{itemize}
    \item HTTP or HTTPS connections.
    \item Digest or basic authentication.
    \item Connection pooling or caching.
\end{itemize}

Nothing in \texttt{monero-rpc} enforces a particular method of connection;
\texttt{post} abstracts this detail away.

\subsubsection{\texttt{rpc\_call} and \texttt{json\_rpc\_call}}
\label{sec:monero-rpc-rpc-trait-jsonrpc}

\begin{verbatim}
fn rpc_call<Params: Serialize + Debug,
            Response: DeserializeOwned + Debug>(
    &self,
    route: &str,
    params: Option<Params>
) -> impl Future<Output = Result<Response, RpcError>> + Send;

fn json_rpc_call<Response: DeserializeOwned + Debug>(
    &self,
    method: &str,
    params: Option<Value>
) -> impl Future<Output = Result<Response, RpcError>> + Send;
\end{verbatim}
% https://github.com/serai-dex/serai/blob/48db06f901952b24bb38d7c7e256f798f08512cd/networks/monero/rpc/src/lib.rs#L262-L287 and https://github.com/serai-dex/serai/blob/48db06f901952b24bb38d7c7e256f798f08512cd/networks/monero/rpc/src/lib.rs#L289-L302

These two methods use \texttt{post} internally, providing:

\begin{enumerate}
    \item \texttt{rpc\_call}: For “raw” RPC calls, typically when a daemon
    endpoint does not follow the JSON-RPC 2.0 standard but instead uses
    POST-based JSON data (for example, \texttt{get\_transactions}).

    \item \texttt{json\_rpc\_call}: For endpoints explicitly requiring
    \texttt{json\_rpc} protocol usage (e.g. \\\texttt{get\_block\_header\_by\_height}).
    The parameters and method name are placed into the JSON-RPC structure, sent,
    and the result is deserialized back to a typed \texttt{Response}.
\end{enumerate}

Both automatically handle JSON deserialization into strongly typed Rust
structures.  If the response is malformed or the node returns an error-like
structure, \texttt{RpcError::InvalidNode} or \texttt{RpcError::ConnectionError}
may be produced.

\subsubsection{\texttt{get\_height}, \texttt{get\_transactions}, and Block-Related Methods}
\rigo{not sure why Block-Related Methods is a different font. Is it a function? Because then it can be the same font}

The crate includes convenience methods for:
\begin{itemize}
    \item \texttt{get\_height()}: Returns the current chain height (the number
    of blocks, with the genesis block counted as height 1). % https://github.com/serai-dex/serai/blob/48db06f901952b24bb38d7c7e256f798f08512cd/networks/monero/rpc/src/lib.rs#L338-L354
    \item \texttt{get\_block}, \texttt{get\_block\_by\_number}, and
    \texttt{get\_block\_hash}: Retrieve and verify specific blocks, either by
    zero-indexed block number or by 32-byte hash.  The code ensures the block’s
    serialized hash matches the node’s reported identifier, guarding against
    untrusted or corrupted data. % https://github.com/serai-dex/serai/blob/48db06f901952b24bb38d7c7e256f798f08512cd/networks/monero/rpc/src/lib.rs#L504-L584
    \item \texttt{get\_transactions} and \texttt{get\_pruned\_transactions}: % https://github.com/serai-dex/serai/blob/48db06f901952b24bb38d7c7e256f798f08512cd/networks/monero/rpc/src/lib.rs#L483
    Fetch zero or more transactions by their 32-byte hash, verifying they match
    local expectations (checking the computed transaction hash).  Pruned or
    unpruned forms are selectively parsed via the relevant entry point.
    \item \texttt{get\_transaction} (unpruned) and
    \texttt{get\_pruned\_transaction}: Single-transaction convenience wrappers. % https://github.com/serai-dex/serai/blob/48db06f901952b24bb38d7c7e256f798f08512cd/networks/monero/rpc/src/lib.rs#L485-L494
\end{itemize}

Block retrieval can optionally return a \texttt{ScannableBlock}, packaging a
\texttt{Block} and its pruned non-coinbase transactions, allowing scanning for
RingCT outputs. % https://github.com/serai-dex/serai/blob/48db06f901952b24bb38d7c7e256f798f08512cd/networks/monero/rpc/src/lib.rs#L586-L655

\subsubsection{\texttt{publish\_transaction}}
\label{sec:monero-rpc-rpc-trait-publish}

\begin{verbatim}
fn publish_transaction(
    &self,
    tx: &Transaction
) -> impl Future<Output = Result<(), RpcError>> + Send;
\end{verbatim}
% https://github.com/serai-dex/serai/blob/48db06f901952b24bb38d7c7e256f798f08512cd/networks/monero/rpc/src/lib.rs#L742-L777

Publishes a \texttt{Transaction} to the network.  It leverages the
\texttt{send\_raw\_transaction} route (or equivalent) on the Monero daemon.  
Should the daemon reject the transaction (for instance, if it conflicts with a
recently seen double spend), an appropriate \texttt{RpcError} is returned.

\subsubsection{\texttt{get\_fee\_rate}}
\label{sec:monero-rpc-rpc-trait-fee-estimate}

\begin{verbatim}
fn get_fee_rate(
    &self,
    priority: FeePriority
) -> impl Future<Output = Result<FeeRate, RpcError>> + Send;
\end{verbatim}
% https://github.com/serai-dex/serai/blob/48db06f901952b24bb38d7c7e256f798f08512cd/networks/monero/rpc/src/lib.rs#L675-L740

Requests fee estimates for a particular \texttt{FeePriority} (one of
\texttt{Unimportant}, \texttt{Normal}, \texttt{Elevated}, \texttt{Priority}, or
\texttt{Custom}).  Internally, it interprets the JSON response to obtain the fee
in “per-weight” units and a quantization mask.  The resulting \texttt{FeeRate}
structure:

\begin{itemize}
    \item \texttt{per\_weight}: The base fee per weight unit.
    \item \texttt{mask}: A mask used to round the final transaction fee upward.
\end{itemize}

The consumer can compute the final fee for a transaction based on the
transaction weight and this \texttt{FeeRate}.  If an unrecognized or invalid
priority is requested, \texttt{RpcError::InvalidPriority} is raised.

\subsubsection{\texttt{generate\_blocks}}
\label{sec:monero-rpc-rpc-trait-generate-blocks}

While not part of a typical production Monero daemon (where block generation is
not done via RPC in mainnet environments), \texttt{generate\_blocks} is useful
in a local testing context (e.g.\ in a private regtest setup).  It creates
blocks, awarding the block reward to the specified \texttt{Address}.

\begin{verbatim}
fn generate_blocks<const ADDR_BYTES: u128>(
    &self,
    address: &Address<ADDR_BYTES>,
    block_count: usize
) -> impl Future<Output = Result<(Vec<[u8; 32]>, usize), RpcError>> + Send;
\end{verbatim}
% https://github.com/serai-dex/serai/blob/48db06f901952b24bb38d7c7e256f798f08512cd/networks/monero/rpc/src/lib.rs#L779-L810

This returns a list of the newly mined blocks’ hashes and the resulting height.

\subsubsection{\texttt{DecoyRpc} Trait}
\label{sec:monero-rpc-decoy-rpc}

The \texttt{DecoyRpc} trait extends the base RPC for selecting decoy outputs.  
When constructing Monero transactions, ring signatures typically reference
random “decoy” outputs on-chain, enabling privacy.  \texttt{DecoyRpc} adds
specialized queries for zero-amount RingCT outputs:

\begin{itemize}
    \item \texttt{get\_output\_distribution\_end\_height()}: Reports the maximum
    block index used in the distribution.  Typically this equals the chain
    height. % https://github.com/serai-dex/serai/blob/48db06f901952b24bb38d7c7e256f798f08512cd/networks/monero/rpc/src/lib.rs#L1007-L1013
    \item \texttt{get\_output\_distribution(range: Range<usize>)}: Returns a
    vector of cumulative output counts for that block range.  By focusing on
    zero-amount outputs, it leverages the \texttt{get\_output\_distribution}
    endpoint. % https://github.com/serai-dex/serai/blob/48db06f901952b24bb38d7c7e256f798f08512cd/networks/monero/rpc/src/lib.rs#L1015-L1022
    \item \texttt{get\_outs(indexes: \&[u64])}: Fetches details for a batch of
    RingCT outputs, including the block height, unlocking status, compressed key
    (\texttt{C\_i}), and the commitment (\texttt{mask}). % https://github.com/serai-dex/serai/blob/48db06f901952b24bb38d7c7e256f798f08512cd/networks/monero/rpc/src/lib.rs#L1024-L1028
    \item \texttt{get\_unlocked\_outputs(indexes, height, fingerprintable)}:
    Returns the subset of outputs that are actually unlocked by the node’s
    current chain state, or by a purely deterministic check (when
    \texttt{fingerprintable\_deterministic} is true).  This helps ensure
    decoys represent valid, spendable outputs from the node’s perspective. % https://github.com/serai-dex/serai/blob/48db06f901952b24bb38d7c7e256f798f08512cd/networks/monero/rpc/src/lib.rs#L1030-L1046
\end{itemize}

In typical usage, one might maintain a local database of zero-amount outputs
(built incrementally), but \texttt{DecoyRpc} allows direct on-demand retrieval
from a node if required.  Implementing \texttt{DecoyRpc} on top of \texttt{Rpc}
is straightforward, as the crate includes a default blanket implementation for
any type satisfying \texttt{Rpc}.

\subsubsection{Supporting Types and Structures}
\label{sec:monero-rpc-supporting-types}


% Replacing the subsubsection function with the paragraph function
\subsubsubsection{ \texttt{ScannableBlock}}
\label{sec:monero-rpc-supporting-types-scannableblock}

\begin{verbatim}
pub struct ScannableBlock {
    pub block: Block,
    pub transactions: Vec<Transaction<Pruned>>,
    pub output_index_for_first_ringct_output: Option<u64>,
}
\end{verbatim}
% https://github.com/serai-dex/serai/blob/48db06f901952b24bb38d7c7e256f798f08512cd/networks/monero/rpc/src/lib.rs#L76-L87

\texttt{ScannableBlock} encapsulates a fully read, unpruned coinbase
(\textit{miner}) transaction plus the pruned forms of other transactions, along
with an optional \texttt{output\_index\_for\_first\_ringct\_output}.  This index
helps to avoid repeated queries for the output index of every single
RingCT output in the block, which can be a privacy leak.

% Replacing the subsubsection function with the paragraph function
\subsubsubsection{ \texttt{FeeRate}}
\label{sec:monero-rpc-supporting-types-feerate}

\begin{verbatim}
pub struct FeeRate {
    per_weight: u64,
    mask: u64,
}
\end{verbatim}
% https://github.com/serai-dex/serai/blob/48db06f901952b24bb38d7c7e256f798f08512cd/networks/monero/rpc/src/lib.rs#L89-L152

This data structure is an interpreted result from the node’s fee estimate.  
The 

\texttt{FeeRate::calculate\_fee\_from\_weight} method performs final rounding:
\[
\text{fee} = \left\lceil \frac{\text{per\_weight}
             \times \text{tx\_weight}}{\text{mask}} \right\rceil
             \times \text{mask}.
\]
% https://github.com/serai-dex/serai/blob/48db06f901952b24bb38d7c7e256f798f08512cd/networks/monero/rpc/src/lib.rs#L138-L151


% Replacing the subsubsection function with the paragraph function
\subsubsubsection{ \texttt{FeePriority}}
\label{sec:monero-rpc-supporting-types-feepriority}

\begin{verbatim}
pub enum FeePriority {
    Unimportant,
    Normal,
    Elevated,
    Priority,
    Custom { priority: u32 },
}
\end{verbatim}
% https://github.com/serai-dex/serai/blob/48db06f901952b24bb38d7c7e256f798f08512cd/networks/monero/rpc/src/lib.rs#L154-L173

Monero typically recognizes four distinct priority levels: “unimportant,”
“normal,” “elevated,” and “priority.”  Each internally maps to a numeric
multiplier for the base fee.  The \texttt{Custom} variant allows direct numeric
control, though the node might reject unknown or extreme values.

\subsubsection{Security Considerations}
\label{sec:monero-rpc-security}

\begin{enumerate}
    \item \textbf{Node Trust and Validation}: Since
    \texttt{monero-rpc} can operate on untrusted nodes, it attempts some
    validation (for example, verifying that a transaction’s hash matches the
    requested hash).  However, in general, the node is relied upon for essential
    data about chain state.  The \texttt{InvalidNode} error signals a mismatch
    from protocol expectations.
    % https://github.com/serai-dex/serai/blob/48db06f901952b24bb38d7c7e256f798f08512cd/networks/monero/rpc/src/lib.rs#L418-L428 for example.
    
    \item \textbf{Confidentiality of Queries}: Repeatedly querying the node
    about certain output indexes or unconfirmed transactions can leak usage
    patterns.  Future or external solutions may integrate local caches or batch
    requests to reduce fingerprinting.
    
    \item \textbf{Pruned Transactions}: If the node can only supply pruned
    data for certain legacy or low-mixin transactions, the library raises
    \texttt{RpcError::PrunedTransaction} if unpruned data is strictly required.  
    Users must ensure they only proceed when they can meaningfully handle a
    pruned transaction (e.g.\ scanning coinbase outputs).
\end{enumerate}

\subsubsection{Summary of the \texttt{monero-rpc} Crate}
\label{sec:monero-rpc-conclusion}

The \texttt{monero-rpc} crate provides a clear, typed interface for interacting
with Monero daemons, either trusted or untrusted.  By defining the \texttt{Rpc}
trait, it ensures a uniform approach to retrieving, verifying, and publishing
transactions, while the \texttt{DecoyRpc} trait extends it to fetch random ring
outputs in a convenient, standardized manner.  This design integrates into
higher layers of the \texttt{monero-serai} stack, making it easy to develop new
functionality (e.g.\ bridging, multi-signature orchestrations, or custom
wallets) in a modular and verifiable way.


% ----------------------------------------------------%
\subsection{monero-bulletproofs (v0.1.0)}
\label{sec:monero-bulletproofs-impl}

The \texttt{monero-bulletproofs} crate implements both the original Bulletproofs range proof scheme and the newer Bulletproofs+ scheme for the Monero protocol.  The library is located within the Serai repository under \texttt{\small networks/monero/ringct/bulletproofs}.  Its purpose is to generate and verify range proofs over amounts in confidential transactions, ensuring that the amounts are in a valid range (specifically $[0, 2^{64})$) without revealing the amounts themselves.  Below is an overview of its structure, methods, and verification flow, referencing the relevant source files.

\subsubsection{Structure and Entry Points}
The crate exposes its main functionality via:
\begin{itemize}
    \item \texttt{Bulletproof}: An enum representing either the original Bulletproof or a Bulletproof+ proof. % https://github.com/serai-dex/serai/blob/48db06f901952b24bb38d7c7e256f798f08512cd/networks/monero/ringct/bulletproofs/src/lib.rs#L59-L69
    \item \texttt{prove} and \texttt{prove\_plus}: Functions that construct the respective Bulletproof or Bulletproof+ proofs for a list of commitments. % https://github.com/serai-dex/serai/blob/48db06f901952b24bb38d7c7e256f798f08512cd/networks/monero/ringct/bulletproofs/src/lib.rs#L108-L146
    \item \texttt{verify} and \texttt{batch\_verify}: Functions that verify these proofs, either in a standalone or batched manner. % https://github.com/serai-dex/serai/blob/48db06f901952b24bb38d7c7e256f798f08512cd/networks/monero/ringct/bulletproofs/src/lib.rs#L148-L173
\end{itemize}

Internally, the crate is separated into several modules:
\begin{itemize}
    \item \texttt{core}: Provides low-level operations such as multi-exponentiation routines and challenge product computation.  
    \item \texttt{batch\_verifier}: Holds accumulators that enable batching multiple Bulletproof or Bulletproof+ verifications into a single multi-exponentiation.  
    \item \texttt{original}: Implements the original Bulletproofs approach, including the inner-product proof (\texttt{IpProof}) used by \texttt{AggregateRangeProof}.  
    \item \texttt{plus}: Implements Bulletproofs+ as described by the Monero project.  It includes the improved weighted inner product proof (\texttt{WipProof}) and the aggregated range proof mechanism that uses it.  
    \item \texttt{scalar\_vector} and \texttt{point\_vector}: Utility types for vectors of \texttt{Scalar} and \texttt{EdwardsPoint}, offering safe indexing and algebraic operations.  
    \item \texttt{tests}: Internal tests that validate correctness across both schemes, employing batched and single verifications.  
\end{itemize}

\subsubsection{Range Proof Representation}
The top-level enum \texttt{Bulletproof} is:
\begin{itemize}
    \item \texttt{Original(OriginalProof)}: The classic Bulletproof structure using the original protocol design.  
    \item \texttt{Plus(PlusProof)}: The Bulletproof+ structure, notably smaller and with a slightly different internal proving mechanism.  
\end{itemize}

Each structure internally contains the group elements and scalars required for proof verification.  For instance, \texttt{OriginalProof} holds:
\begin{itemize}
    \item \texttt{(A, S, T1, T2)}: Points used in the range proof polynomial commitment.  
    \item \texttt{tau\_x, mu, t\_hat}: Challenge-related scalars for the polynomial identity check.  
    \item \texttt{ip}: An \texttt{IpProof} implementing the inner product argument.  
\end{itemize}
% https://github.com/serai-dex/serai/blob/48db06f901952b24bb38d7c7e256f798f08512cd/networks/monero/ringct/bulletproofs/src/original/mod.rs#L29-L39

Likewise, \texttt{PlusProof} holds:
\begin{itemize}
    \item \texttt{A}: The aggregated vector commitment.  
    \item \texttt{wip (WipProof)}: A specialized weighted inner product proof, the core cryptographic novelty of Bulletproofs+.  
\end{itemize}
% https://github.com/serai-dex/serai/blob/48db06f901952b24bb38d7c7e256f798f08512cd/networks/monero/ringct/bulletproofs/src/plus/aggregate_range_proof.rs#L41-L47

\subsubsection{Aggregated Range Proof Generation}
Both Bulletproof and Bulletproof+ provide \emph{aggregated} proofs that combine multiple range proofs for multiple outputs into a single proof whose size grows only logarithmically in the number of outputs.  Each scheme follows a ``commit-and-challenge'' approach similar to standard Bulletproof protocols, but they differ in their choice of transcript hashing and the particular steps for the inner product argument.  

\paragraph{Original Bulletproof (\texttt{original} module)}
The original code is in \texttt{original/mod.rs} and \\
\texttt{original/inner\_product.rs}, closely mirroring the \textit{Bulletproofs} paper.  Key steps:
\begin{enumerate}
    \item \emph{Precomputation:} Generators \texttt{G} and \texttt{H} are allocated, one for each bit of each commitment.  The maximum number of commitments is specified by \texttt{MAX\_COMMITMENTS}. % MAX_COMMITMENTS imported at https://github.com/serai-dex/serai/blob/48db06f901952b24bb38d7c7e256f798f08512cd/networks/monero/ringct/bulletproofs/src/original/mod.rs#L9 and generators at https://github.com/serai-dex/serai/blob/48db06f901952b24bb38d7c7e256f798f08512cd/networks/monero/ringct/bulletproofs/src/original/mod.rs#L17 (generators generated in https://github.com/serai-dex/serai/blob/48db06f901952b24bb38d7c7e256f798f08512cd/networks/monero/ringct/bulletproofs/build.rs)
    \item \emph{Polynomial Commitments:} The protocol encodes bits of the amounts in vectors $(a_L, a_R)$, from which it forms commitments $(A, S)$. % https://github.com/serai-dex/serai/blob/48db06f901952b24bb38d7c7e256f798f08512cd/networks/monero/ringct/bulletproofs/src/original/mod.rs#L120-L166
    \item \emph{Challenges:} The code uses Keccak-256 (converted to a \texttt{Scalar} via \texttt{keccak256\_to\_scalar}) to derive random challenges $y, z, x$. % https://github.com/serai-dex/serai/blob/48db06f901952b24bb38d7c7e256f798f08512cd/networks/monero/ringct/bulletproofs/src/original/mod.rs#L65-L82 transcript_A_S for y,z challenges, transcript_T12 for x challenge
    \item \emph{Constrained Polynomial Identity:} The aggregator constructs $(T_1, T_2)$ to bind the polynomial terms, and merges everything with the challenge $x$ to produce $t_{\hat{}}$, $\tau_x$, and $\mu$. %
    \item \emph{Inner Product Proof:} The \texttt{IpProof} is generated by recursively splitting vectors in half and committing to partial inner products, storing the commitments $(L_i, R_i)$. % See https://github.com/serai-dex/serai/blob/48db06f901952b24bb38d7c7e256f798f08512cd/networks/monero/ringct/bulletproofs/src/original/mod.rs#L184-L222 for T1 and T2 commitments, transcript/challenge x, t_hat, tau_x, and mu.
\end{enumerate}

|paragraph{Bulletproof+ (\texttt{plus} module)}
Bulletproof+ is provided by the \texttt{plus/} folder, primarily in \texttt{weighted\_inner\_product.rs} and \texttt{aggregate\_range\_proof.rs}.  The official Monero code slightly alters the structure of the proof to reduce proof size, while still maintaining logarithmic size.  Notable elements:
\begin{enumerate}
    \item The \emph{Weighted Inner Product} (\texttt{WipStatement}) generalizes the classic inner product.  It incorporates certain transformations by weighting vectors with powers of $y$ and rearranging final group elements. % https://github.com/serai-dex/serai/blob/48db06f901952b24bb38d7c7e256f798f08512cd/networks/monero/ringct/bulletproofs/src/plus/weighted_inner_product.rs#L15-L28
    \item The \emph{Generators} are re-labeled for Monero's usage: \texttt{g\_bold} and \texttt{h\_bold} in code, though the actual roles are reversed (Monero uses $H$ as the value basepoint and $G$ for mask derivation). % https://github.com/serai-dex/serai/blob/48db06f901952b24bb38d7c7e256f798f08512cd/networks/monero/ringct/bulletproofs/src/plus/mod.rs#L25-L35
    \item Aggregation is performed similarly with powers of $z$ and $y$, but the final step uses a more efficient \texttt{WipProof}.  
    \item The function \texttt{prove\_plus} sets up the aggregated statement, computes an $A$ commitment, then calls \texttt{WipStatement::prove} to produce the final weighted inner product proof. % See https://github.com/serai-dex/serai/blob/48db06f901952b24bb38d7c7e256f798f08512cd/networks/monero/ringct/bulletproofs/src/plus/aggregate_range_proof.rs#L186-L203 for z powers and https://github.com/serai-dex/serai/blob/48db06f901952b24bb38d7c7e256f798f08512cd/networks/monero/ringct/bulletproofs/src/plus/aggregate_range_proof.rs#L213-L234 for y powers and aggregation.
\end{enumerate}

At verification time, \texttt{aggregate\_range\_proof.rs} re-derives the same hashed challenges from the set of commitments and checks consistency with the \texttt{WipProof}.  As with the original proof, a batched approach merges all terms into a single multi-scalar multiplication check.  

\subsubsection{Core Routines and Utilities}

\paragraph{\texttt{core/mod.rs}}
Contains low-level primitives:
\begin{itemize}
    \item \texttt{multiexp} and \texttt{multiexp\_vartime}: Perform (variable-time) multi-exponentiation to accumulate the proof checks. % https://github.com/serai-dex/serai/blob/48db06f901952b24bb38d7c7e256f798f08512cd/networks/monero/ringct/bulletproofs/src/core.rs#L31-L74
    \item \texttt{challenge\_products}: Generates the partial products of the challenges $(e_i, e_i^{-1})$ for the recursive proof steps, optimizing repeated multiplications. % https://github.com/serai-dex/serai/blob/48db06f901952b24bb38d7c7e256f798f08512cd/networks/monero/ringct/bulletproofs/src/core.rs#L50-L74
\end{itemize}

These functions are the primary means of combining scalar-point pairs for cryptographic checks throughout the library.  

\paragraph{\texttt{batch\_verifier.rs}}
Implements the \texttt{BatchVerifier} structure, storing:
\begin{itemize}
    \item \texttt{original}: An accumulator for the original Bulletproof scheme. % https://github.com/serai-dex/serai/blob/48db06f901952b24bb38d7c7e256f798f08512cd/networks/monero/ringct/bulletproofs/src/batch_verifier.rs#L55-L62
    \item \texttt{plus}: An accumulator for Bulletproof+. % https://github.com/serai-dex/serai/blob/48db06f901952b24bb38d7c7e256f798f08512cd/networks/monero/ringct/bulletproofs/src/batch_verifier.rs#L64-L73
\end{itemize}

Each accumulator collects scalar multipliers for fixed and variable basepoints (respectively \texttt{G}, \texttt{H}, per-commitment bases, etc.).  Finally, \texttt{verify} executes a \emph{single} call to \texttt{vartime\_multiscalar\_mul} to confirm all queued proofs in batch.  

\paragraph{\texttt{scalar\_vector.rs} and \texttt{point\_vector.rs}}
Define lightweight wrappers around \texttt{Vec<Scalar>} and \texttt{Vec<EdwardsPoint>} to simplify:
\begin{itemize}
    \item Indexing with bounds checks (in debug mode). % https://github.com/serai-dex/serai/blob/48db06f901952b24bb38d7c7e256f798f08512cd/networks/monero/ringct/bulletproofs/src/scalar_vector.rs#L16-L26
    \item Element-wise operations: addition, subtraction, or multiplication on each vector element. % https://github.com/serai-dex/serai/blob/48db06f901952b24bb38d7c7e256f798f08512cd/networks/monero/ringct/bulletproofs/src/scalar_vector.rs#L28-L85
    \item Inner product computation: a straightforward $(\mathbf{a},\mathbf{b}) \mapsto \sum a_i b_i$. % https://github.com/serai-dex/serai/blob/48db06f901952b24bb38d7c7e256f798f08512cd/networks/monero/ringct/bulletproofs/src/scalar_vector.rs#L124-L130
\end{itemize}

These wrappers help maintain correctness by preventing mistakes in indexing or length mismatches between vectors.  

\subsubsection{Transcript and Fiat--Shamir Challenges}
Throughout the protocol, challenges ($y, z, x, \dots$) are computed by hashing various group elements and scalars with Keccak-256.  This is done via:
\begin{itemize}
    \item \texttt{keccak256\_to\_scalar(...)}: Converts a Keccak-256 hash into a \texttt{Scalar}. % https://github.com/serai-dex/serai/blob/48db06f901952b24bb38d7c7e256f798f08512cd/networks/monero/ringct/bulletproofs/src/plus/transcript.rs#L13-L17
    \item Combined with domain separators and partial transcripts (e.g.\ $\texttt{A}$, $\texttt{S}$, $\texttt{T1}$, $\texttt{T2}$ in the original scheme). % https://github.com/serai-dex/serai/blob/48db06f901952b24bb38d7c7e256f798f08512cd/networks/monero/ringct/bulletproofs/src/plus/transcript.rs#L8-L11
\end{itemize}

In \texttt{plus/transcript.rs}, Bulletproof+ uses an additional domain separation constant \\
\verb|bulletproof_plus_transcript|, plus Monero's \texttt{hash\_to\_point} trick, though its necessity is not entirely clear from the code.  

\subsubsection{Security Observations}
\begin{itemize}
    \item \emph{Cofactor clearing}: By multiplying all external inputs (\texttt{EdwardsPoint} commitments) by \texttt{INV\_EIGHT} then re-multiplying by 8, the code ensures those points lie in the primary subgroup.  This is critical to avoid torsion-based forgeries. % See https://github.com/serai-dex/serai/blob/48db06f901952b24bb38d7c7e256f798f08512cd/networks/monero/ringct/bulletproofs/src/original/mod.rs#L61-L62 for INV_EIGHT multiplication and https://github.com/serai-dex/serai/blob/48db06f901952b24bb38d7c7e256f798f08512cd/networks/monero/ringct/bulletproofs/src/batch_verifier.rs#L23-L53 for is_identity check.
    \item \emph{Batch verification correctness}: Accumulated scalars and points are carefully updated to ensure the final check equals the identity only if all aggregated proofs are valid.  If any proof is malformed, the final multi-exponentiation will yield a non-identity point. % See latter reference above.
    \item \emph{Randomness and transcript:} The crate depends on externally provided randomness (via \texttt{rand\_core} or through \texttt{OsRng} in tests).  The Fiat--Shamir heuristic is used for non-interactive challenge derivation. % See https://github.com/serai-dex/serai/blob/48db06f901952b24bb38d7c7e256f798f08512cd/networks/monero/ringct/bulletproofs/src/plus/transcript.rs for Fiat-Shamir.
\end{itemize}


% --------------------------------------------------- %
\subsection{\texttt{monero-address (v0.1.0)}}

\subsubsection{Purpose}

The \texttt{monero-address} crate provides functionality for handling Monero addresses, including standard, subaddress, integrated, and featured addresses.  It also supports encoding and decoding Monero addresses in Base58Check format.  

\subsubsection{Internal Dependencies}

\begin{itemize}
    \item \texttt{monero-io (v0.1.0)}.  
    \item \texttt{monero-primitives (v0.1.0)}.  
\end{itemize}

\subsubsection{Structure}

A standard library crate, with the corresponding entry point at \path{/wallet/address/src/lib.rs}.  

\begin{itemize}
    \item \texttt{base58check} module.  
    \item Tests at \path{/wallet/address/src/tests.rs}.  
\end{itemize}

\subsubsection{Functionality}

The \texttt{monero-address} crate defines several types of Monero addresses and provides encoding, decoding, and validation functionality.  

\paragraph{Address Types}

The \texttt{monero-address} crate defines the following address types.  

\begin{itemize}
    \item \textbf{Legacy:} A standard Monero address (public spend and view keys).  
    \item \textbf{Legacy Integrated:} A legacy address with an embedded 8-byte payment ID.  
    \item \textbf{Subaddress:} A derived address that allows the receiver to differentiate sources of funds.  
    \item \textbf{Featured Address:} An extended address format supporting subaddresses, embedded payment IDs, and a guarantee against the burning bug.  
\end{itemize}
% https://github.com/serai-dex/serai/blob/48db06f901952b24bb38d7c7e256f798f08512cd/networks/monero/wallet/address/src/lib.rs#L24-L58

Each address type is represented using the \texttt{AddressType} enum, which provides methods for checking whether an address is a subaddress, retrieving its embedded payment ID, and determining if it is guaranteed. % https://github.com/serai-dex/serai/blob/48db06f901952b24bb38d7c7e256f798f08512cd/networks/monero/wallet/address/src/lib.rs#L60-L86

\paragraph{Network Identification}

The \texttt{Network} enum defines three possible Monero network types.  

\begin{itemize}
    \item \textbf{Mainnet:} The primary Monero blockchain.  
    \item \textbf{Stagenet:} A staging network with the same rules as mainnet, used for testing.  
    \item \textbf{Testnet:} A network used to test new features before they are deployed.  
\end{itemize}
% https://github.com/serai-dex/serai/blob/48db06f901952b24bb38d7c7e256f798f08512cd/networks/monero/wallet/address/src/lib.rs#L186-L199

Each network is associated with specific address prefix bytes to distinguish them from each other. % https://github.com/serai-dex/serai/blob/48db06f901952b24bb38d7c7e256f798f08512cd/networks/monero/wallet/address/src/lib.rs#L339-L347

\paragraph{Base58Check Encoding and Decoding}

The \texttt{base58check} module implements Base58Check encoding and decoding, ensuring that Monero addresses include a checksum for error detection. % https://github.com/serai-dex/serai/blob/48db06f901952b24bb38d7c7e256f798f08512cd/networks/monero/wallet/address/src/base58check.rs#L86-L106

\begin{itemize}
    \item \texttt{encode}: Encodes a byte array into Base58Check format.  
    \item \texttt{decode}: Decodes a Base58Check string into raw bytes.  
    \item \texttt{encode\_check}: Computes a checksum and encodes a byte array into Base58Check.  
    \item \texttt{decode\_check}: Decodes a Base58Check string and verifies its checksum.  
\end{itemize}

\paragraph{Address Parsing and Generation}

Monero addresses can be created from public spend and view keys, or parsed from strings. % https://github.com/serai-dex/serai/blob/48db06f901952b24bb38d7c7e256f798f08512cd/networks/monero/wallet/address/src/lib.rs#L401-L461

\begin{itemize}
    \item \texttt{MoneroAddress::new}: Creates a new Monero address given a network, address type, and key pair.  
    \item \texttt{MoneroAddress::from\_str}: Parses an address string for a given network.  
    \item \texttt{MoneroAddress::from\_str\_with\_unchecked\_network}: Parses an address string without verifying the network.  
    \item \texttt{MoneroAddress::to\_string}: Serializes an address to a Base58Check-encoded string.  
\end{itemize}

\paragraph{Address Validation}

The crate ensures that parsed addresses meet the expected format. % https://github.com/serai-dex/serai/blob/48db06f901952b24bb38d7c7e256f798f08512cd/networks/monero/wallet/address/src/lib.rs#L419-L440

\begin{itemize}
    \item \textbf{Length validation:} Ensures addresses are the correct length based on their type.  
    \item \textbf{Checksum verification:} Ensures that decoded addresses pass the Base58Check validation.  
    \item \textbf{Key validation:} Ensures that the spend and view keys are valid Edwards25519 points.  
\end{itemize}

\subsubsection{Serialization Details}

The \texttt{monero-address} crate serializes addresses into a compact binary format before encoding them into Base58Check.  The serialization structure is as follows. % https://github.com/serai-dex/serai/blob/48db06f901952b24bb38d7c7e256f798f08512cd/networks/monero/wallet/address/src/lib.rs#L381-L399

\begin{enumerate}
    \item \textbf{Network Byte:} A single byte indicating the network and address type.
    \item \textbf{Spend Key:} A 32-byte compressed Edwards25519 public key.
    \item \textbf{View Key:} A 32-byte compressed Edwards25519 public key.
    \item \textbf{Optional Fields:}
    \begin{itemize}
        \item \textbf{Integrated Payment ID:} An 8-byte payment ID (only for integrated addresses).
        \item \textbf{Feature Flags:} A variable-length integer encoding subaddress, payment ID, and guarantee status.
    \end{itemize}
    \item \textbf{Checksum:} A 4-byte Keccak-256 hash of the serialized data.
\end{enumerate}
% See https://github.com/serai-dex/serai/blob/48db06f901952b24bb38d7c7e256f798f08512cd/networks/monero/wallet/address/src/lib.rs#L381-L399 and https://github.com/serai-dex/serai/blob/48db06f901952b24bb38d7c7e256f798f08512cd/networks/monero/wallet/address/src/base58check.rs#L86-L91 for checksum details.

\subsubsection{Testing}

The \texttt{monero-address} crate includes comprehensive tests to validate the correctness of its encoding, decoding, and parsing functions. % https://github.com/serai-dex/serai/blob/48db06f901952b24bb38d7c7e256f798f08512cd/networks/monero/wallet/address/src/tests.rs

\begin{itemize}
    \item \textbf{Base58Check Encoding Tests:} Ensures that encoded addresses correctly round-trip through decoding.  
    \item \textbf{Standard Address Tests:} Validates correct parsing and serialization of standard Monero addresses.  
    \item \textbf{Integrated Address Tests:} Ensures proper handling of embedded payment IDs.  
    \item \textbf{Subaddress Tests:} Validates the ability to generate and recognize subaddresses.  
    \item \textbf{Featured Address Tests:} Ensures correct parsing and serialization of extended features.  
    \item \textbf{Vector-Based Tests:} Uses predefined address vectors to ensure consistency with expected values.  
\end{itemize}

The test suite ensures that all address types are handled correctly and provides coverage for both expected and edge cases.  


% --------------------------------------- %
\subsection{monero-clsag (v0.1.0)}
The \texttt{monero-clsag} crate implements Compact Linkable Spontaneous Anonymous Group (CLSAG) signatures and provides a FROST-inspired threshold signing mechanism.  
The implementation consists of two main components:

\subsubsection{Core CLSAG Implementation}
\begin{description}
\item[ClsagContext] \hfill  % https://github.com/serai-dex/serai/blob/48db06f901952b24bb38d7c7e256f798f08512cd/networks/monero/ringct/clsag/src/lib.rs#L65-L72

Holds the context needed for signing:
\begin{itemize}
\item A commitment opening (mask and amount)
\item Selected decoy positions and ring
\end{itemize}

\item[Clsag Signature] \hfill % https://github.com/serai-dex/serai/blob/48db06f901952b24bb38d7c7e256f798f08512cd/networks/monero/ringct/clsag/src/lib.rs#L222-L231

A signature consisting of:
\begin{itemize}
\item D: Difference of commitment randomness scaling the key image generator
\item s: Vector of responses for each ring member
\item c1: First challenge in the ring
\end{itemize}

\item[Core Algorithm] \hfill  % https://github.com/serai-dex/serai/blob/48db06f901952b24bb38d7c7e256f798f08512cd/networks/monero/ringct/clsag/src/lib.rs#L96-L220

The core signing/verification algorithm:
\begin{enumerate}
\item Takes ring members, key image I, pseudo-out P, and message m
\item For signing:
  \begin{itemize}
  \item Generates random nonces for the real signer
  \item Computes the key image generator from the public key
  \item Calculates initial commitments A and AH
  \end{itemize}
\item For both signing and verifying:
  \begin{itemize}
  \item Computes challenges using Keccak256
  \item Performs ring calculations over the Ed25519 group
  \item Validates signature components
  \end{itemize}
\end{enumerate}
\end{description}

\subsubsection{FROST-Inspired Threshold Signing}
\begin{description}
\item[ClsagMultisig] \hfill % https://github.com/serai-dex/serai/blob/48db06f901952b24bb38d7c7e256f798f08512cd/networks/monero/ringct/clsag/src/multisig.rs#L115-L137

Implements threshold signing for CLSAG:
\begin{itemize}
\item Uses FROST key generation and coordination
\item Extends CLSAG to support threshold key images
\item Preserves CLSAG's linkability property
\end{itemize}

\item[Key Components] \hfill  % See https://github.com/serai-dex/serai/blob/48db06f901952b24bb38d7c7e256f798f08512cd/networks/monero/ringct/clsag/src/multisig.rs#L57-L68 and https://github.com/serai-dex/serai/blob/48db06f901952b24bb38d7c7e256f798f08512cd/networks/monero/ringct/clsag/src/multisig.rs#L86-L97
\begin{itemize}
\item \textbf{ClsagMultisigMaskSender}: A channel for communicating commitment masks.  
\item \textbf{ClsagAddendum}: Key image shares produced during signing.  
\item \textbf{Interim}: Stores partial signature data during the protocol.  
\end{itemize}

\item[Protocol Flow] \hfill  % https://github.com/serai-dex/serai/blob/48db06f901952b24bb38d7c7e256f798f08512cd/networks/monero/ringct/clsag/src/multisig.rs#L172-L245
\begin{enumerate}
\item Initialize with a transcript and CLSAG context
\item Share and aggregate key image contributions
\item Generate and share nonces via FROST
\item Produce partial signatures
\item Verify shares and construct the final signature
\end{enumerate}

\item[Security Properties]  \hfill % https://github.com/serai-dex/serai/blob/48db06f901952b24bb38d7c7e256f798f08512cd/networks/monero/ringct/clsag/src/multisig.rs#L315-L378

\begin{itemize}
\item Maintains unforgeability of CLSAG
\item Preserves one-time key image property
\item Requires a threshold of signers to complete
\item Uses transcript-based challenge generation
\end{itemize}
\end{description}

The implementation leverages the \texttt{curve25519-dalek} library for Ed25519 operations and includes comprehensive test vectors covering both single-signer and threshold signing scenarios.  
All operations maintain constant-time properties to prevent timing side-channels.  


% -------------------------------------- %
\subsection{monero-simple-request-rpc (v0.1.0)}

\paragraph{Purpose}
The \texttt{monero-simple-request-rpc} crate provides an HTTP(S)-based transport layer for performing RPC calls to a Monero daemon.  This crate is designed to be minimal and efficient, avoiding dependencies on larger libraries like \texttt{reqwest}, while supporting both authenticated and unauthenticated connections.  The crate implements the \texttt{Rpc} trait from the \texttt{monero-rpc} crate.

\paragraph{Internal Dependencies}
\begin{itemize}
    \item \texttt{monero-rpc (v0.1.0)}: Provides the \texttt{Rpc} trait and associated error types.  
    \item \texttt{simple-request}: Handles HTTP(S) request and response processing.  
    \item \texttt{digest-auth}: Implements HTTP Digest Authentication for authenticated connections.  
    \item \texttt{tokio}: Used for asynchronous operations and synchronization primitives.  
    \item \texttt{hex}: Used for encoding and decoding hexadecimal data.
\end{itemize}

\paragraph{Structure}
The crate is a standard library crate with its entry point at \path{/rpc/simple-request/src/lib.rs}.  The main component is the \texttt{SimpleRequestRpc} struct, which encapsulates the connection logic, authentication handling, and request processing.  The implementation includes:
\begin{itemize}
    \item \textbf{Authentication Handling}: Supports both authenticated and unauthenticated RPC connections.  Authentication uses HTTP Digest Authentication when credentials are provided in the URL. % See Authentication enum at https://github.com/serai-dex/serai/blob/48db06f901952b24bb38d7c7e256f798f08512cd/networks/monero/rpc/simple-request/src/lib.rs#L20-L33
    \item \textbf{Thread Safety}: Uses \texttt{Arc<Mutex>} for safe concurrent access to authentication state and nonce management. % See Arc<Mutex> usage at https://github.com/serai-dex/serai/blob/48db06f901952b24bb38d7c7e256f798f08512cd/networks/monero/rpc/simple-request/src/lib.rs#L27-L32
    \item \textbf{Request Processing}: Sends RPC requests and processes responses, including retries for stale authentication challenges. % See inner_post in https://github.com/serai-dex/serai/blob/48db06f901952b24bb38d7c7e256f798f08512cd/networks/monero/rpc/simple-request/src/lib.rs#L130-L280
    \item \textbf{Timeout Management}: Allows for customizable request timeouts, defaulting to 30 seconds. % See declaration at https://github.com/serai-dex/serai/blob/48db06f901952b24bb38d7c7e256f798f08512cd/networks/monero/rpc/simple-request/src/lib.rs#L18 and usage in https://github.com/serai-dex/serai/blob/48db06f901952b24bb38d7c7e256f798f08512cd/networks/monero/rpc/simple-request/src/lib.rs#L62-L77 and after https://github.com/serai-dex/serai/blob/48db06f901952b24bb38d7c7e256f798f08512cd/networks/monero/rpc/simple-request/src/lib.rs#L129
\end{itemize}
% See Struct at https://github.com/serai-dex/serai/blob/48db06f901952b24bb38d7c7e256f798f08512cd/networks/monero/rpc/simple-request/src/lib.rs#L35-L43 and implementation at https://github.com/serai-dex/serai/blob/48db06f901952b24bb38d7c7e256f798f08512cd/networks/monero/rpc/simple-request/src/lib.rs#L45-L127

\paragraph{Detailed Functionality}
\begin{description}
    \item[\texttt{SimpleRequestRpc::new}]  
    Creates a new \texttt{SimpleRequestRpc} instance with a default timeout of 30 seconds.  If the provided URL contains credentials, they are parsed and used for Digest Authentication. % https://github.com/serai-dex/serai/blob/48db06f901952b24bb38d7c7e256f798f08512cd/networks/monero/rpc/simple-request/src/lib.rs#L62-L68

    \item[\texttt{SimpleRequestRpc::with\_custom\_timeout}]  
    Similar to \texttt{new}, but allows specifying a custom timeout duration.  It parses the URL for authentication details and initializes the appropriate client. % https://github.com/serai-dex/serai/blob/48db06f901952b24bb38d7c7e256f798f08512cd/networks/monero/rpc/simple-request/src/lib.rs#L70-L126

    \item[\texttt{SimpleRequestRpc::post}]  
    Implements the \texttt{post} method from the \texttt{Rpc} trait.  It sends a POST request to the specified RPC route with the provided payload, handling retries for authentication failures due to stale challenges. % https://github.com/serai-dex/serai/blob/48db06f901952b24bb38d7c7e256f798f08512cd/networks/monero/rpc/simple-request/src/lib.rs#L283-L293

    \item[\texttt{SimpleRequestRpc::inner\_post}]  
    An internal method that performs the actual request logic.  It handles both authenticated and unauthenticated requests and processes the response to extract the body or detect errors. % https://github.com/serai-dex/serai/blob/48db06f901952b24bb38d7c7e256f798f08512cd/networks/monero/rpc/simple-request/src/lib.rs#L130-L135

    \item[\texttt{SimpleRequestRpc::digest\_auth\_challenge}]  
    Extracts and parses the \texttt{WWW-Authenticate} header from the server response to initialize or update the Digest Authentication state. % https://github.com/serai-dex/serai/blob/48db06f901952b24bb38d7c7e256f798f08512cd/networks/monero/rpc/simple-request/src/lib.rs#L46-L60

    \item[\texttt{Authentication Retry Logic}]  
    Implements a two-attempt retry mechanism for authentication failures, with specific handling for stale nonces and connection errors.  This retry logic is important for maintaining connection stability during authentication state changes. % https://github.com/serai-dex/serai/blob/48db06f901952b24bb38d7c7e256f798f08512cd/networks/monero/rpc/simple-request/src/lib.rs#L253-L275
\end{description}

\paragraph{Authentication State Management}
The authentication state is managed through an internal \texttt{Authentication} enum with two variants:
\begin{itemize}
    \item \texttt{Unauthenticated}: Contains a single \texttt{Client} instance for all requests.  
    \item \texttt{Authenticated}: Contains a username, password, and a thread-safe connection state managed through \texttt{Arc<Mutex<(Option<(WwwAuthenticateHeader, u64)>, Client)>>}.
\end{itemize}

\paragraph{Testing}
Integration tests are located in \path{/rpc/simple-request/tests/tests.rs}.  These tests validate:
\begin{itemize}
    \item RPC functionality for retrieving blockchain height, blocks, and hardfork versions.  
    \item Block generation with different amounts of blocks (1 and 5).  
    \item Authentication handling for valid and stale challenges.  
    \item Decoy RPC functionality and output distribution queries.  
    \item Sequential test execution using \texttt{LazyLock<Mutex<()>>} to prevent test interference.  
    \item Error handling for various failure scenarios.
\end{itemize}
% https://github.com/serai-dex/serai/blob/48db06f901952b24bb38d7c7e256f798f08512cd/networks/monero/rpc/simple-request/tests/tests.rs#L16-L65

\paragraph{Error Handling}
The implementation provides comprehensive error handling for:
\begin{itemize}
    \item Connection failures with detailed error messages.  
    \item Authentication failures, including stale nonce detection.  
    \item Invalid or malformed responses.  
    \item Timeout errors with a configurable duration.  
    \item URL parsing and validation errors.
\end{itemize}

\paragraph{Summary}
The \texttt{monero-simple-request-rpc} crate is a lightweight, efficient solution for connecting to a Monero daemon via HTTP(S).  It adheres to the \texttt{Rpc} trait interface, enabling seamless integration with the broader \texttt{monero-rpc} ecosystem.  Its design prioritizes thread safety, robust authentication handling, and comprehensive error management, making it suitable for both development and production use cases.



% --------------------------------------------- %
\subsection{\texttt{monero-borromean (v0.1.0)}}
\label{sec:monero_borromean}

This crate provides a Borromean ring signature--based approach to a 64-bit range proof within the Monero protocol. It defines two primary types:

\begin{itemize}
  \item \texttt{BorromeanSignatures} % https://github.com/serai-dex/serai/blob/48db06f901952b24bb38d7c7e256f798f08512cd/networks/monero/ringct/borromean/src/lib.rs#L18-L28
  \item \texttt{BorromeanRange} % https://github.com/serai-dex/serai/blob/48db06f901952b24bb38d7c7e256f798f08512cd/networks/monero/ringct/borromean/src/lib.rs#L74-L79
\end{itemize}

Both types are specialized to operate with 64-bit range proofs and rely on Curve25519-based group operations to construct and verify Borromean ring signatures.

\paragraph{Data Structures.}
\begin{itemize}
  \item \texttt{BorromeanSignatures}
  \begin{itemize}
    \item Stores exactly 64 Borromean ring signatures in two arrays of scalars, \texttt{s0} and \texttt{s1}, each of length 64.
    \item Holds the final challenge scalar \texttt{ee}.
    \item Implements custom \texttt{read} and \texttt{write} methods using the \texttt{monero-io} crate to produce or consume canonical byte encodings. % https://github.com/serai-dex/serai/blob/48db06f901952b24bb38d7c7e256f798f08512cd/networks/monero/ringct/borromean/src/lib.rs#L31-L49
    \item Its \texttt{verify} function checks correctness by: % https://github.com/serai-dex/serai/blob/48db06f901952b24bb38d7c7e256f798f08512cd/networks/monero/ringct/borromean/src/lib.rs#L51-L72
      \begin{enumerate}[label=(\alph*)]
        \item Performing an iterative double-scalar multiplication on each signature component. % See section referenced above, especially https://github.com/serai-dex/serai/blob/48db06f901952b24bb38d7c7e256f798f08512cd/networks/monero/ringct/borromean/src/lib.rs#L56-L60 and https://github.com/serai-dex/serai/blob/48db06f901952b24bb38d7c7e256f798f08512cd/networks/monero/ringct/borromean/src/lib.rs#L62-L66
        \item Accumulating hash outputs in a transcript. % https://github.com/serai-dex/serai/blob/48db06f901952b24bb38d7c7e256f798f08512cd/networks/monero/ringct/borromean/src/lib.rs#L67
        \item Ensuring the final hash matches \texttt{ee}. % https://github.com/serai-dex/serai/blob/48db06f901952b24bb38d7c7e256f798f08512cd/networks/monero/ringct/borromean/src/lib.rs#L70
      \end{enumerate}
  \end{itemize}

  \item \texttt{BorromeanRange} % https://github.com/serai-dex/serai/blob/48db06f901952b24bb38d7c7e256f798f08512cd/networks/monero/ringct/borromean/src/lib.rs#L74-L79
  \begin{itemize}
    \item Contains a \texttt{BorromeanSignatures} field and 64 \texttt{bit\_commitments}, each an \texttt{EdwardsPoint}.
    \item Like \texttt{BorromeanSignatures}, it has \texttt{read} and \texttt{write} functions for serialized I/O. % https://github.com/serai-dex/serai/blob/48db06f901952b24bb38d7c7e256f798f08512cd/networks/monero/ringct/borromean/src/lib.rs#L82-L94
    \item Its \texttt{verify} method checks that: % https://github.com/serai-dex/serai/blob/48db06f901952b24bb38d7c7e256f798f08512cd/networks/monero/ringct/borromean/src/lib.rs#L96-L111
      \begin{enumerate}[label=(\alph*)]
        \item The sum of all \texttt{bit\_commitments} equals the provided \texttt{commitment}. % https://github.com/serai-dex/serai/blob/48db06f901952b24bb38d7c7e256f798f08512cd/networks/monero/ringct/borromean/src/lib.rs#L99-L101
        \item For each bit, subtracting \(\texttt{H\_pow\_2}[i]\) from \texttt{bit\_commitments}[i] produces a second set of points. % https://github.com/serai-dex/serai/blob/48db06f901952b24bb38d7c7e256f798f08512cd/networks/monero/ringct/borromean/src/lib.rs#L103-L108
        \item The embedded \texttt{BorromeanSignatures} correctly verifies under those two sets of points. % https://github.com/serai-dex/serai/blob/48db06f901952b24bb38d7c7e256f798f08512cd/networks/monero/ringct/borromean/src/lib.rs#L110
      \end{enumerate}
  \end{itemize}
\end{itemize}

\paragraph{Purpose and Scope.} 
This crate provides an older, yet still relevant, Monero-based range proof system.  While newer systems like Bulletproofs have largely supplanted Borromean proofs in many contexts, the \texttt{monero-borromean} crate is crucial for maintaining backwards compatibility with legacy proofs and for verifying archived transactions that used Borromean range proofs. It accepts both \texttt{std} and \texttt{no\_std} environments, and depends on a few of the core Monero crates:
\begin{itemize}
  \item \texttt{monero-io} for reading and writing typed data.
  \item \texttt{monero-generators} for fixed-base scalar multiplication with \(H^2\).
  \item \texttt{monero-primitives} for cryptographic primitives such as \texttt{UnreducedScalar}.
\end{itemize}

\paragraph{Verification Logic.}
\begin{enumerate}[label=(\roman*)]
  \item \texttt{BorromeanSignatures::verify} loops over 64 “bits” (i.e., the separate ring signatures). For each bit: % https://github.com/serai-dex/serai/blob/48db06f901952b24bb38d7c7e256f798f08512cd/networks/monero/ringct/borromean/src/lib.rs#L51-L72
  \begin{itemize}
    \item It checks the challenge scalar by computing intermediate elliptic-curve multiplications. % https://github.com/serai-dex/serai/blob/48db06f901952b24bb38d7c7e256f798f08512cd/networks/monero/ringct/borromean/src/lib.rs#L56-L59
    \item It accumulates partial results into a 2048-byte transcript. Z% https://github.com/serai-dex/serai/blob/48db06f901952b24bb38d7c7e256f798f08512cd/networks/monero/ringct/borromean/src/lib.rs#L67
  \end{itemize}
  \item Finally, if the scalar derived from hashing the transcript matches the stored scalar \texttt{ee}, the signature is valid % https://github.com/serai-dex/serai/blob/48db06f901952b24bb38d7c7e256f798f08512cd/networks/monero/ringct/borromean/src/lib.rs#L70
\end{enumerate}

\paragraph{Conclusion.} 
The \texttt{monero-borromean (v0.1.0)} crate implements older-style Borromean range proofs used historically in Monero transactions.  It is straightforward, cleanly implemented, line 3 enforces \texttt{missing\_docs} denial to encourage thorough documentation and offers \texttt{no\_std} support.  Its modular APIs for reading, writing, and verifying signatures make it an easy integration point for other projects that might require Borromean signature functionality or need to validate historical transactions.




% --------------------------------------- %
\subsection{monero-mlsag (v0.1.0)}

The \texttt{monero-mlsag} crate provides functionality for Multilayered Linkable Spontaneous Anonymous Group (MLSAG) signatures as used in the Monero protocol.  
The crate is organized around two primary data structures: \texttt{RingMatrix} and \texttt{Mlsag}.

\subsubsection{Ring Matrix}

\begin{description}
\item[Structure] \hfill 

The \texttt{RingMatrix} type encapsulates a matrix of Edwards points used for MLSAG verification:  
\begin{itemize}
\item Internal representation: \texttt{Vec<Vec<EdwardsPoint>>}.  
\item Zeroizes on drop for security.  
\item Must contain at least 2 ring members.  
\item All members must have equal length.  
\end{itemize}
% https://github.com/serai-dex/serai/blob/48db06f901952b24bb38d7c7e256f798f08512cd/networks/monero/ringct/mlsag/src/lib.rs#L42-L46

\item[Construction Methods] \hfill 
\begin{enumerate}
\item \texttt{new}: Creates a ring matrix from a pre-formatted vector of vectors.  
  \begin{itemize}
  \item Validates matrix dimensions.  
  \item Ensures minimum ring size of 2.  
  \item Ensures consistent member lengths.  
  \end{itemize}
% https://github.com/serai-dex/serai/blob/48db06f901952b24bb38d7c7e256f798f08512cd/networks/monero/ringct/mlsag/src/lib.rs#L49-L64

\item \texttt{individual}: Constructs a ring matrix for single output verification.  
  \begin{itemize}
  \item Takes a ring of \texttt{[EdwardsPoint; 2]} arrays.  
  \item Takes a pseudo-output point.  
  \item Subtracts the pseudo-output from the second column.  
  \end{itemize}
\end{enumerate}
% https://github.com/serai-dex/serai/blob/48db06f901952b24bb38d7c7e256f798f08512cd/networks/monero/ringct/mlsag/src/lib.rs#L66-L76

\item[Utilities] \hfill 
\begin{itemize}
\item \texttt{members()}: Returns the count of ring members. % https://github.com/serai-dex/serai/blob/48db06f901952b24bb38d7c7e256f798f08512cd/networks/monero/ringct/mlsag/src/lib.rs#L83-L86
\item \texttt{member\_len()}: Returns the length of each member vector. % https://github.com/serai-dex/serai/blob/48db06f901952b24bb38d7c7e256f798f08512cd/networks/monero/ringct/mlsag/src/lib.rs#L83-L86
\item \texttt{iter()}: Provides an iterator over matrix members as slices. % https://github.com/serai-dex/serai/blob/48db06f901952b24bb38d7c7e256f798f08512cd/networks/monero/ringct/mlsag/src/lib.rs#L78-L81
\end{itemize}
\end{description}

\subsubsection{MLSAG Signature}

\begin{description}
\item[Structure] \hfill 

The \texttt{Mlsag} type represents a complete MLSAG signature:
\begin{itemize}
\item \texttt{ss}: A matrix of response scalars (\texttt{Vec<Vec<Scalar>>}).  
\item \texttt{cc}: A challenge scalar.  
\item Implements zeroization for security.  
\end{itemize}
% https://github.com/serai-dex/serai/blob/48db06f901952b24bb38d7c7e256f798f08512cd/networks/monero/ringct/mlsag/src/lib.rs#L98-L103

\item[Serialization] \hfill

Provides binary serialization methods.  
\begin{itemize}
\item \texttt{write}: Serializes to a writer. % https://github.com/serai-dex/serai/blob/48db06f901952b24bb38d7c7e256f798f08512cd/networks/monero/ringct/mlsag/src/lib.rs#L106-L112
  \begin{itemize}
  \item Writes the \texttt{ss} matrix elements.  
  \item Writes the \texttt{cc} challenge.  
  \end{itemize}
\item \texttt{read}: Deserializes from a reader. % https://github.com/serai-dex/serai/blob/48db06f901952b24bb38d7c7e256f798f08512cd/networks/monero/ringct/mlsag/src/lib.rs#L114-L122
  \begin{itemize}
  \item Takes the expected mixin count.  
  \item Takes the expected width of the \texttt{ss} matrix.  
  \item Reconstructs the signature structure.  
  \end{itemize}
\end{itemize}

\item[Verification] \hfill 

The \texttt{verify} method validates an MLSAG signature.  
\begin{enumerate}
\item Input validation.  
  \begin{itemize}
  \item Validates that the key image count matches the ring member length minus 1.  
  \item Ensures consistent matrix dimensions.  
  \item Validates key image properties (non-identity, torsion-free).  
  \end{itemize}
% https://github.com/serai-dex/serai/blob/48db06f901952b24bb38d7c7e256f798f08512cd/networks/monero/ringct/mlsag/src/lib.rs#L124-L186

\item Challenge reconstruction.  
  \begin{itemize}
  \item Maintains a message buffer for hash computation. % https://github.com/serai-dex/serai/blob/48db06f901952b24bb38d7c7e256f798f08512cd/networks/monero/ringct/mlsag/src/lib.rs#L137
  \item Iterates through ring members and key images. % https://github.com/serai-dex/serai/blob/48db06f901952b24bb38d7c7e256f798f08512cd/networks/monero/ringct/mlsag/src/lib.rs#L150-L179
  \item Computes \(L = sG + (c_i \times P)\) for each entry. % https://github.com/serai-dex/serai/blob/48db06f901952b24bb38d7c7e256f798f08512cd/networks/monero/ringct/mlsag/src/lib.rs#L157
  \item For linkable layers, computes \(R = s \times \mathrm{Hp}(P) + (c_i \times I)\). % https://github.com/serai-dex/serai/blob/48db06f901952b24bb38d7c7e256f798f08512cd/networks/monero/ringct/mlsag/src/lib.rs#L171
  \item Updates the challenge using Keccak256. % https://github.com/serai-dex/serai/blob/48db06f901952b24bb38d7c7e256f798f08512cd/networks/monero/ringct/mlsag/src/lib.rs#L176
  \end{itemize}

\item Final verification.  
  \begin{itemize}
  \item Checks that the reconstructed challenge matches the signature. % https://github.com/serai-dex/serai/blob/48db06f901952b24bb38d7c7e256f798f08512cd/networks/monero/ringct/mlsag/src/lib.rs#L181-L184
  \item Returns a \texttt{Result} indicating validity.  
  \end{itemize}
\end{enumerate}
\end{description}

\subsubsection{Aggregate Ring Matrix Builder}

\begin{description}
\item[Purpose] \hfill

The \texttt{AggregateRingMatrixBuilder} facilitates the construction of ring matrices for aggregate signatures.  
\begin{itemize}
\item Manages key ring vectors.  
\item Tracks amount commitments.  
\item Handles pseudo-output calculations.  
\end{itemize}
% https://github.com/serai-dex/serai/blob/48db06f901952b24bb38d7c7e256f798f08512cd/networks/monero/ringct/mlsag/src/lib.rs#L188-L196

\item[Construction] \hfill 

Created with transaction outputs and a fee. % https://github.com/serai-dex/serai/blob/48db06f901952b24bb38d7c7e256f798f08512cd/networks/monero/ringct/mlsag/src/lib.rs#L199-L208
\begin{itemize}
\item Takes a slice of output commitment points.  
\item Takes the fee amount as a \texttt{u64}.  
\item Computes the initial sum of outputs. % https://github.com/serai-dex/serai/blob/48db06f901952b24bb38d7c7e256f798f08512cd/networks/monero/ringct/mlsag/src/lib.rs#L206
\end{itemize}

\item[Ring Addition] \hfill 

The \texttt{push\_ring} method builds the matrix incrementally. % https://github.com/serai-dex/serai/blob/48db06f901952b24bb38d7c7e256f798f08512cd/networks/monero/ringct/mlsag/src/lib.rs#L210-L229
\begin{itemize}
\item Validates ring dimensions.  
\item Separates key and amount components.  
\item Updates running sums.  
\end{itemize}

\item[Finalization] \hfill 

The \texttt{build} method produces the final \texttt{RingMatrix}. % https://github.com/serai-dex/serai/blob/48db06f901952b24bb38d7c7e256f798f08512cd/networks/monero/ringct/mlsag/src/lib.rs#L231-L237
\begin{itemize}
\item Combines key and amount components.  
\item Validates the final matrix structure.  
\item Returns the complete ring matrix.  
\end{itemize}
\end{description}

\subsubsection{Error Handling}

The crate defines the \texttt{MlsagError} enum for various failure modes.  
\begin{itemize}
\item \texttt{InvalidRing}: Ring size or structure issues.  
\item \texttt{InvalidAmountOfKeyImages}: Incorrect key image count.  
\item \texttt{InvalidSs}: Response matrix dimension mismatch.  
\item \texttt{InvalidKeyImage}: Invalid key image properties.  
\item \texttt{InvalidCi}: Challenge verification failure.  
\end{itemize}
% https://github.com/serai-dex/serai/blob/48db06f901952b24bb38d7c7e256f798f08512cd/networks/monero/ringct/mlsag/src/lib.rs#L21-L40




