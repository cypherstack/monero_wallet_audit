%
% This template provides an in-depth structure for writing a formal protocol specification.
% It incorporates:
%  - Extended commentary for each section of a protocol spec
%  - BibTeX for reference management
%  - Glossaries and an index for terminology and ease of navigation
%
% To compile:
%  1) pdflatex specs_template.tex
%  2) bibtex specs_template.aux (depending on your LaTeX environment, or just "bibtex specs_template")
%  3) makeglossaries specs_template
%  4) pdflatex specs_template.tex (repeat as needed to resolve references/index)
%
%-------------------------------------------------------------------------------------------%

\documentclass[12pt,a4paper]{article}

%-------------------- Package Imports --------------------%
\usepackage[utf8]{inputenc}
\usepackage[T1]{fontenc}
\usepackage{lmodern}               % Improved font rendering
\usepackage{geometry}              % Page geometry
\geometry{margin=1in}             % Set margins to your preference
\usepackage{hyperref}              % Clickable links and refs
\usepackage{graphicx}              % For images
\usepackage{enumitem}              % Better control of list environments
\usepackage{xcolor}                % Color text if needed
\usepackage{makeidx}               % For creating an index
\makeindex                         % Initialize index creation
\usepackage[style=alphabetic]{biblatex} % BibLaTeX for references
\addbibresource{references.bib}    % External .bib file for references
\usepackage[toc, acronym]{glossaries}   % Glossaries and acronyms
\makeglossaries                    % Initialize glossary creation

%-------------------- Example Glossary Entries --------------------%
\newglossaryentry{protocol}{
  name={protocol},
  description={A set of rules and structures governing communication between two or more parties}
}
\newglossaryentry{specification}{
  name={specification},
  description={A detailed, standardized description of the requirements and behaviors of a system}
}
\newglossaryentry{p2p}{
  name={P2P},
  description={Peer-to-peer, a decentralized communications model in which each party can act as both client and server}
}

%-------------------- Title, Author, Date --------------------%
\title{\textbf{Protocol Specification Template}}
\author{Your Name \\ \texttt{your.email@domain.com}}
\date{\today}

%===================== DOCUMENT START =====================%
\begin{document}

\maketitle

\tableofcontents
\clearpage

%-------------------- Section: Introduction / Scope --------------------%
\section{Introduction}

This specification defines parts of the Monero protocol. 

\subsection{Scope \& Goals}
\begin{itemize}[noitemsep]
  \item \textbf{Purpose}: Summarize the protocol’s objective and the 
  primary use-cases it covers.
  \item \textbf{Intended Audience}: Identify who should implement or 
  reference this specification (e.g., library maintainers, client 
  developers).
  \item \textbf{Goals and Non-Goals}: Clearly outline what the protocol 
  is designed to achieve, and highlight any features explicitly 
  out-of-scope.
\end{itemize}

%-------------------- Section: Terminology & Definitions --------------------%
\section{Notation}
Provide a clear and consistent notation
used throughout this specification.


\section{Definitions}
Provide clear and consistent definitions of concepts used throughout this specification.

\begin{itemize}[noitemsep]
  \item \textbf{Key Terms}: Distinct words or phrases with specific 
  technical meanings. 
  \item \textbf{Reference to Glossary}: Ensure each term is cross-referenced 
  to your \texttt{glossaries} entries or an appendix.
\end{itemize}

\textit{Example: see the \texttt{glossaries} package for automatically 
generating a glossary of terms.}


%-------------------- Section: Data Structures & Message Formats --------------------%
\section{Data Structures and Message Formats}
This section is critical for implementers, as it defines exactly how data 
is encoded, transmitted, and parsed.
\begin{itemize}[noitemsep]
  \item \textbf{Packet / Message Layout}: Describe each field’s purpose, 
  size, data type, and any relevant constraints.
  \item \textbf{Encoding and Serialization}: Specify whether the protocol 
  uses JSON, binary, Protobuf, or a custom encoding. Include examples if 
  necessary.
  \item \textbf{Field Ordering and Endianness}: Clearly define how multi-byte 
  fields are laid out and how they should be read (e.g., big-endian).
\end{itemize}
\appendix
% \section{Appendix: Implementation Examples}
% Use this appendix to show sample code snippets, additional diagrams, or 
% test vectors.

\section{Appendix: Glossary and Index Usage}
Below is a demonstration of how to print a glossary of terms mentioned in 
this document, as well as an index.

% Print the glossary
\clearpage
\printglossary[title=Glossary of Terms]

\clearpage
\printindex

\end{document}
